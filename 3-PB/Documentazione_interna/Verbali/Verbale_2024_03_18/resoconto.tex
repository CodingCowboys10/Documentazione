\section{Resoconto} \label{sec:resoconto}
\subsection{Discussioni} \label{subsec:resdiscussione}
\begin{enumerate}
    \item La riunione inizia con una discussione globale sullo stato attuale del Minimum Viable Product. Sono state create le Server Actions che andranno a sostituire le API route, per ottenere un maggiore stato di sicurezza dell'applicativo. La parte della chat ora soddisfa quasi tutti i requisiti obbligatori, oltre a parte dei requisiti desiderabili individuati. È da rivedere il back-end relativo alla chat e alle operazioni su database secondo l'architettura e i pattern adottati. Interventi migliorativi sono attesi sul prompt dato al LLM e sulle query di ordinamento delle sessioni attive nel sistema. La discussione termina con l'esposizione di Menon Giovanni dei providers implementati per la gestione di tema, modello, chat e sessioni di conversazione.
    \item La discussione verte sulle modifiche da apportare ad alcuni casi d'uso, dove emerge la necessità di rivedere la descrizione di essi. In particolare, "eliminazione chat history" ed "eliminazione sessione" saranno meglio descritte come "eliminazione di una sessione di conversazione" ed "eliminazione di tutte le sessioni attive". Questo intervento alla descrizione di questi casi d'uso è necessario per renderli non ambigui e coerenti alle funzionalità del prodotto.
    \item La riunione continua con una discussione sul tempo necessario per terminare il progetto. La data inizialmente preventivata del 8 aprile sembra essere ampiamente coerente allo stato dei lavori. Nella consegna del progetto, sarà ultimato un MVP che soddisfa tutti i requisiti obbligatori, ma anche parte dei desiderabili.\\
    È però necessaria una votazione per chiarificare definitivamente la volontà o meno di continuare con la CA. L'esito di una votazione positiva avrebbe impatti sul termine effettivo del progetto, prolungando anche l'impegno dell'azienda proponente in esso.
    \item L'incontro termina con una panoramica sullo stato dei documenti. Le Norme\_di\_progetto sono state aggiornate, così come la Specifica\_architetturale. Corposi interventi sono comunque previsti per quest'ultimo, così come è atteso un aggiornamento del Piano\_di\_progetto e del Piano\_di\_qualifica.\\
    Importanti avanzamenti sono attesi anche per quanto riguarda la campagna di testing.
\end{enumerate}

\subsection{Votazioni} \label{subsec:resvotazione}
\begin{enumerate}
    \item È avvenuta una votazione sulla volontà di protrarre la durata del progetto, così da prendere parte anche alla Customer Acceptance. Di seguito l'esito:
\end{enumerate}
\begingroup
    \setlength{\tabcolsep}{10pt}
    \renewcommand{\arraystretch}{1.5}
    \rowcolors{2}{oddrow}{evenrow}
    \begin{tabularx}{\textwidth}{| l | l | X |}
         \hline
         \rowcolor{headerrow}\textbf{\textcolor{white}{Proposta}} & \textbf{\textcolor{white}{Sommario}} & \textbf{\textcolor{white}{Mittente}} \\
         \hline
         Terminare con la PB. & 100\%  & Andrea Cecchin, Marco Dolzan, Francesco Ferraioli, Francesco Giacomuzzo, Leonardo Lago, Giovanni Menon, Anna Nordio.\\
         \hline
         Effettuare la CA. & 0\% &  \\
         \hline
         Astenuti & 0\% & \\
         \hline
    \end{tabularx}
\endgroup

\subsection{Azioni da intraprendere}
{
    \setlength{\tabcolsep}{10pt}
            \renewcommand{\arraystretch}{1.5}
            \rowcolors{2}{oddrow}{evenrow}
            \begin{xltabular}{\textwidth}{| l | l | l | X |}
                 \hline
                 \rowcolor{headerrow}\textbf{\textcolor{white}{Codice issue}} & \textbf{\textcolor{white}{Assegnatario}} & \textbf{\textcolor{white}{Scadenza}} & \textbf{\textcolor{white}{Descrizione}} \\
                 \hline
                 CC-252 & Andrea Cecchin & 2024/03/20 & Aggiornamento UC-25 e UC-27.\\
                 \hline
                 CC-247 & Andrea Cecchin & 2024/03/20 & Stesura verbale riunione 2024/03/18.\\
                 \hline
                 CC-205 & Giovanni Menon & 2024/03/24 & Aggiornamento sezione architettura di sistema della Specifica\_architetturale.\\
                 \hline
                 CC-250 & Progettisti & 2024/03/24 & Aggiornamento sezione architettura di back-end della Specifica\_architetturale.\\
                 \hline
                 CC-253 & Progettisti & 2024/03/24 & Aggiornamento sezione architettura di front-end della Specifica\_architetturale.\\
                 \hline
            \end{xltabular}
}

\subsection{Prossima riunione} \label{subsec:riunione}
Viene fissata una riunione per la settimana successiva.