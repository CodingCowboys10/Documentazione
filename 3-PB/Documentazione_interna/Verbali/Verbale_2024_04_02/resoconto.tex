\section{Resoconto} \label{sec:resoconto}
\subsection{Discussioni} \label{subsec:resdiscussione}
\begin{enumerate}
    \item Vengono presentate alcune proposte di modifica al documento Analisi\_dei\_requisiti sollevate durante la settimana. Dopo una discussione con tutti i partecipanti alla riunione, si decide di mantenere il documento invariato e di procedere con l'approvazione definitiva alla versione 3.0 in previsione della revisione PB.
    \item Nel complesso il documento Specifica\_architetturale presenta tutte le sezioni ed è vicino alla conclusione. È necessario un intervento riguardante l'architettura di deployment, che potrà essere risolto solo con la definizione finale del docker compose con cui effettuare il deployment dell'applicativo. Si fissa una scadenza di 24 ore per terminare i lavori in tal senso, così da poter approvare anche l'ultimo documento necessario per effettuare la revisione PB.
    \item Vengono esposti i test relativi al software, che posso essere considerati conclusi. Nonostante la difficoltà iniziale nel codificare i test relativi al front-end, anche questi ultimi sono stati implementati raggiungendo un code coverage del 84,6\%, ampiamente sopra la soglia minima definita dal Piano\_di\_qualifica.
    \item Dopo aver visionato i documenti Norme\_di\_progetto e Glossario si decide di non programmare ulteriori interventi. Questi documenti sono quindi pronti per l'approvazione alla versione 2.0 finale. Il Manuale\_utente presenta risolte tutte le sezioni chiave del documento, che per essere terminato attende la definizione finale della procedura di deployment dell'applicazione. Per quanto riguarda il Piano\_di\_qualifica, verrà ultimato con l'aggiornamento finale dei cruscotti negli ultimi giorni di progetto. Infine, per quanto concerne il documento Piano\_di\_progetto, è parzialmente pronto l'intervento conclusivo che aspetta solo la fine dei lavori per essere aggiunto.
\end{enumerate}
\newpage
\subsection{Votazioni} \label{subsec:resvotazione}
\begin{enumerate}
    \item È avvenuta una votazione per approvare la proposta di estensione automatica di una settimana dello sprint 11, qualora la richiesta di sostenere il colloquio conclusivo della PB non arrivasse entro il giorno 2024/04/07.\\
    La proposta è stata sollevata per evitare di dichiarare la creazione di uno sprint con alcuna reale attività pianificabile.\\
    Di seguito l'esito:
\end{enumerate}
\begingroup
    \setlength{\tabcolsep}{10pt}
    \renewcommand{\arraystretch}{1.5}
    \rowcolors{2}{oddrow}{evenrow}
    \begin{tabularx}{\textwidth}{| l | l | X |}
         \hline
         \rowcolor{headerrow}\textbf{\textcolor{white}{Proposta}} & \textbf{\textcolor{white}{Sommario}} & \textbf{\textcolor{white}{Mittente}} \\
         \hline
         Favorevoli. & 100\%  & Andrea Cecchin, Marco Dolzan, Francesco Ferraioli, Francesco Giacomuzzo, Leonardo Lago, Giovanni Menon, Anna Nordio.\\
         \hline
         Contrari. & 0\% &  \\
         \hline
         Astenuti. & 0\% &  \\
         \hline
    \end{tabularx}
\endgroup

\subsection{Prossima riunione} \label{subsec:riunione}


Non vengono fissate ulteriori riunioni interne, ma avranno luogo qualora ne emerga la necessità.\\
La prossima e ultima riunione esterna è fissata al giorno 2024/04/10, alla presenza dei proponenti, per organizzare l'intervento del gruppo ad una conferenza dell'azienda a cui il gruppo è stato invitato.
