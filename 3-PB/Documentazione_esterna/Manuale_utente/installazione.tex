\chapter{Installazione} \label{cap:Inst}
\section{Clonazione Repository da Github}
L'applicazione Knowledge Management AI può essere scaricata gratuitamente da Github al seguente link: \\ \url{https://github.com/CodingCowboys10/kmai} \textit{(Ultimo accesso: 2024/04/12)}; \\
In alternativa, può essere clonata mediante terminale con il comando:
    \begin{lstlisting}
        $ git clone https://github.com/CodingCowboys10/kmai.git
    \end{lstlisting}

\section{Configurazione}
Recarsi nella cartella appena copiata alla  directory kmai/3-PB/mvp e creare un file denominato .env.local. Inserire quanto segue:
    \begin{lstlisting}
        OPENAI_API_KEY="sk-xxxx"
        NEXT_PUBLIC_HOSTED_VERSION=false
        AWS_SDK_JS_SUPPRESS_MAINTENANCE_MODE_MESSAGE=1
        export chroma_server_cors_allow_origins='["*"]'
    \end{lstlisting}
Nella prima riga, alla voce OPENAI\_API\_KEY, inserire tra i doppi apici la key di OpenAI, nel formato sk-xxxxxx...\\
Nella seconda riga, alla voce NEXT\_PUBLIC\_HOSTED\_VERSION, impostare:
\vspace{-0.3cm}
\begin{itemize}[itemsep=-2pt]
    \item true se si desidera utilizzare la versione Web\_Only;
    \item false se si desidera utilizzare la versione Full.
\end{itemize}
\vspace{-0.3cm}
Salvare e chiudere il file.\\
\textit{(Se alla chiusura, il file nella cartella non comparisse, assicurarsi che la visualizzazione degli elementi nascosti sia attiva, per poterlo visualizzare)}\\ \\
\textbf{Solo per Windows}: eseguire come amministratore il file SetHost.bat presente nella directory.

\section{Docker}
L'applicazione per essere avviata necessita dell'installazione di Docker, come riportato nella sezione \ref{cap:Req}.\\
All'interno della directory: kmai/3-PB/mvp eseguire da terminale il comando di installazione della versione desiderata:
\vspace{-0.3cm}
\begin{itemize}
    \item versioni Full\_CPU e Web\_Only:
\end{itemize}
\vspace{-0.4cm}
    \begin{lstlisting}
        $ docker compose -f docker-compose-cpu.yml up
    \end{lstlisting}
\vspace{-0.4cm}
\begin{itemize}
    \item versione Full\_GPU:
\end{itemize}
\vspace{-0.4cm}
    \begin{lstlisting}
        $ docker compose -f docker-compose-gpu.yml up
    \end{lstlisting}
Attendere la fine dell'installazione.\\ \\ \\
\textbf{Solo per la versione Full}: nel caso si desideri utilizzare la versione Full del prodotto, eseguire il seguente comando per scaricare il modello Starling (4.1 GB):
\vspace{-0.3cm}
\begin{itemize}
    \item versione Full\_CPU:
\end{itemize}
\vspace{-0.4cm}
    \begin{lstlisting}
        $ docker compose -f docker-compose-cpu.yml exec -it ollama ollama pull starling-lm:latest
    \end{lstlisting}
\vspace{-0.4cm}
\begin{itemize}
    \item versione Full\_GPU:
\end{itemize}
\vspace{-0.4cm}
    \begin{lstlisting}
        $ docker compose -f docker-compose-gpu.yml exec -it ollama ollama pull starling-lm:latest
    \end{lstlisting}
Attendere la fine dell'installazione.\\ \\ \\

\noindent Sarà infine possibile utilizzare il prodotto accedendo all'indirizzo web: \\ \\
\url{http://localhost:3000}
