\chapter{Installazione} \label{cap:Inst}
\section{Clonazione Repository da Github}
L'applicazione KMAI può essere scaricata gratuitamente da Github al seguente link: \url{https://github.com/CodingCowboys10/kmai} \textit{(Ultimo accesso: 2024/04/04)}; \\
Oppure clonata mediante terminale con il comando:
    \begin{lstlisting}
        $ git clone https://github.com/CodingCowboys10/kmai.git
    \end{lstlisting}

\section{Configurazione}
Recarsi nella cartella appena copiata alla seguente directory: \\
kmai/3-PB/mvp e creare un file nominato .env.local\\
Inserire quanto segue:
    \begin{lstlisting}
        OPENAI_API_KEY="sk-xxxx"
        NEXT_PUBLIC_HOSTED_VERSION=false
        AWS_SDK_JS_SUPPRESS_MAINTENANCE_MODE_MESSAGE=1
        export chroma_server_cors_allow_origins='["*"]'
    \end{lstlisting}
Nella prima riga, OPENAI\_API\_KEY inserire tra i doppi apici la key di OpenAI. es. sk-xxxxxx...\\
Nella seconda riga NEXT\_PUBLIC\_HOSTED\_VERSION impostare:\\ true se si desidera utilizzare la versione WEB\_only,\\ false per le modalità FULL.\\
Salvare e chiudere il file.\\
\textit{(Se alla chiusura, il file nella cartella non compare, è normale, per visualizzarlo attivare la visualizzazione delle cartelle nascoste)}

\section{Docker}
L'applicazione per essere avviata necessita dell'installazione di Docker, come riportato nella sezione \ref{cap:Req}.\\
All'interno della directory: kmai/3-PB/mvp eseguire da terminale il comando di installazione della versione desiderata: \\
versione CPU \& WEB only:
    \begin{lstlisting}
        $ sudo docker compose -f docker-compose-cpu.yml up
    \end{lstlisting}
versione GPU
    \begin{lstlisting}
        $ sudo docker compose -f docker-compose-gpu.yml up
    \end{lstlisting}
Attendere la fine dell'installazione.\\\\
SOLO per versione FULL: nel caso si desideri utilizzare la versione FULL del prodotto eseguire il seguente comando per scaricare il modello LLM (4.1 GB):\\
versione CPU:
    \begin{lstlisting}
        $ sudo docker compose -f docker-compose-cpu.yml exec -it ollama ollama pull starling-lm:latest
    \end{lstlisting}
versione GPU:
    \begin{lstlisting}
        $ sudo docker compose -f docker-compose-gpu.yml exec -it ollama ollama pull starling-lm:latest
    \end{lstlisting}
Attendere la fine di tutte le installazioni, ed infine accedere all'indirizzo web: \\ \\
\url{http://localhost:3000}
