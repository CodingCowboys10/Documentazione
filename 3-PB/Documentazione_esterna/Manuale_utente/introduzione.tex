\chapter{Introduzione} \label{cap:intro}

\section{Scopo del documento}
Lo scopo del documento è illustrare le istruzioni per l'uso e le funzionalità fornite dall'applicazione. Vengono indicati i requisiti minimi di funzionamento, come installare l'applicazione e come utilizzarla in maniera corretta.

\section{Glossario}
Al fine di prevenire ed evitare possibili ambiguità nei termini e acronimi presenti all’interno della documentazione, è stato realizzato un glossario dove sono riportati i relativi significati (vedasi Glossario\_v2.0). All’interno di ogni documento i termini specifici, che quindi hanno una definizione all’interno del Glossario, saranno contrassegnati con una ‘G’ aggiunta a pedice e trascritti in corsivo. Tale prassi sarà rispettata solamente per la prima occorrenza del termine o acronimo.

\section{Riferimenti}
\subsection{Normativi}
\begin{itemize}
    \item Norme\_di\_progetto\_v2.0;
    \item Capitolato d'appalto C1: \\ \url{https://www.math.unipd.it/~tullio/IS-1/2023/Progetto/C1.pdf} \\ \textit{(Ultimo accesso: 2024/02/19)};
    \item Verbali esterni;
    \item Regolamento Progetto Didattico: \\
    \url{https://www.math.unipd.it/~tullio/IS-1/2023/Dispense/PD2.pdf} \\ \textit{(Ultimo accesso: 2024/02/19)}.
\end{itemize}

\subsection{Informativi}
\begin{itemize}
    \item Analisi\_dei\_requisiti\_v3.0;
    \item Specifica\_architetturale\_v1.0;
\end{itemize}
