\section{Resoconto} \label{sec:resoconto}
\subsection{Discussioni} \label{subsec:resdiscussione}
\begin{enumerate}
    \item L'incontro si è aperto con l'aggiornamento sullo stato di avanzamento del progetto. Il gruppo ha informato i proponenti di aver sostenuto e superato entrambi i colloqui relativi alla revisione di avanzamento RTB, riportandone i risultati. In particolare, viene portata all'attenzione dei rappresentanti di AzzurroDigitale la modifica apportata all'Analisi dei requisiti, già anticipata via Slack, giunta ora alla versione 2.0.\\Da questo momento, il gruppo è a tutti gli effetti nella seconda fase del progetto.
    \item Interrogati su quali requisiti porrà l'attenzione nella realizzazione del MVP, il gruppo ha informato l'azienda circa la volontà di non limitare i propri sforzi alla soddisfazione dei soli requisiti obbligatori. Durante la PB, sarà obiettivo del team soddisfare anche parte dei requisiti classificati come desiderabili e opzionali. Ad esempio, constatato il forte interesse dell'azienda, uno dei vari requisiti che si vorrà soddisfare è quello relativo alla ricerca della fonte precisa di una risposta.
    \item La discussione è poi virata sugli obiettivi di avanzamento del gruppo per la riunione successiva. In particolare, è stato concordato con i proponenti di formalizzare, nei verbali esterni, questi obiettivi decisi durante le varie riunioni.\\A tal proposito, viene introdotta la sezione "Azioni da intraprendere", a cui si rimanda per conoscere gli obiettivi per l'incontro successivo.
    \item È stato consegnato una parte di un documento reale da utilizzare per valutare il comportamento del prodotto. In particolare, il documento test in questione riporta dei dati in forma tabellare: il gruppo dovrà evidenziare il livello di criticità del prodotto di questo progetto nell'interagire con il contenuto non testuale/discorsivo di documenti PDF.\\Viene inoltre ipotizzata la possibilità, una volta giunti ad un MVP, di consegnare il prodotto anche attraverso un cloud, presentando Oracle come possibile cloud service provider. Qualora questa ipotesi venisse realizzata, il prodotto accessibile via cloud si limiterebbe al funzionamento con OpenAI, escludendo LLM locali.
\end{enumerate}

\newpage
\subsection{Azioni da intraprendere} \label{subsec:action}

{
\setlength{\tabcolsep}{10pt}
\renewcommand{\arraystretch}{1.5}
\rowcolors{2}{oddrow}{evenrow}
\begin{xltabular}{\textwidth}{| X | c |}
    \hline
    \rowcolor{headerrow} \textbf{\textcolor{white}{Azione}} & \textbf{\textcolor{white}{Scadenza}} \\
    \hline
    \endhead
    Solida progettazione e selection architetturale. & 2024/02/28\\
    \hline
    Sviluppo di uno scheletro del codice. & 2024/02/28\\
    \hline
    \rowcolor{white} \caption{Azioni concordate da intraprendere}
    \label{tab:reqimp}
\end{xltabular}
}

\subsection{Prossima riunione} \label{subsec:riunione}
È stata fissata una riunione per mercoledì 28 febbraio.
