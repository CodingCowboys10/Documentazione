\section{Resoconto} \label{sec:resoconto}
\subsection{Discussioni} \label{subsec:resdiscussione}
\begin{enumerate}
    \item La riunione è iniziata informando i rappresentanti dell'azienda proponente riguardo alla definizione del MVP che ha guidato il processo di sviluppo. Questa chiarezza ha contribuito a dissipare eventuali incertezze sullo stato del prodotto.
    \item Si procede con una presentazione del MVP, evidenziando i principali miglioramenti apportati dall'ultima riunione. In particolare, la home page è stata rimossa ed è stato aggiunto un collegamento interno per consentire la transizione dalla pagina di chat alla pagina dei documenti. Inoltre, vengono illustrati i requisiti aggiunti nelle ultime settimane, con particolare attenzione ai requisiti opzionali implementati.
    \item In seguito alla presentazione, viene inviato su Slack il documento \textit{AttestazioneMvp}, il quale, una volta firmato, attesta che le aspettative sul prodotto sono state soddisfatte.
    \item La riunione si conclude con lo scambio di ulteriori dettagli riguardanti la riunione del 15 aprile, alla quale parteciperanno entrambi i gruppi coinvolti nel capitolato C1.
\end{enumerate}


\subsection{Azioni da intraprendere} \label{subsec:action}

{
\setlength{\tabcolsep}{10pt}
\renewcommand{\arraystretch}{1.5}
\rowcolors{2}{oddrow}{evenrow}
\begin{xltabular}{\textwidth}{| X | c |}
    \hline
    \rowcolor{headerrow} \textbf{\textcolor{white}{Azione}} & \textbf{\textcolor{white}{Scadenza}} \\
    \hline
    \endhead
    Bozza di presentazione per il 15 aprile.  & 2024/04/10\\
    \hline
    
    \rowcolor{white} \caption{Azioni concordate da intraprendere}
    \label{tab:reqimp}
\end{xltabular}
}
\subsection{Allegati}
In allegato al verbale esterno del 26 marzo 2024, verrà fornito il documento\\ Verbale\_2024\_03\_26\_v1.0\_AttestazioneMVP, il quale dimostrerà l'approvazione del lavoro svolto per l'MVP.

\subsection{Prossima riunione} \label{subsec:riunione}
Viene fissata una riunione per il giorno 10 aprile.

