\section{Resoconto} \label{sec:resoconto}
\subsection{Discussioni} \label{subsec:resdiscussione}
\begin{enumerate}
    \item La riunione si è aperta con l'esposizione di quanto fatto dal gruppo nelle ultime settimane. È stata informata l'azienda della realizzazione dell'interfaccia utente dell'applicazione, oltre che dell'implementazione delle funzionalità riguardanti la gestione dei documenti. La parte dell'applicativo riguardante la chat è già in fase di implementazione e sono stati svolti dei test per misurare i benefici effettivi di utilizzare Runnable Sequence e Agents di Langch\ccgloss{ai}n, per l'implementazione del \ccgloss{chatbot}.
    \item La discussione è poi virata sul design dato dal gruppo al \ccgloss{back-end} dell'applicazione. In particolare, è stato esposto con un \ccgloss{UML} come il Controller Service \ccgloss{Repository} Pattern sia stato impiegato nell'implementazione delle funzionalità riguardanti \ccgloss{MinIO} e \ccgloss{ChromaDB}. Dal confronto che ne è seguito, è sorta la necessità di rivedere la progettazione di queste parti, andando ad accorpare in un unico controller e in un unico service ogni classe riguardante la stessa area di funzionalità.\\
    Tale direttiva si rifletterà dunque sull'implementazione del pattern nel controller, service e repository relativi alla \ccgloss{chat history}.\\
    In questo momento della discussione, non emergono modifiche da effettuare sulle firme dei metodi già individuati.
    \item Viene presentato l'MVP allo stato attuale. Viene dimostrata la possibilità di aggiungere ed eliminare documenti, ricercarli per nome e data, oltre che eliminarli. Viene anche mostrata la chat, mostrandone il funzionamento con il modello locale.\\
    Da questa fase della riunione emerge l'apprezzamento per la possibilità di ottenere documento e pagina esatta della fonte della risposta. Viene consigliata dai \ccgloss{proponenti} la possibilità di indirizzare l'utente direttamente alla pagina del documento, senza che sia esso a cercarla manualmente.\\
    Il prodotto dimostrato non è comunque finale, in quanto sono necessari alcuni interventi per allinearlo a tutti i \ccgloss{requisiti} obbligatori.
    \item L'incontro termina con l'invito all'intero gruppo di presenziare ad una conferenza aziendale di AzzurroDigitale lunedì 15 aprile 2024.\\
    Durante questo evento, è attesa un'esposizione del lavoro fatto dal team, descrivendo il \ccgloss{capitolato} da cui è nato questo \ccgloss{progetto}, le sfide affrontate e mostrando una demo del prodotto finale. È richiesta la preparazione di alcune slide per la presentazione, da mostrare ai proponenti con una settimana di anticipo.
\end{enumerate}

\subsection{Discussioni successive}
Successivamente al termine della riunione, è avvenuta una discussione su \ccgloss{Slack}. Da questa è emerso come, dopo visione del codice, la struttura data al back-end dal gruppo sembra essere corretta nell'impostazione di controller, usecase e repository. Rimane invece valida la considerazione sull'unione di alcuni usecases separati erroneamente.\\
La discussione si riflette sulle effettive azioni da intraprendere.

\subsection{Azioni da intraprendere} \label{subsec:action}

{
\setlength{\tabcolsep}{10pt}
\renewcommand{\arraystretch}{1.5}
\rowcolors{2}{oddrow}{evenrow}
\begin{xltabular}{\textwidth}{| X | c |}
    \hline
    \rowcolor{headerrow} \textbf{\textcolor{white}{Azione}} & \textbf{\textcolor{white}{Scadenza}} \\
    \hline
    \endhead
   Rivisitazione della suddivisione di alcuni usecases del Service. & 2024/03/26\\
    \hline
    Implementazione delle rimanenti funzionalità relative a requisiti obbligatori. & 2024/03/26\\
    \hline
    Preparazione di una presentazione sul progetto. & 2024/04/08\\
    \hline
    \rowcolor{white} \caption{Azioni concordate da intraprendere}
    \label{tab:reqimp}
\end{xltabular}
}

\subsection{Prossima riunione} \label{subsec:riunione}
Viene fissata una riunione per martedì 26 marzo.
\newpage    
