\chapter{Analisi dei rischi}\label{chap:rischi}

In questa sezione del Piano di progetto sono esposti ed analizzati tutti i rischi riscontrabili durante la realizzazione del progetto.\\Ogni rischio, associato a problematiche organizzative o tecnologiche, è accompagnato da una previsione della probabilità di occorrenza, dalla stima della sua pericolosità e da un'azione di mitigazione, mirata a contenere tale problematica qualora si verificasse.\\Ad ogni effettiva occorrenza di uno di questi rischi, sarà documentata l'efficacia della misura mitigativa adottata per arginare il problema.

\section{Rischi organizzativi}
I rischi organizzativi si riferiscono alle possibili difficoltà che il gruppo potrebbe dover affrontare nell'ambito delle comunicazioni interne al team, con il \ccgloss{proponente}, nella pianificazione del progetto, nell'assegnazione e completamento di task.

\subsection*{Sovrastima tempo di completamento task}
\begin{description}
    \item[Codice:] RO-1
    \item[Descrizione:] Nel corso del progetto, potrebbe essere assegnata una task, a uno o più membri del team, sovrastimando l'impegno orario utile al suo completamento. Questo potrebbe portare a una riduzione inutile delle aspettative di avanzamento in uno \ccgloss{sprint}.
    \item[Mitigazione:] Il membro del team che ha completato la task di impegno sovrastimato, si adopererà ad affiancare un compagno a cui potrebbe essere stata assegnata un'altra task sottostimata per impegno orario necessario. In ogni caso, sarà evitato l'inutilizzo del tempo disponibile del membro in questione.
\end{description}

\subsection*{Sottostima tempo di completamento task}
\begin{description}
    \item[Codice:] RO-2
    \item[Descrizione:] Nel corso del progetto, potrebbe essere assegnata una task, a uno o più membri del team, sottostimando l'impegno orario necessario al suo completamento. Questo potrebbe portare a ritardi e slittamenti di altri compiti.
    \item[Mitigazione:] Il membro del team in questione sarà chiamato a segnalare la sottostima dei tempi di completamento della task, in tempi utili a far sì che sia affiancato da un altro membro del gruppo. Inoltre il gruppo si impegna ad aumentare il numero di task da creare, riducendo il carico di lavoro per ciascuna di esse.
\end{description}

\subsection*{Divergenza di vedute}
\begin{description}
    \item[Codice:] RO-3
    \item[Descrizione:] Potrebbero sorgere conflitti interni al gruppo, con difficoltà nel prendere o accettare decisioni.
    \item[Mitigazione:] Ogni divergenza di vedute sarà placata tramite una discussione interna al gruppo, accompagnata da una votazione. Ogni membro del gruppo sarà chiamato ad accettare il risultato di tale voto, legittimando la volontà della maggioranza.
\end{description}

\subsection*{Velocità di avanzamento sbilanciata}
\begin{description}
    \item[Codice:] RO-4
    \item[Descrizione:] Un membro del gruppo potrebbe essere meno esperto di un altro nel completare una determinata tipologia di task, creando una situazione di squilibrio di conoscenze e velocità di avanzamento all'interno del team.
    \item[Mitigazione:] Il membro più esperto affiancherà quello con più difficoltà, nel tentativo di uniformare il tempo di avanzamento, aumentando le conoscenze del membro meno esperto.
\end{description}

\subsection*{Stima errata dei costi di progetto}
\begin{description}
    \item[Codice:] RO-5
    \item[Descrizione:] Nel momento del preventivo, a causa della inesperienza, il team potrebbe aver sottostimato i costi utili a portare a termine il progetto.
    \item[Mitigazione:] Tramite una oculata sorveglianza dell'avanzamento del progetto, anche grazie ai consuntivi di periodo, il gruppo riorganizzerà le attività per evitare il superamento del tetto di costo.
\end{description}

\subsection*{Stima errata delle tempistiche di progetto}
\begin{description}
    \item[Codice:] RO-6
    \item[Descrizione:] Nel momento della pianificazione, il gruppo potrebbe sottostimare i tempi necessari a raggiungere le \ccgloss{milestone} del progetto, causando un ritardo.
    \item[Mitigazione:] Tramite una oculata sorveglianza dell'avanzamento del progetto, il team riorganizzerà le attività in modo di azzerare, o quanto meno limitare, lo slittamento delle milestone.
\end{description}

\subsection*{Problemi di salute}
\begin{description}
    \item[Codice:] RO-7
    \item[Descrizione:] Un membro del gruppo potrebbe essere costretto a bloccare il proprio avanzamento nel progetto, per un periodo più o meno lungo, a causa di problemi di salute.
    \item[Mitigazione:] Il team cercherà di svolgere le task assegnate al membro ammalato, nel tentativo di evitare rallentamenti allo stati si avanzamento del progetto. Una volta tornato operativo, il membro in questione recupererà le ore produttive non svolte nei successivi periodi di avanzamento.
\end{description}

\subsection*{Rallentamento da esami}
\begin{description}
    \item[Codice:] RO-8
    \item[Descrizione:] Uno o più membri del gruppo potrebbero diminuire le ore dedicate al progetto, nel periodo di svolgimento degli esami universitari e degli altri progetti didattici.
    \item[Mitigazione:] I membri del gruppo che devono svolgere un minore numero di esami, cercheranno di aumentare il loro carico di lavoro, in attesa che tutti gli altri membri tornino a pieno regime.
\end{description}

\subsection*{Dubbi irrisolti}
\begin{description}
    \item[Codice:] RO-9
    \item[Descrizione:] Potrebbero sorgere dei dubbi su come fare una determinata cosa, senza che il team riesca a risolverlo con una discussione interna.
    \item[Mitigazione:] Il dubbio sarà risolto tramite l'esposizione di tale nel \ccgloss{diario di bordo} successivo, o avvicinando direttamente i professori negli istanti che precedono e seguono le lezioni.
\end{description}

\subsection*{Difficoltà nel fissare incontri}
\begin{description}
    \item[Codice:] RO-10
    \item[Descrizione:] Potrebbe essere difficile trovare un giorno o un orario nel quale tutti i membri del gruppo possono essere presenti ad una riunione interna o con il proponente.
    \item[Mitigazione:] Sarà eseguita una votazione interna al gruppo per individuare lo slot temporale nel quale sono presenti più membri possibili, e sarà chiesto un sacrificio affinché ogni membro sia presente, potendo comunque attivare una modalità di riunione duale, con collegato da remoto il membro che non può essere presente dal vivo.
\end{description}

\subsection*{Rettifica data di un incontro}
\begin{description}
    \item[Codice:] RO-11
    \item[Descrizione:] Potrebbe essere necessario spostare una riunione, interna o con il proponente, perché un membro solleva in ritardo la problematica di non poter essere presente.
    \item[Mitigazione:] In caso di incontro con il proponente, sarà chiesto di spostare la riunione ad un giorno in cui ogni membro del team può essere presente. In caso di riunione interna, sarà attivata la modalità duale di incontro.
\end{description}

\subsection*{Difficoltà nel contattare il proponente}
\begin{description}
    \item[Codice:] RO-12
    \item[Descrizione:] Potrebbe risultare difficile entrare in contatto con il proponente, con questo ultimo che potrebbe non rispondere alle mail o ai messaggi.
    \item[Mitigazione:] Nel caso i contatti asincroni non dovessero funzionare, sarà contattata telefonicamente l'azienda del proponente.
\end{description}

\subsection*{Difficoltà comunicative con il proponente}
\begin{description}
    \item[Codice:] RO-13
    \item[Descrizione:] Potrebbero sorgere delle difficoltà comunicative con il proponente, con delle incomprensioni nelle discussioni e nelle scelte.
    \item[Mitigazione:] Qualora il team dovesse accorgersi di incomprensioni con il proponente, questo sarà contattato per fissare un incontro quanto prima per discutere e superare queste difficoltà.
\end{description}


\section{Rischi tecnologici}

\subsection*{Inesperienza}
\begin{description}
    \item[Codice:] RT-1
    \item[Descrizione:] Potrebbero sorgere delle difficoltà causate dall'utilizzo di tecnologie mai utilizzate prima.
    \item[Mitigazione:] I membri del gruppo con maggiore esperienza affiancheranno quelli con minor dimestichezza, con questi ultimi chiamati a pareggiare la differenza di conoscenza tramite l'auto-apprendimento.
\end{description}

\subsection*{Dispositivi posseduti non adeguati}
\begin{description}
    \item[Codice:] RT-2
    \item[Descrizione:] Potrebbero sorgere delle difficoltà causate dall'inadeguatezza dei dispositivi personali possedute da un membro, causando l'impossibilità a lavorare con le tecnologie richieste dal progetto.
    \item[Mitigazione:] Il membro del gruppo cercherà di sopperire al problema affidandosi a tecnologie utili, come ad esempio macchine virtuali.
\end{description}

\subsection*{Tecnologie scelte inadeguate}
\begin{description}
    \item[Codice:] RT-3
    \item[Descrizione:] In una fase più avanzata dell'analisi dei \ccgloss{requisiti}, il gruppo potrebbe accorgersi di aver scelto delle tecnologie non utili al completamento del progetto.
    \item[Mitigazione:] Chiedendo consiglio al proponente, il gruppo individuerà nuove tecnologie con cui lavorare.
\end{description}

\section{Tracciamento dei rischi}
Nelle tabelle sottostanti vengono tracciati i rischi sopra individuati, andando ad associare ad ognuno di essi una stima della probabilità di occorrenza della problematica e la sua pericolosità, ovvero l'impatto che essa avrebbe nel normale svolgimento del progetto.\\
Sia la stima dell'occorrenza che quella della pericolosità del rischio sono misurate con la seguente scala a cinque livelli:
\begin{itemize}
    \item alta: il valore maggiore, indica una quasi certa probabilità che il rischio occorra, o un elevato impatto negativo nella normale realizzazione del progetto quando riferito alla pericolosità;
    \item medio-alta;
    \item media: il valore intermedio, indica una probabilità media di occorrenza del rischio, o un impatto negativo medio nella normale realizzazione del progetto quando riferito alla pericolosità;
    \item medio-bassa;
    \item bassa: il valore minore, indica una quasi nulla probabilità che il rischio occorra, o un bassissimo impatto negativo nella normale realizzazione del progetto quando riferito alla pericolosità.
\end{itemize}

{
\setlength{\tabcolsep}{10pt}
\renewcommand{\arraystretch}{1.5}
\rowcolors{2}{oddrow}{evenrow}
\begin{xltabular}{\textwidth}{| l | l | l |}
        \hline
        \rowcolor{headerrow} \textbf{\textcolor{white}{Codice}} & \textbf{\textcolor{white}{Occorrenza}} & \textbf{\textcolor{white}{Pericolosità}} \\
        \hline
        \endfirsthead
        \hline
        \rowcolor{headerrow} \textbf{\textcolor{white}{Codice}} & \textbf{\textcolor{white}{Occorrenza}} & \textbf{\textcolor{white}{Pericolosità}} \\
        \hline
        \endhead
        RO-1 & Alta & Media \\
        \hline
        RO-2 & Alta & Medio-alta \\
        \hline
        RO-3 & Medio-alta & Medio-bassa \\
        \hline
        RO-4 & Alta & Media \\
        \hline
        RO-5 & Medio-alta & Alta \\
        \hline
        RO-6 & Medio-alta & Alta \\
        \hline
        RO-7 & Alta & Medio-bassa \\
        \hline
        RO-8 & Media & Media \\
        \hline
        RO-9 & Medio-alta & Medio-bassa \\
        \hline
        RO-10 & Medio-bassa & Medio-bassa \\
        \hline
        RO-11 & Medio-bassa & Medio-bassa \\
        \hline
        RO-12 & Bassa & Alta \\
        \hline
        RO-13 & Medio-bassa & Medio-alta \\
        \hline
        RT-1 & Alta & Medio-alta \\
        \hline
        RT-2 & Medio-bassa & Alta \\
        \hline
        RT-3 & Media & Alta \\
        \hline
    \rowcolor{white} \caption{Stima di occorrenza e pericolosità dei rischi}
    \label{tab:stimarischi}
\end{xltabular}
}

\section{Valutazione delle misure mitigative}
Nella seguente tabella vengono riportati solo i rischi che hanno trovato una effettiva occorrenza nel corso del progetto. Per ognuna di queste problematiche affrontate, è riportato il grado di efficacia della relativa misura mitigativa adottata, accompagnata da una analisi.\\
L'efficacia di una misura mitigativa è misurata con la seguente scala a tre livelli:
\begin{itemize}
    \item alta: la misura mitigativa ha neutralizzato l'impatto del rischio secondo le attese;
    \item media: la misura mitigativa ha almeno in parte neutralizzato l'impatto del rischio, ma è individuabile una misura migliore;
    \item bassa: la misura mitigativa è risultata inadeguata, e sarà quanto prima sostituita da una nuova.
\end{itemize}
In caso di bassa efficacia di una misura mitigativa, sarà compito del gruppo individuare subito un'alternativa, aggiornando la mitigazione individuata per quel rischio, così come la valutazione della sua efficacia.\\ \\

{
\setlength{\tabcolsep}{10pt}
\renewcommand{\arraystretch}{1.5}
\rowcolors{2}{oddrow}{evenrow}
\begin{xltabular}{\textwidth}{| l | l | X |}
    \hline
    \rowcolor{headerrow} \textbf{\textcolor{white}{Codice}} & \textbf{\textcolor{white}{Efficacia mitigazione}} & \textbf{\textcolor{white}{Analisi}} \\
    \hline
    \endfirsthead
    \hline
    \rowcolor{headerrow} \textbf{\textcolor{white}{Codice}} & \textbf{\textcolor{white}{Efficacia mitigazione}} & \textbf{\textcolor{white}{Analisi}} \\
    \hline
    \endhead
    RO-2 & Alta & L'aumento del numero delle \ccgloss{issue} con conseguente diminuzione del carico di lavoro ha portato ad un miglioramento dell'efficienza di conclusione di un compito. \\
    \hline
    RO-3 & Alta & La divergenza interna è stata superata velocemente con una discussione interna al gruppo, senza la necessità di ricorrere ad una votazione formale. \\
    \hline
    RO-9 & Alta & È stato necessario più volte chiedere consiglio ai professori per superare un dubbio sorto, e non risolto, in una discussione interna al team. Grazie alle lezioni rovesciate, alle discussioni post diario di bordo e alle domande poste prima delle lezioni, è stato possibile giungere a una risoluzione dei dubbi. \\
    \hline
    RO-10 & Alta & Grazie a dei semplici sondaggi, è stato possibile fissare incontri massimizzando il numero di membri presenti, anche ricorrendo a delle riunioni da remoto se necessario. \\
    \hline
    RO-11 & Alta & Grazie alla disponibilità del proponente, è stato possibile riorganizzare un meeting modificando la data dell'incontro. \\
    \hline
    RT-2 & Media & Grazie alla disponibilità del proponente, è stato possibile utilizzare un account \ccgloss{OpenAI}, riducendo la necessità di specifiche configurazioni dei dispositivi. Nonostante questo il gruppo, per volere del proponente, ha la necessità di utilizzare \ccgloss{LLM} locali, costringendo i membri che possiedono dispositivi inadeguati a non poter effettuare test su questi modelli.\\
    \hline
    \rowcolor{white} \caption{Efficacia delle misure mitigative}
    \label{tab:mitigazioni}
\end{xltabular}
}


\newpage