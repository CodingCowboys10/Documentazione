\chapter{Introduzione} \label{cap:intro}
\section{Scopo del documento}
Il documento ha l'obiettivo di descrivere dettagliatamente l'architettura logica e di deployment, i design pattern adottati e le tecnologie impiegate nella realizzazione del prodotto per il progetto Knowledge Management AI.
\section{Glossario}
Al fine di prevenire ed evitare possibili ambiguità nei termini e acronimi presenti all’interno della documentazione, è stato realizzato un glossario dove sono riportati i relativi significati (vedasi Glossario\_v2.0). All’interno di ogni documento i termini specifici, che quindi hanno una definizione all’interno del Glossario, saranno contrassegnati con una ‘G’ aggiunta a pedice e trascritti in corsivo. Tale prassi sarà rispettata solamente per la prima occorrenza del termine o acronimo.
\section{Riferimenti}
\subsection{Normativi}
\begin{itemize}
    \item Norme\_di\_progetto\_v2.0;
    \item Capitolato d'appalto C1: \\ \url{https://www.math.unipd.it/~tullio/IS-1/2023/Progetto/C1.pdf} \textit{(Ultimo accesso: 2024/02/10)};
    \item Verbali esterni;
    \item Regolamento Progetto Didattico: \\
    \url{https://www.math.unipd.it/~tullio/IS-1/2023/Dispense/PD2.pdf} \textit{(Ultimo accesso: 2024/02/10)}.
\end{itemize}

\subsection{Informativi}
\begin{itemize}
    \item Analisi\_dei\_requisiti\_v3.0;
    \item Piano\_di\_qualifica\_v2.0;
    \item Slide dell’insegnamento di Ingegneria del Software, in particolare:
        \begin{itemize}
            \item I pattern architetturali: \\ \url{https://www.math.unipd.it/~rcardin/swea/2022/Software%20Architecture%20Patterns.pdf} \textit{(Ultimo accesso: 2024/02/10)};
            \item Il pattern Dependency Injection:\\ \url{https://www.math.unipd.it/~rcardin/swea/2022/Design%20Pattern%20Architetturali%20-%20Dependency%20Injection.pdf} \textit{(Ultimo accesso: 2024/02/10)};
            \item Progettazione Software:\\ \url{https://www.math.unipd.it/~tullio/IS-1/2023/Dispense/T6.pdf} \textit{(Ultimo accesso: 2024/02/10)};
            \item Qualità del Software:\\ \url{https://www.math.unipd.it/~tullio/IS-1/2023/Dispense/T7.pdf} \textit{(Ultimo accesso: 2024/02/10)};
            \item Pattern della GoF:\\ \url{https://www.math.unipd.it/~rcardin/swea/2022/Design%20Pattern%20Creazionali.pdf} \textit{(Ultimo accesso: 2024/02/10)};\\
            \url{https://www.math.unipd.it/~rcardin/swea/2021/Design%20Pattern%20Comportamentali_4x4.pdf} \textit{(Ultimo accesso: 2024/02/10)};\\
            \url{https://www.math.unipd.it/~rcardin/swea/2022/Design%20Pattern%20Strutturali.pdf} \textit{(Ultimo accesso: 2024/02/10)};
            \item Programmazione SOLID:\\ \url{https://www.math.unipd.it/~rcardin/swea/2021/SOLID%20Principles%20of%20Object-Oriented%20Design_4x4.pdf} \textit{(Ultimo accesso: 2024/02/10)}.
        \end{itemize}
    \item Sito ufficiale ChromaDB: \\
    \url{https://www.trychroma.com} \textit{(Ultimo accesso: 2024/02/20)}.
    \item Documentazione ufficiale LangChain JS: \\
    \url{https://js.langchain.com/docs/get_started/introduction} \textit{(Ultimo accesso: 2024/02/20)}.
    \item Documentazione ufficiale MinIO: \\
    \url{https://min.io/docs/minio/kubernetes/upstream/} \textit{(Ultimo accesso: 2024/02/20)}.
    \item Documentazione ufficiale Next.js: \\
    \url{https://nextjs.org/docs} \textit{(Ultimo accesso: 2024/02/20)}.
    \item Sito ufficiale Node.js: \\
    \url{https://nodejs.org/en} \textit{(Ultimo accesso: 2024/02/20)}.
    \item Sito ufficiale OpenAI API: \\
    \url{https://openai.com/blog/openai-api} \textit{(Ultimo accesso: 2024/02/20)}.
    \item Sito ufficiale React: \\
    \url{https://it.legacy.reactjs.org} \textit{(Ultimo accesso: 2024/02/20)}.
    \item Appunti su "Clean Architecture" di Robert C. Martin: \\
    \url{https://mirkorap16.gitbook.io/clean-architecture/architettura-clean} \textit{(Ultimo accesso: 2024/03/18)}.
\end{itemize}
