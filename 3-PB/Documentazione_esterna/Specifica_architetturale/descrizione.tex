\chapter{Descrizione del prodotto} \label{cap:descr}
\section{Scopo del prodotto}
Lo scopo del prodotto è fornire alle aziende un’applicazione di knowledge management che permetta ai propri lavoratori di ricevere risposte, chiare e immediate, a domande inerenti al contesto lavorativo.\\
Questa applicazione deve permettere a ogni dipendente di ottenere informazioni sul funzionamento dei macchinari aziendali, sui processi di lavorazione dei diversi prodotti, sulle norme di sicurezza da rispettare, o più in generale istruzioni su come eseguire una azione di qualsiasi tipo, pur rimanendo vincolati al contesto d’uso.\\
Le informazioni date all'utente, tramite un’interfaccia \ccgloss{chatbot}, sono contenute nei documenti inseriti nel sistema dall’azienda stessa, e dovranno essere esposte in modo chiaro e comprensibile ad ogni lavoratore, così da rendere le informazioni utili più accessibili e velocemente reperibili.


\section{Funzionalità}
Questa applicazione prevede due principali funzionalità: la gestione della documentazione aziendale, e l’interazione con un’interfaccia chat per reperire le informazioni contenute nei documenti.\\
La gestione dei documenti deve permettere il caricamento di un nuovo documento, l’eliminazione e la consultazione dei documenti già presenti.\\
Nell’interfaccia del chatbot, l’utente deve essere in grado di fare una domanda riguardante un’informazione contenuta all’interno dei documenti aziendali, e ricevere una risposta corretta, chiara ed esaustiva.


