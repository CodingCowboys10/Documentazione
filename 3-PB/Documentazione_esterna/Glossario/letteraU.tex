\chapter{U}
\section{UML}\label{sec:Unified Modeling Language}
Acronimo per Unified Modeling Language. Si tratta di un linguaggio di modellazione e di specifica basato sul paradigma orientato agli oggetti.

\section{Use case}\label{sec:Use Cases}
In riferimento all'analisi dei requisiti, gli use case sono un insieme di scenari, ovvero sequenze di azioni, che hanno in comune un obiettivo per un utente chiamato attore.\\ \\
In riferimento alla programmazione, lo Use case è la classe che presenta la business logic dell'applicazione. Corrisponde al Service del \ccgloss{CSR} pattern.
