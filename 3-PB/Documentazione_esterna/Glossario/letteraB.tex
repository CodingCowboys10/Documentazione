\chapter{B}
\section{Back end}\label{sec:Back-end}
La parte di amministrazione di un sito o web app accessibile solo da \ccgloss{amministratori} della pagina web.

\section{Backlog}
Il backlog è un artefatto ufficiale di \ccgloss{Scrum} che consiste in un elenco di attività ordinato per priorità. Il Backlog viene costantemente rivisto e riordinato in base alle necessità degli utenti o del cliente, nuove idee, o in seguito alle esigenze di mercato, ma anche in base a suggerimenti da parte del team.

\section{Baseline}
La baseline di progetto è un punto di riferimento fisso utilizzato per confrontare le prestazioni dei progetti nel tempo. Le baseline di progetto sono utilizzate dai project manager per verificare l'andamento dell'ambito, della programmazione e dei costi del progetto fino al suo completamento.

\section{Branch}
Un branch rappresenta una linea indipendente di sviluppo. I branch fungono da astrazione per il processo di modifica/\ccgloss{commit}.

\section{Bug}\label{sec:Bugs}
Un \ccgloss{bug} è un guasto che causa un malfunzionamento del software. Il bug di solito è attribuibile ad errori di codice.

\section{Bug reporting}
Il monitoraggio dei bug è il processo di registrazione e monitoraggio dei bug o degli errori durante un test di software. Viene anche chiamato monitoraggio delle anomalie o dei problemi.
