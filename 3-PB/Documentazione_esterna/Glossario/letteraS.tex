\chapter{S}

\section{Scrum}
Scrum è un framework di gestione dei progetti \ccgloss{Agile} che aiuta i team a strutturare e gestire il proprio lavoro attraverso un insieme di valori, principi e pratiche. Proprio come una squadra di gioco del rugby (sport a cui deve il nome), il framework Scrum incoraggia i team a imparare attraverso l'esperienza, a organizzarsi in modo autonomo mentre lavorano su un problema e a riflettere sui risultati conseguiti e sugli insuccessi per migliorare continuamente.

\section{SEMAT}
Indicatori di misura del progresso effettivo di un progetto, basati su un sistema a sei livelli.

\section{Slack}
Slack è un software che rientra nella categoria degli strumenti di collaborazione aziendale utilizzato per inviare messaggi in modo istantaneo ai membri del team.

\section{Sottotask}
Un Sottotask rappresenta una scomposizione più granulare del lavoro necessario per completare un \ccgloss{ticket} standard.

\section{Sprint}
Uno sprint è un periodo di tempo fisso in un ciclo di sviluppo continuo, nel nostro caso specifico 2 settimane, in cui il team completa il lavoro del loro \ccgloss{backlog} di prodotto. Alla fine dello sprint, un team avrà in genere creato e implementato un incremento di prodotto funzionante.

\section{SQLite}\label{sec:SQLite3}
SQLite è un sistema di gestione di database relazionali leggero e incorporato. Si tratta di una libreria software scritta in linguaggio C che fornisce un motore di database SQL transazionale senza la necessità di un server separato.

\section{Squash}
In Git rappresenta l'azione di combinare o unire più commit consecutivi in uno unico.

\section{Stakeholder}
Lo stakeholder, detto anche portatore di interesse, è un soggetto o gruppo coinvolto in un'iniziativa economica, società o altro progetto e con interessi legati all’esecuzione o all’andamento dell'iniziativa stessa.
