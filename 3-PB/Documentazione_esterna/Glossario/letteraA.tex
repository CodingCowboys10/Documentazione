\chapter{A}

\section{Agile}
\ccgloss{Modello di sviluppo} caratterizzato da una comunicazione efficace all'interno del gruppo e con lo \ccgloss{stakeholder} e che dà più importanza a un codice funzionante che a una documentazione esaustiva, più importanza al sapersi adattare a imprevisti che a seguire passo passo un piano.

\section{AI} \label{sec:Artificial Intelligence}
Acronimo per Artificial Intelligence, ovvero la disciplina che studia se e in che modo si possano realizzare sistemi informatici intelligenti in grado di simulare la capacità e il comportamento del pensiero umano.

\section{Amministratore} \label{sec:Amministratori}
Nell'ambito di un progetto definisce, controlla e mantiene l'ambiente IT di lavoro tramite delle azioni di selezione e messa in opera di risorse informatiche a supporto del \ccgloss{way of working}. 

\section{Analista}\label{sec:Analisti}
Nell'ambito di un progetto, chi conosce il dominio del problema e individua i \ccgloss{requisiti} per avviare una progettazione efficace ed efficiente.

\section{Angular}
\ccgloss{Framework} \ccgloss{open source} per lo sviluppo di applicazioni web.

\section{API}\label{sec:Application Programming Interface}
Acronimo per Application Programming Interface, ovvero un insieme di procedure (in genere raggruppate per strumenti specifici) atte a risolvere uno specifico problema di comunicazione tra diversi computer o tra diversi software o tra diversi componenti di software; spesso tale termine designa le librerie software di un linguaggio di programmazione, sebbene più propriamente le API sono il metodo con cui le librerie vengono usate per sopperire ad uno specifico problema di scambio di informazioni.

\section{Atlassian}
Atlassian Corporation Plc è una compagnia di software enterprise che sviluppa prodotti per sviluppatori di software, \ccgloss{project management} e per la gestione dei contenuti. È nota soprattutto per la sua applicazione di tracciamento delle segnalazioni, \ccgloss{Jira}, e per l'applicazione wiki e collaborativa Confluence.

\section{Attore}\label{sec:Attori}
Negli \ccgloss{use cases}, indica il \ccgloss{ruolo} ricoperto da un utente nell'interazione con un sistema.

\section{Automazione}\label{sec:Automazioni}
Un sistema di istruzioni che consente di svolgere una serie ripetibile di processi in sostituzione del lavoro manuale sui sistemi informatici.

\section{AzzurroDigitale}
Azienda proponente del \ccgloss{capitolato} "\ccgloss{Knowledge Management} \ccgloss{Ai}" scelto dal gruppo.
