\chapter{D}


\section{Dashboard}
Il termine inglese dashboard significa letteralmente "pannello di controllo" o "cruscotto". In informatica, le dashboards sono interfacce grafiche per l'utente, cioè una disposizione di vari elementi grafici che servono a visualizzare i dati o a gestire i sistemi. In altri termini si può definire come una panoramica di informazioni e grafici che indicano dati di prestazione.

\section{Database}
Basi di dati in italiano, un database è una raccolta sistematica di dati organizzati archiviata elettronicamente.

\section{Data Source}\label{sec:Data Sources}
Classe che presenta un riferimento diretto al database a cui si riferisce. Essa presenta i metodi concreti necessari alle operazioni di lettura e scrittura sui dati.

\section{Dependency Injection}
Design pattern architetturale rivolto alla separazione del comportamento di una componente dalla risoluzione delle sue dipendenze. Spesso si ottiene mediante l'utilizzo di librerie e framework esterni che gestiscono il ciclo di vita degli oggetti.

\section{Diagramma dei casi d'uso}\label{sec:Diagrammi dei casi d'uso}
Rappresentazione grafica degli use cases che mette in evidenza \ccgloss{attori} e servizi del sistema. Si tratta di un grafo i cui nodi sono attori e use cases mentre gli archi rappresentano la comunicazione tra attori e use cases e i legami tra gli use cases stessi. Spesso per realizzarlo si usa \ccgloss{UML}.

\section{Diagramma di \ccgloss{Gantt}}\label{sec:Diagrammi di Gantt}
Rappresentazione grafica della dislocazione temporale delle attività. In particolare, viene utilizzato per rappresentare la durata, la sequenzialità e il parallelismo delle attività oltre a permettere un confronto tra stime e progressi.

\section{Diario di bordo}\label{sec:Diari di bordo}
Presentazione che il gruppo fornitore presenta al committente nella quale si monitorano costantemente il rapporto tra costi sostenuti e avanzamento conseguito.

\section{Discord}
Piattaforma di messaggistica istantanea dove gli utenti comunicano con videochiamate, chiamate vocali o messaggi in chat private o come membri di un server.

\section{Docker}
Docker è una piattaforma open source progettata per automatizzare il rilascio, la scalabilità e la gestione di applicazioni. Utilizza la tecnologia dei container per incapsulare un'applicazione e tutte le sue dipendenze, garantendo che l'applicazione possa essere eseguita in modo coerente su qualsiasi ambiente di sviluppo, test o produzione.

\section{Docker Compose}
Strumento progettato per gestire il deploy di applicazioni tramite la definizione di container. Questo tool di Docker permette di raggruppare e avviare un insieme di container con un solo comando.
