\chapter{C}

\section{CA}\label{sec:Customer Acceptance}
Acronimo di Customer Acceptance. Terza ed ultima revisione di avanzamento nell'ambito del progetto didattico in corso, rappresenta il collaudo finale del prodotto software finito in presenza di committenti e proponenti. A differenza delle altre due baseline, è l'unica facoltativa.

\section{Capitolato}\label{sec:Capitolati}
Il capitolato è un documento in cui sono descritte nel dettaglio tutte le lavorazioni e le opere che saranno realizzate. Vengono indicate: le modalità e i tempi di esecuzione e consegna delle opere;i costi per l'esecuzione.

\section{Chat history}
Cronologia delle conversazioni precedenti tenute con un interlocutore. L'interlocutore nel prodotto si riferirà sempre ad un \ccgloss{chatbot}.

\section{Chatbot}
Un chatbot è un programma che usa AI ed \ccgloss{LLM} per capire le domande dei clienti e automatizzare le relative risposte, simulando la conversazione umana.

\section{ChatGPT}
ChatGPT è un chatbot basato su intelligenza artificiale e apprendimento automatico sviluppato da \ccgloss{OpenAI} specializzato nella conversazione con un utente umano. La sigla GPT sta per Generative Pre-trained Transformer, ovvero "trasformatore generativo pre-addestrato". 

\section{ChromaDB}\label{sec:Chroma}
ChromaDB è un database open source per l'archiviazione e la ricerca di vettori di \ccgloss{embeddings}.

\section{Clean}\label{sec:Clean Architecture}
In riferimento all'architettura di sistema, rappresenta una tipologia di architettura ove gli elementi costituenti sono logicamente organizzati in modo concentrico. Quelli più interni rappresentano gli elementi di alto livello del software, mentre quelli più esterni rappresentano gli elementi di basso livello del software. La regola base su cui si basa questa architettura è la regola della dipendenza, che spiega come le dipendenze presenti nel codice sorgente devono puntare solo all'interno, verso le politiche di alto livello. Ovvero nulla in un cerchio interno può conoscere qualcosa di un cerchio più esterno.

\section{Cloud computing}\label{sec:Cloud}
Indica un'erogazione di servizi offerti su richiesta da un fornitore a un utente finale attraverso la rete internet (come l'archiviazione, l'elaborazione o la trasmissione dati), a partire da un insieme di risorse preesistenti, configurabili e disponibili in remoto sotto forma di architettura distribuita.

\section{Code coverage}
La code coverage è un metodo per misurare la quantità di codice di un programma che è stato effettivamente eseguito durante un test, così da determinare se il codice è stato testato in modo completo ed efficace.

\section{Commit}\label{sec:Commits}
Il comando commit è l'azione che nel software \ccgloss{Git} permette di aggiungere, rimuovere e modificare i file del \ccgloss{repository}.

\section{Committente}\label{sec:Committenti}
Il \ccgloss{committente} è quel soggetto o organizzazione per conto del quale viene realizzata un'opera o svolto un servizio. In questo progetto i committenti sono i professori Tullio Vardanega e Riccardo Cardin.

\section{Componente}\label{sec:Componenti}
In riferimento ai componenti React, funzione Javascript che ritorna un elemento JSX, il quale quando viene renderizzato nel DOM (Document Object Model) costituisce una parte della interfaccia utente.

\section{Consulenza}\label{sec:Consulenze}
É la professione di un consulente, ovvero una persona che, avendo accertata qualifica in una materia, consiglia e assiste il proprio committente nello svolgimento di cure, atti, pratiche o progetti fornendo o implementando informazioni, pareri o soluzioni attraverso le proprie conoscenze e le proprie capacità di problem solving.

\section{Consuntivo}\label{sec:Consuntivi}
Rendiconto, sia delle imprese sia degli enti pubblici, dei risultati di un dato periodo di attività.

\section{Controller}\label{sec:Controllers}
Classe del back-end dell'applicativo che gestisce la chiamata allo Use case correlato, gestendo il formato dei dati ritornati e gli eventuali errori emersi nel processo.

\section{CPU}
Acronimo di Central Processing Unit, è il sottosistema che implementa le funzionalità fondamentali dell'elaboratore, coordinando l'esecuzione delle operazioni tra gli eventuali sottosistemi periferici.

\section{CSR}\label{sec:Controller-Service-Repository}
Acronimo di Controller-Service-Repository, è un design pattern adottato per gestire le operazioni sui dati lato back-end. Come intuibile dal nome, prevede la definizione di classi Controller, Service, o Use case, e Repository.

\section{CSS}\label{sec:Cascading Style Sheets}
CSS, acronimo Cascading Style Sheets, è un linguaggio usato per definire la formattazione e l'aspetto grafico di documenti HTML, XHTML e XML, come ad esempio i siti e le relative pagine web.

\section{CUDA}\label{sec:Cascading Style Sheets}
CUDA, acronimo di Compute Unified Device Architecture, è un'architettura hardware per l'elaborazione parallela sviluppata da NVIDIA, e impiegata nell'utilizzo delle proprie GPU successive alla serie GeForce 8.