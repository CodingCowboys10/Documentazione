\chapter{R}

\section{React}
React (noto anche come React.js o ReactJS) è una libreria open-source, \ccgloss{front-end}, JavaScript per la creazione di interfacce utente.

\section{Repository}\label{sec:Repo}
In riferimento alla gestione della configurazione, un repository, o repo, è un archivio digitale centralizzato che gli sviluppatori utilizzano per apportare e gestire le modifiche al codice sorgente di un'applicazione. Gli sviluppatori hanno la necessità di archiviare e condividere cartelle, file di testo e altri tipi di documenti durante lo sviluppo del software. Un repo ha caratteristiche che consentono agli sviluppatori di tenere traccia delle modifiche al codice, di modificare simultaneamente i file e di collaborare in modo efficiente allo stesso progetto da qualsiasi luogo.\\ \\
In riferimento alla programmazione, il Repository è la classe chiamata dal Service, \ccgloss{responsabile} di operare sulle fonti dei dati. In essa è posta la logica di persistenza dell'applicativo.

\section{Requisito}\label{sec:Requisiti}
Un requisito è una descrizione di qualcosa che il sistema dovrà fare, o di un vincolo che il sistema dovrà rispettare durante l’esecuzione dei suoi compiti.

\section{Responsabile}
Chi, nell'ambito di un progetto, governa il team e rappresenta il progetto verso l'esterno. Ha responsabilità di scelta e approvazione oltre che di pianificazione, gestione delle risorse, coordinamento e gestione delle relazioni esterne. 

\section{Risorsa}\label{sec:Risorse}
Una risorsa è una qualsiasi entità o elemento utilizzato dal sistema per eseguire operazioni, processi o attività.

\section{RTB}\label{sec:Requirements and Technology Baseline}
Acronimo per Requirements and Technology Baseline. Prima revisione di avanzamento obbligatoria nell'ambito del progetto didattico in corso, fissa i requisiti da soddisfare e motiva la tecnologie adottate, dimostrandone adeguatezza e compatibilità.
\section{Ruolo}\label{sec:Ruoli}
Incarico che ogni membro di un team che lavora a un progetto può assumere, in particolare i ruoli possibili in questo progetto sono responsabile, amministratore, \ccgloss{analista}, \ccgloss{progettista}, programmatore e \ccgloss{verificatore}.
