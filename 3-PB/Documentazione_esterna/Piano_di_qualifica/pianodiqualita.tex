\chapter{Piano di qualità}

\section{Introduzione}
\label{sec:qualityintro}
La qualità di un progetto è fortemente influenzata dalla qualità dei processi che lo compongono.\\ Stabilire delle metriche che valutino i processi e ne giudichino la qualità è perciò imperativo per raggiungere standard qualitativi apprezzabili.\\ Verranno indicati, per ogni metrica che lo prevede, il valore accettabile e quello preferibile.

\section{Qualità di processo} \label{sec:qualityproc}

{
\setlength{\tabcolsep}{10pt}
\renewcommand{\arraystretch}{1.5}
\rowcolors{2}{oddrow}{evenrow}
\begin{xltabular}{\textwidth}{| l | X | c | c |}
    \hline
    \rowcolor{headerrow} \textbf{\textcolor{white}{Codice}} & \textbf{\textcolor{white}{Nome}} & \textbf{\textcolor{white}{Accettabile}} & \textbf{\textcolor{white}{Preferibile}} \\
    \hline
    \endfirsthead
    \hline
    \rowcolor{headerrow} \textbf{\textcolor{white}{Codice}} & \textbf{\textcolor{white}{Nome}} & \textbf{\textcolor{white}{Accettabile}} & \textbf{\textcolor{white}{Preferibile}} \\ 
    \endhead
    MPC-1 & Varianza di Budget & \pm 10\% & \pm 0\% \\
    \hline
    MPC-2 & Varianza dell’impegno orario & \pm 5\% & \pm 0\% \\
    \hline
    MPC-3 & Earned Value & >= MPC-4 &  \\
    \hline
    MPC-4 & Actual Cost & - & - \\
    \hline
    MPC-5 & Planned Value & - & - \\
    \hline
    MPC-6 & Cost Variance & \pm 150 & 0 \\
    \hline
    MPC-7 & Schedule Variance & \pm 150 & 0 \\
    \hline
    MPC-8 & Cost Performance Index & 1 \pm 0.1 & 1 \\
    \hline
    MPC-9 & Schedule Performance Index & 1 \pm 0.1 & 1 \\
    \hline
    MPC-10 & Estimate to Complete & - & -\\
    \hline
    MPC-11 & Estimate at Completion & \pm5\% di MPC-12 & MPC-12 \\
    \hline
    MPC-12 & Budget at Completion & - & - \\
    \hline
    MPC-13 & Code Coverage & 75\% & 100\% \\
    \hline
    MPC-14 & Misure di mitigazione insufficienti & 3 & 0 \\
    \hline
    MPC-15 & Rischi inattesi & 3 & 0 \\
    \hline
    \rowcolor{white} \caption{Metriche di qualità di processo}
    \label{tab:mpc}
\end{xltabular}
}

\subsection{Varianza di Budget}
\begin{description}
    \item[Codice:] MPC-1
    \item[Processo:] Fornitura.
    \item[Formula:] 
    \begin{equation}
        100\biggl(\frac{\text{Budget Consuntivato} - \text{Budget Preventivato}}{\text{Budget Preventivato}}\biggr)
    \end{equation}
    \item[Descrizione:] Questa metrica valuta la percentuale di variazione del budget tra preventivo e consuntivo in uno sprint. Il valore è positivo quando viene preventivato un budget inferiore a quello effettivamente utilizzato, mentre è negativo quando viene preventivato un budget maggiore a quello effettivamente utilizzato.
\end{description}

\subsection{Varianza dell’impegno orario}
\begin{description}
    \item[Codice:] MPC-2
    \item[Processo:] Fornitura.
    \item[Formula:]
    \begin{equation}
        100\biggl(\frac{\text{Ore Consuntivate} - \text{Ore Preventivate}}{\text{Ore Preventivate}}\biggr)
    \end{equation}
    \item[Descrizione:] Questa metrica valuta la percentuale di variazione dell'impegno orario complessivo tra preventivo e consuntivo in uno sprint. Il valore è positivo quando viene preventivato un impegno orario inferiore a quello effettivamente svolto, mentre è negativo quando viene preventivato un impegno orario maggiore a quello effettivamente svolto.
\end{description}

\subsection{Earned Value}
\begin{description}
    \item[Codice:] MPC-3
    \item[Processo:] Fornitura.
    \item[Formula:]
    \begin{equation}
        \text{Budget Preventivato} * \% \text{Completamento di Attività Sprint} 
    \end{equation}
    \item[Descrizione:] Questa metrica rappresenta il valore effettivo del lavoro realizzato alla fine di uno sprint. Il valore è sempre positivo e deve essere comparato a quello dell'actual cost (MPC-4):
    \begin{itemize}
        \item se l'earned value è maggiore significa che è stato speso di meno di quello preventivato per lo sprint;
        \item se l'earned value è minore significa che è stato speso di più di quello preventivato per lo sprint.
    \end{itemize}
\end{description}

\subsection{Actual Cost}
\begin{description}
    \item[Codice:] MPC-4
    \item[Processo:] Fornitura.
    \item[Formula:]
    \begin{equation}
        \sum_{i=1}^{nSprint} \text{Budget Consuntivato}_{i}
    \end{equation}
    \item[Descrizione:] Questa metrica rappresenta il costo totale effettivamente sostenuto in base al lavoro eseguito nello sprint.
\end{description}

\subsection{Planned Value}
\begin{description}
    \item[Codice:] MPC-5
    \item[Processo:] Fornitura.
    \item[Formula:]
    \begin{equation}
        \text{Acutal Cost}_{sprint - 1} + \text{Budget Preventivato}_{sprint} 
    \end{equation}
    \item[Descrizione:] Questa metrica rappresenta il totale dei costi pianificati allo sprint e viene calcolata prima che esso inizi.
\end{description}

\subsection{Cost Variance}
\begin{description}
    \item[Codice:] MPC-6
    \item[Processo:] Fornitura.
    \item[Formula:] 
    \begin{equation}
        \text{Earned Value} - \text{Actual Cost}
    \end{equation}
    \item[Descrizione:] Questa metrica rappresenta lo scostamento dai costi pianificati. Un valore positivo indica che il lavoro effettivamente prodotto è costato meno di quello preventivato. Un valore negativo indica che il progetto sta costando più del previsto.
\end{description}

\subsection{Schedule Variance}
\begin{description}
    \item[Codice:] MPC-7
    \item[Processo:] Fornitura.
    \item[Formula:]
    \begin{equation}
        \text{Earned Value} - \text{Planned Value}
    \end{equation}
    \item[Descrizione:] Questa metrica rappresenta lo scostamento dai tempi pianificati. Un valore positivo indica che il progetto è in anticipo rispetto a quanto preventivato. Un valore negativo indica che si è svolto meno lavoro rispetto a quello previsto.
\end{description}

\subsection{Cost Performance Index}
\begin{description}
    \item[Codice:] MPC-8
    \item[Processo:] Fornitura.
    \item[Formula:]
    \begin{equation}
        \frac{\text{Earned Value}}{\text{Actual Cost}}
    \end{equation}
    \item[Descrizione:] Questa metrica rappresenta l'indice di efficienza economica del progetto. Un valore inferiore ad 1 indica che il progetto è in sovra-budget ovvero che sta richiedendo un budget maggiore rispetto a quello preventivato. Un valore superiore ad 1 indica che il progetto è in sotto-budget ovvero che sta richiedendo un budget minore rispetto a quello preventivato.
\end{description}

\subsection{Schedule Performance Index}
\begin{description}
    \item[Codice:] MPC-9
    \item[Processo:] Fornitura.
    \item[Formula:]
    \begin{equation}
        \frac{\text{Earned Value}}{\text{Planned Value}}
    \end{equation}
    \item[Descrizione:] Questa metrica rappresenta l'indice di efficienza temporale del progetto. Un valore inferiore ad 1 indica che il progetto sta procedendo più lentamente di quanto preventivato. Un valore superiore ad 1 indica che il progetto sta procedendo più velocemente di quanto preventivato.
\end{description}

\subsection{Estimate to Complete}
\begin{description}
    \item[Codice:] MPC-10
    \item[Processo:] Fornitura.
    \item[Formula:] 
    \begin{equation}
        \frac{\text{Budget at Completion} - \text{Earned Value}}{\text{Cost Performance Index}}
    \end{equation}
    \item[Descrizione:] Questa metrica rappresenta il costo totale ancora da sostenere per il completamento del progetto.
\end{description}

\subsection{Estimate at Completion}
\begin{description}
    \item[Codice:] MPC-11
    \item[Processo:] Fornitura.
    \item[Formula:]
    \begin{equation}
        \text{Actual Cost} + \text{Estimate to Complete}
    \end{equation}
    \item[Descrizione:] Questa metrica rappresenta il costo totale alla fine del progetto in base all'andamento attuale.
\end{description}

\subsection{Budget at Completion}
\begin{description}
    \item[Codice:] MPC-12
    \item[Processo:] Fornitura.
    \item[Formula:] Non è presenta una formula.
    \item[Descrizione:] Questa metrica rappresenta il budget totale del progetto.
\end{description}

\subsection{Code Coverage}
\begin{description}
    \item[Codice:] MPC-13
    \item[Processo:] Sviluppo.
    \item[Formula:] Non è presenta una formula.
    \item[Descrizione:] Questa metrica rappresenta la percentuale di codice attraversato dei test rispetto al totale della code base.
\end{description}

\subsection{Misure di mitigazione insufficienti}
\begin{description}
    \item[Codice:] MPC-14
    \item[Processo:] Risoluzione dei problemi.
    \item[Formula:] Non è presenta una formula.
    \item[Descrizione:] Questa metrica rappresenta il numero totale di misure di mitigazione previste che si sono rivelate insufficienti.
\end{description}

\subsection{Rischi inattesi}
\begin{description}
    \item[Codice:] MPC-15
    \item[Processo:] Risoluzione dei problemi.
    \item[Formula:] Non è presenta una formula.
    \item[Descrizione:] Questa metrica rappresenta il numero totale di rischi inattesi (non analizzati) che si sono verificati.
\end{description}
\newpage
\section{Qualità di prodotto} \label{sec:qualityprod}

{
\setlength{\tabcolsep}{10pt}
\renewcommand{\arraystretch}{1.5}
\rowcolors{2}{oddrow}{evenrow}
\begin{xltabular}{\textwidth}{| l | X | c | c |}
    \hline
    \rowcolor{headerrow} \textbf{\textcolor{white}{Codice}} & \textbf{\textcolor{white}{Nome}} & \textbf{\textcolor{white}{Accettabile}} & \textbf{\textcolor{white}{Preferibile}} \\
    \hline
    \endfirsthead
    \hline
    \rowcolor{headerrow} \textbf{\textcolor{white}{Codice}} & \textbf{\textcolor{white}{Nome}} & \textbf{\textcolor{white}{Accettabile}} & \textbf{\textcolor{white}{Preferibile}} \\ 
    \endhead
    MPD-1 & Indice di Gulpease & \ge 40 & \ge 60 \\
    \hline
    MPD-2 & Requisiti obbligatori soddisfatti & 100\% & 100\% \\
    \hline
    MPD-3 & Requisiti desiderabili soddisfatti & \ge 0\% & 100\% \\
    \hline
    MPD-4 & Requisiti opzionali soddisfatti & \ge 0\% & 100\% \\
    \hline
    \rowcolor{white} \caption{Metriche di qualità di prodotto}
    \label{tab:mpd}
\end{xltabular}
}

\subsection{Indice di Gulpease}
\begin{description}
    \item[Codice:] MPD-1
    \item[Processo:] Documentazione.
    \item[Formula:] 
    \begin{equation}
    89 +
    \frac{\text{300 (numero Frasi) - 10 (numero Lettere)}}{\text{numero Parole}}
    \label{MPD-1}
    \end{equation}
    \item[Descrizione:] L'indice di Gulpease misura il grado di leggibilità di un testo, stimando la difficoltà di lettura per una persona in base alla sua istruzione. Calcolato su una scala da 1 a 100, un indice superiore a 80 identifica un testo facilmente comprensibile per chiunque con un'istruzione elementare. Un valore pari o superiore a 60 identifica un testo comprensible per le persone con istruzione media. Un punteggio superiore a 40 indica che il testo è comprensibile alle persone con un'istruzione superiore, dunque al target di questo progetto.\\Il valore minimo accettato farà dunque riferimento a quest'ultima soglia, con un valore preferibile fissato a 60 o più.
\end{description}

\subsection{Requisiti obbligatori soddisfatti}
\begin{description}
    \item[Codice:] MPD-2
    \item[Processo:] Sviluppo.
    \item[Formula:]
    \begin{equation}
        100\biggl(\frac{\text{numero Requisiti Obbligatori Soddisfatti}}{\text{numero Requisiti Obbligatori}}\biggr)
    \end{equation}
    \item[Descrizione:] Indica la percentuale dei requisiti obbligatori soddisfatti.\\ In quanto obbligatori, non è definita alcuna soglia di tolleranza tra valore accettabile e preferibile. È necessario giungere al 100\% di completamento.
\end{description}

\subsection{Requisiti desiderabili soddisfatti}
\begin{description}
    \item[Codice:] MPD-3
    \item[Processo:] Sviluppo.
    \item[Formula:]
    \begin{equation}
        100\biggl(\frac{\text{numero Requisiti Desiderabili Soddisfatti}}{\text{numero Requisiti Desiderabili}}\biggr)
    \end{equation}
    \item[Descrizione:] Indica la percentuale dei requisiti desiderabili soddisfatti.\\ In quanto non obbligatori, il valore accettabile è fissato a 0\%, con valore preferibile 100\%.
\end{description}

\subsection{Requisiti opzionali soddisfatti}
\begin{description}
    \item[Codice:] MPD-4
    \item[Processo:] Sviluppo.
    \item[Formula:]
    \begin{equation}
        100\biggl(\frac{\text{numero Requisiti Opzionali Soddisfatti}}{\text{numero Requisiti Opzionali}}\biggr)
    \end{equation}
    \item[Descrizione:] Indica la percentuale dei requisiti opzionali soddisfatti.\\ In quanto non obbligatori, il valore accettabile è fissato a 0\%, con valore preferibile 100\%.
\end{description}
\newpage
\section{Campagna di verifica} \label{sec:test}
In questa sezione viene defintita la campagna di verifica tramite i test finalizzati a dimostrare l'adempimento, da parte del prodotto del progetto, ai requisiti individuati nella relativa fase di analisi.\\
Tali test, coerentemente a quanto prefigge il V model e il documento Norme\_di\_progetto\_v2.0, sono individuati parallelamente alle attività di sviluppo.\\
Ogni test sarà identificato da un codice alfanumerico, da una descrizione generale, dall'identificazione del requisito da cui tale test deriva e dal tracciamento della implementazione, o meno, di tale nel prodotto.

\subsection{Test di accettazione}

\begingroup
\setlength{\tabcolsep}{10pt}
\renewcommand{\arraystretch}{1.5}
\rowcolors{2}{oddrow}{evenrow}
\begin{xltabular}{\textwidth}{| c | X | X | c |}
    \hline
    \rowcolor{headerrow} \textbf{\textcolor{white}{Codice}} & \textbf{\textcolor{white}{Descrizione}} & \textbf{\textcolor{white}{Fonte}} & \textbf{\textcolor{white}{Stato}}\\
    \hline
    \endhead
%%TA-0 & Verificare che l’utente, all'avvio dell'applicazione, possa selezionare se  recarsi nella parte di chat; recarsi nella parte dei documenti; & UC-0, UC-0.0 & \ding{55} \\
    \hline
    
    TA-1 & Verificare che l’utente possa selezionare un modello di LLM con cui interagire. & UC-1,\newline UC-2 & \textcolor{xmarkcolor}{\ding{55}}  \\
    \hline
    TA-2 & Verificare che l’utente possa visualizzare i documenti presenti sul modello selezionato.
     & UC-3 &\textcolor{xmarkcolor}{\ding{55}}  \\
    \hline
    TA-3 & Verificare che l’utente, una volta visualizzata la lista dei documenti, possa: \begin{enumerate}
        \item selezionare un singolo documento;
        \item visualizzare il nome del documento;
        \item visualizzare la data di inserimento del documento;
        \item visualizzare il contenuto del documento;
        \item visualizzare lo stato del documento;
        \item visualizzare la dimensione del documento.
    \end{enumerate} & UC-4, \newline UC-4.1,\newline UC-4.2,\newline UC-4.3,\newline UC-4.4,\newline UC-4.5,\newline UC-4.6 & \textcolor{xmarkcolor}{\ding{55}}  \\
    \hline
    TA-4 & Verificare che l’utente quando effettua una ricerca tra i documenti visualizzati possa:
    \begin{enumerate}
        \item visualizzare la barra di ricerca;
        \item inserire il nome del documento;
        \item inserire la data di caricamento del documento;
        \item inserire il tag del documento.
    \end{enumerate}& UC-5, \newline UC-5.1,\newline UC-5.2,\newline UC-5.3 & \textcolor{xmarkcolor}{\ding{55}}  \\
    \hline
     TA-5 & Verificare che l’utente quando voglia assegnare un tag ad un documento possa:
    \begin{enumerate}
        \item visualizzare i documenti caricati;
        \item selezionare un documento senza tag;
        \item visualizzare la lista di tag presenti;
        \item inserire il tag al documento.
    \end{enumerate}& UC-6 & \textcolor{xmarkcolor}{\ding{55}}  \\
    \hline
    TA-6 & Verificare che l’utente, quando voglia rimuovere un tag ad un documento, possa:
    \begin{enumerate}
        \item visualizzare i documenti caricati;
        \item selezionare un documento con un tag associato;
        \item rimuovere il tag al documento.
    \end{enumerate}& UC-7 & \textcolor{xmarkcolor}{\ding{55}}  \\
    \hline
     TA-7 & Verificare che l’utente quando voglia creare un tag possa:
    \begin{enumerate}
        \item visualizzare i tag già esistenti nel sistema;
        \item inserire il nome del nuovo tag;
        \item inserire il colore del nuovo tag;
        \item inserire la descrizione del nuovo tag.
    \end{enumerate}& UC-8,\newline UC-8.1,\newline UC-8.2, \newline UC-8.3 & \textcolor{xmarkcolor}{\ding{55}} \\
    \hline
    TA-8 & Verificare che l’utente possa:
    \begin{enumerate}
        \item visualizzare i tag già esistenti nel sistema;
        \item visualizzare un singolo;
        \item visualizzare il colore del tag;
        \item visualizzare la descrizione del tag.
    \end{enumerate}& UC-9,\newline UC-10,\newline UC-10.1,\newline UC-10.2,\newline UC-10.3 & \textcolor{xmarkcolor}{\ding{55}} \\
    \hline
     TA-9 & Verificare che l’utente possa:
    \begin{enumerate}
        \item visualizzare i tag già esistenti nel sistema;
        \item selezionare un singolo tag;
        \item rimuovere il tag selezionato dal sistema.
        
    \end{enumerate}& UC-11 & \textcolor{xmarkcolor}{\ding{55}} \\
    \hline
     TA-10 & Verificare che l’utente possa:
    \begin{enumerate}
        \item rimuovere un documento presente nel sistema;
        \item confermare la rimozione del documento.
        
    \end{enumerate}& UC-12,\newline UC-12.1 & \textcolor{xmarkcolor}{\ding{55}} \\
    \hline
    TA-11 & Verificare che l’utente, quando carica un documento, possa:
    \begin{enumerate}
        \item caricare un documento tramite file system;
        \item caricare un documento tramite trascinamento.
        
    \end{enumerate}& UC-13,\newline UC-13.1,\newline UC-13.2 & \textcolor{xmarkcolor}{\ding{55}} \\
    \hline
    TA-12 & Verificare che il sistema, quando deve inviare un messaggio d'errore in caricamento file all'utente, possa:
    \begin{enumerate}
        \item mostrare un messaggio d'errore nel caso il nome del file sia già presente;
        \item mostrare un messaggio d'errore nel caso il formato del file non sia accettato;
        \item mostrare un messaggio d'errore nel caso il file sia corrotto.
        
    \end{enumerate}& UC-14, \newline UC-14.1,\newline UC-14.2,\newline UC-14.3 & \textcolor{xmarkcolor}{\ding{55}} \\
    \hline

    TA-13 & Verificare che l’utente, quando deve modificare lo stato di un documento, possa:
    \begin{enumerate}
        \item bloccare un documento;
        \item sbloccare un documento.
        
    \end{enumerate}& UC-15,\newline UC-16 & \textcolor{xmarkcolor}{\ding{55}} \\
    \hline
    
    TA-14 & Verificare che l’utente, prima di inviare domande al chatbot, possa:
    \begin{enumerate}
        \item visualizzare la lista delle lingue supportate;
        \item selezionare una lingua.
        
    \end{enumerate}& UC-17,\newline UC-18 & \textcolor{xmarkcolor}{\ding{55}} \\
    \hline

    TA-15 & Verificare che l’utente, quando deve inviare domande al chatbot, possa:
    \begin{enumerate}
        \item digitare la domanda in una barra di digitazione;
        \item inserire la domanda tramite input vocale;
        \item visualizzare un errore a causa della mancata ricezione dell'input vocale.
        
    \end{enumerate}& UC-19,\newline UC-19.1,\newline UC-19.2,\newline UC-20 & \textcolor{xmarkcolor}{\ding{55}} \\
    \hline
    TA-16 & Verificare che l’utente, una volta inviata una domanda, possa:
    \begin{enumerate}
        \item visualizzare la risposta;
        \item visualizzare una risposta di cortesia;
        \item visualizzare un errore a causa del mancato invio di una risposta.

    \end{enumerate}& UC-21,\newline UC-21.1,\newline UC-21.2,\newline UC-22 & \textcolor{xmarkcolor}{\ding{55}} \\
    \hline
     TA-17 & Verificare che l’utente possa:
    \begin{enumerate}
        \item creare una nuova sessione;
        \item visualizzare la lista di sessioni aperte;
        \item visualizzare una singola sessione;
        \item eliminare una sessione;
        \item confermare l'eliminazione di una sessione.
        
        
    \end{enumerate}& UC-23,\newline UC-24,\newline UC-24.1,\newline UC-25,\newline UC-25.1 & \textcolor{xmarkcolor}{\ding{55}} \\
    \hline
    TA-18 & Verificare che l'utente, dopo aver selezionato una sessione, possa:
    \begin{enumerate}
        \item visualizzare la chat history;
        \item eliminare la chat history;
        \item confermare l'eliminazione della chat history.
        
    \end{enumerate}& UC-26,\newline UC-27,\newline UC-27.1 & \textcolor{xmarkcolor}{\ding{55}}  \\
    \hline
    
     TA-19 & Verificare che l'utente, visualizzando una risposta, possa:
    \begin{enumerate}
        \item visualizzare le informazioni della risposta;
        \item visualizzare il documento da cui sono tratte;
        \item visualizzare la pagina del documento da cui sono tratte;
        \item ascoltare la risposta.
        
    \end{enumerate}& UC-28,\newline UC-28.1,\newline UC-28.2,\newline UC-29 & \textcolor{xmarkcolor}{\ding{55}} \\
    \hline

    \rowcolor{white} \caption{Insieme dei test di accettazione} 
    
  %  \label{tab:testActzn}
\end{xltabular}
\endgroup

\newpage
\subsection{Test di sistema}

\begingroup
\setlength{\tabcolsep}{10pt}
\renewcommand{\arraystretch}{1.5}
\rowcolors{2}{oddrow}{evenrow}
\begin{xltabular}{\textwidth}{| c | X | c | c |}
    \hline
    \rowcolor{headerrow} \textbf{\textcolor{white}{Codice}} & \textbf{\textcolor{white}{Descrizione}} & \textbf{\textcolor{white}{Requisito}} & \textbf{\textcolor{white}{Soddisfatto}}\\
    \hline
    \endhead
    TS-1 & Verificare che l’utente possa visualizzare la lista di tutti i modelli LLM supportati dal sistema. & RFO-1 & \textcolor{xmarkcolor}{\ding{55}} \\
    \hline
    TS-2 & Verificare che l’utente possa selezionare il LLM che il sistema deve utilizzare per le operazioni sui documenti e per la generazione delle risposte. & RFO-2 & \textcolor{xmarkcolor}{\ding{55}} \\
    \hline
    TS-3 & Verificare che l'utente possa visualizzare la lista dei documenti presenti nel sistema & RFO-3 & \textcolor{xmarkcolor}{\ding{55}} \\
    \hline
    TS-4 & Verificare che l'utente possa visualizzare delle informazioni per ciascun documento presente nella lista & RFO-4 & \textcolor{xmarkcolor}{\ding{55}} \\
    \hline
    TS-5 & Verificare che l’utente possa visualizzare il nome del documento di interesse. & RFO-5 & \textcolor{xmarkcolor}{\ding{55}} \\
    \hline
    TS-6 & Verificare che l’utente possa visualizzare la data di inserimento del documento di interesse. & RFO-6 & \textcolor{xmarkcolor}{\ding{55}} \\
    \hline
    TS-7 & Verificare che l’utente possa visualizzare i tag applicati al documento di interesse. & RFZ-7 & \textcolor{xmarkcolor}{\ding{55}} \\
    \hline
    TS-8 & Verificare che l’utente possa visualizzare il contenuto del documento di interesse. & RFO-8 & \textcolor{xmarkcolor}{\ding{55}} \\
    \hline
    TS-9 & Verificare che l’utente possa visualizzare lo stato del documento di interesse (bloccato o non bloccato). & RFD-9 & \textcolor{xmarkcolor}{\ding{55}} \\
    \hline
    TS-10 &  Verificare che l’utente possa visualizzare la dimensione del documento di interesse. & RFO-10 & \textcolor{xmarkcolor}{\ding{55}} \\
    \hline
    TS-11 & Verificare che l’utente possa ricercare un documento. & RFO-11 & \textcolor{xmarkcolor}{\ding{55}} \\
    \hline
    TS-12 & Verificare che l’utente possa ricercare un documento per nome. & RFO-12 & \textcolor{xmarkcolor}{\ding{55}} \\
    \hline
    TS-13 & Verificare che l’utente possa ricercare un documento per data di inserimento nel sistema. & RFO-13 & \textcolor{xmarkcolor}{\ding{55}} \\
    \hline
    TS-14 & Verificare che l’utente possaricercare un documento per i propri tag. & RFZ-14 & \textcolor{xmarkcolor}{\ding{55}} \\
    \hline
    TS-15 & Verificare che l’utente possa aggiungere un tag ad un documento. & RFZ-15 & \textcolor{xmarkcolor}{\ding{55}} \\
    \hline
    TS-16 & Verificare che l’utente possa rimuovere un tag da un documento a cui è associato. & RFZ-16 & \textcolor{xmarkcolor}{\ding{55}} \\
    \hline
    TS-17 & Verificare che l’utente possa creare un nuovo tag da salvare nel sistema. & RFZ-17 & \textcolor{xmarkcolor}{\ding{55}} \\
    \hline
    TS-18 & Verificare che l’utente possa aggiungere un nome al tag durante la sua creazione. & RFZ-18 & \textcolor{xmarkcolor}{\ding{55}} \\
    \hline
    TS-19 & Verificare che l’utente possa aggiungere un colore al tag durante la sua creazione. & RFZ-19 & \textcolor{xmarkcolor}{\ding{55}} \\
    \hline
    TS-20 & Verificare che l’utente possa aggiungere una descrizione al tag durante la sua creazione. & RFZ-20 & \textcolor{xmarkcolor}{\ding{55}} \\
    \hline
    TS-21 &  Verificare che l’utente possa visualizzare la lista di tutti i tag presenti nel sistema e associabili a un documento. & RFZ-21 & \textcolor{xmarkcolor}{\ding{55}} \\
    \hline
    TS-22 & Verificare che l'utente possa visualizzare delle informazione per ogni tag presente nel sistema. & RFZ-22 & \textcolor{xmarkcolor}{\ding{55}} \\
    \hline
    TS-23 & Verificare che l'utente possa visualizzare il nome di ogni tag presente nel sistema. & RFZ-23 & \textcolor{xmarkcolor}{\ding{55}} \\
    \hline
    TS-24 & Verificare che l’utente possa visualizzare il colore di ogni tag presente nel sistema. & RFZ-24 & \textcolor{xmarkcolor}{\ding{55}} \\
    \hline
    TS-25 & Verificare che l’utente possa visualizzare la descrizione di ogni tag presente nel sistema. & RFZ-25 & \textcolor{xmarkcolor}{\ding{55}} \\
    \hline
    TS-26 & Verificare che l’utente possa eliminare uno dei tag presenti nel sistema e con esso tutte le associazioni a ogni documento. & RFZ-26 & \textcolor{xmarkcolor}{\ding{55}} \\
    \hline
    TS-27 & Verificare che l’utente possa eliminare uno dei documenti presenti nel sistema & RFO-27 & \textcolor{xmarkcolor}{\ding{55}} \\
    \hline
    TS-28 & Verificare che l’utente possa confermare l’eliminazione di uno dei documenti presenti nel sistema, eliminando definitivamente  ogni informazione associata a quel documento. & RFO-28 & \textcolor{xmarkcolor}{\ding{55}} \\
    \hline
    TS-29 & Verificare che l’utente possa aggiungere un documento nel sistema. & RFO-29 & \textcolor{xmarkcolor}{\ding{55}} \\
    \hline
    TS-30 & Verificare che l’utente possa aggiungere un documento nel sistema tramite trascinamento (drag and drop). & RFD-30 & \textcolor{xmarkcolor}{\ding{55}} \\
    \hline
    TS-31 & Verificare che l’utente possa aggiungere un documento nel sistema tramite navigazione del file system. & RFO-31 & \textcolor{xmarkcolor}{\ding{55}} \\
    \hline
    TS-32 & Verificare che l'utente possa visualizzare un messaggio che lo informa che non è stato possibile inserire il documento. & RFD-32 & \textcolor{xmarkcolor}{\ding{55}} \\
    \hline
    TS-33 & Verificare che l'utente possa visualizzare un messaggio che lo informa che non è stato possibile inserire il documento a causa del nome del file già in uso. & RFD-33 & \textcolor{xmarkcolor}{\ding{55}} \\
    \hline
    TS-34 & Verificare che l’utente possa visualizzare un messaggio che lo informa che non è stato possibile inserire il documento a causa del formato del file non supportato. & RFD-34 & \textcolor{xmarkcolor}{\ding{55}} \\
    \hline
    TS-35 & Verificare che l’utente possa visualizzare un messaggio che lo informa che non è stato possibile inserire il documento a causa della corruzione del file. & RFD-35 & \textcolor{xmarkcolor}{\ding{55}} \\
    \hline
    TS-36 & Verificare che l'utente possa bloccare un documento, così che il sistema non fornisca risposte su tale documento senza doverlo eliminare. & RFD-36 & \textcolor{xmarkcolor}{\ding{55}} \\
    \hline
    TS-37 & Verificare che l'utente possa sbloccare un documento in precedenza bloccato, così che il sistema possa nuovamente fornire risposte su quel particolare documento. & RFD-37 & \textcolor{xmarkcolor}{\ding{55}} \\
    \hline
    TS-38 & Verificare che l’utente possa visualizzare la lista delle lingue supportate dal chatbot. & RFD-38 & \textcolor{xmarkcolor}{\ding{55}} \\
    \hline
    TS-39 &  Verificare che l'utente possa selezionare la lingua utilizzata dal sistema nel fornire le risposte alle sue domande. & RFD-39 & \textcolor{xmarkcolor}{\ding{55}} \\
    \hline
    TS-40 & Verificare che l'utente possa inviare una domanda da porre al chatbot. & RFO-40 & \textcolor{xmarkcolor}{\ding{55}} \\
    \hline
    TS-41 & Verificare che l'utente possa digitare la domanda da porgere al chatbot tramite tastiera. & RFO-41 & \textcolor{xmarkcolor}{\ding{55}} \\
    \hline
    TS-42 & Verificare che l’utente possa inserire la domanda da porgere al chatbot tramite microfono. & RFD-42 & \textcolor{xmarkcolor}{\ding{55}} \\
    \hline
    TS-43 & Verificare che il sistema, dopo non aver registrato alcun input vocale nel tempo limite a seguito del tentativo da parte dell'utente di inserire una domanda tramite microfono, deve notificare un messaggio all'utente che avvisa la mancata trascrizione della domanda. & RFD-43 & \textcolor{xmarkcolor}{\ding{55}} \\
    \hline
    TS-44 & Verificare che l’utente possa visualizzare la risposta alla domanda che ha inviato in precedenza. &  RFO-44 & \textcolor{xmarkcolor}{\ding{55}} \\
    \hline
    TS-45 & Verificare che l’utente possa visualizzare la risposta alla domanda che ha inviato in precedenza, qualora l'informazione sia contenuta all'interno di uno dei documenti presenti nel sistema. &  RFO-45 & \textcolor{xmarkcolor}{\ding{55}} \\
    \hline
    TS-46 & Verificare che l’utente possa visualizzare una risposta di cortesia prodotta dal sistema dopo la ricezione di una domanda non pertinente con alcuna informazione presente in tutti i documenti. & RFO-46 & \textcolor{xmarkcolor}{\ding{55}} \\
    \hline
    TS-47 & Verificare che l'utente possa visualizzare un messaggio che lo informa che c'è stato un errore nel ricevere la risposta entro il tempo limite. & RFO-47 & \textcolor{xmarkcolor}{\ding{55}} \\
    \hline
    TS-48 & Verificare che l'utente possa creare una nuova sessione di conversazione col chatbot. & RFD-48 & \textcolor{xmarkcolor}{\ding{55}} \\
    \hline
    TS-49 & Verificare che l'utente possa visualizzare la lista delle sessioni di conversazione col chatbot attive. & RFD-49 & \textcolor{xmarkcolor}{\ding{55}} \\
    \hline
    TS-50 & Verificare che l'utente possa eliminare una delle sessioni di conversazioni attive nel sistema e con essa tutti i messaggi scambiati in quella conversazione. & RFD-50 & \textcolor{xmarkcolor}{\ding{55}} \\
    \hline
    TS-51 & Verificare che l'utente possa confermare l’eliminazione di una sessione di conversazione e solo dopo deve avvenire l'eliminazione effettiva dei dati associati. & RFD-51 & \textcolor{xmarkcolor}{\ding{55}} \\
    \hline
    TS-52 & Verificare che l'utente possa visualizzare lo scambio di domande e risposte avvenuto in precedenza con il sistema in una stessa sessione & RFO-52 & \textcolor{xmarkcolor}{\ding{55}} \\
    \hline
    TS-53 & Verificare che l'utente possa eliminare lo scambio di domande e risposte avvenuto in precedenza con il chatbot in una stessa sessione. & RFO-53 & \textcolor{xmarkcolor}{\ding{55}} \\
    \hline
    TS-54 & Verificare che l'utente possa confermare l'eliminazione dello scambio di domande e risposte, avvenuto in precedenza con il chatbot in una stessa sessione. & RFO-54 & \textcolor{xmarkcolor}{\ding{55}} \\
    \hline
    TS-55 & Verificare che l'utente possa visualizzare le fonti relative alla risposta. & RFO-55 & \textcolor{xmarkcolor}{\ding{55}} \\
    \hline
    TS-56 & Verificare che l'utente possa visualizzare il nome del documento da cui deriva la risposta. & RFO-56 & \textcolor{xmarkcolor}{\ding{55}} \\
    \hline
    TS-57 & Verificare che l'utente possa visualizzare il numero della pagina del documento da cui deriva la risposta. & RFD-57 & \textcolor{xmarkcolor}{\ding{55}} \\
    \hline
    TS-58 & Verificare che l'utente possa sentire la lettura della risposta ricevuta. & RFD-58 & \textcolor{xmarkcolor}{\ding{55}} \\
    \hline
    TS-59 & Verificare che il sistema garantisca la creazione di almeno una sessione di conversazione. & RFO-59 & \textcolor{xmarkcolor}{\ding{55}} \\
    \hline
    TS-60 & Verificare che il sistema garantisca la creazione di almeno due sessioni di conversazione. & RFD-60 & \textcolor{xmarkcolor}{\ding{55}} \\
    \hline
    TS-61 & Verificare che il sistema interrompa in modo automatico il processo di generazione della risposta ad una domanda, qualora esso dovesse impiegare un tempo superiore ai 30 secondi. & RFO-61 & \textcolor{xmarkcolor}{\ding{55}} \\
    \hline
    TS-62 & Verificare che il sistema interrompa in modo automatico la trascrizione della domanda via input vocale, qualora non fosse rilevato alcuna voce per 5 secondi. & RFD-62 & \textcolor{xmarkcolor}{\ding{55}} \\
    \hline
    TS-63 & Verificare che il sistema salvi, in modo persistente, i messaggi scambiati tra utente e chatbot. & RFO-63 & \textcolor{xmarkcolor}{\ding{55}} \\
    \hline
    TS-64 & Verificare che il sistema effettui correttamente una ricerca semantica (semantic search) tra la domanda posta dall'utente e i gli embedding dei documenti caricati restituendo il documento, o pagina di esso, inerente alla domanda. & RFO-64 & \textcolor{xmarkcolor}{\ding{55}} \\
    \hline
    TS-65 & Verificare che il sistema interroghi il LLM scelto in base alla domanda posta dall'utente e al relativo documento. & RFO-65 & \textcolor{xmarkcolor}{\ding{55}} \\
    \hline
    TS-66 & Verificare che il sistema supporti il caricamento di file PDF. & RFO-66 & \textcolor{xmarkcolor}{\ding{55}} \\
    \hline
    TS-67 & Verificare che il sistema supporti il caricamento di file PDF/A. & RFD-67 & \textcolor{xmarkcolor}{\ding{55}} \\
    \hline
    TS-68 & Verificare che il sistema supporti il caricamento di file con formato .docx, prodotti con Microsoft Word 2007 e versioni successive. & RFD-68 & \textcolor{xmarkcolor}{\ding{55}} \\
    \hline
    TS-69 & Verificare che il sistema supporti il caricamento di file con formato .mp3. & RFZ-69 & \textcolor{xmarkcolor}{\ding{55}} \\
    \hline
    TS-70 & Verificare che il sistema supporti il caricamento di file con formato .mp4. & RFZ-70 & \textcolor{xmarkcolor}{\ding{55}} \\
    \hline
    TS-71 & Verificare che il sistema permetta l'utilizzo di LLM tramite OpenAI. & RVO-1 & \textcolor{xmarkcolor}{\ding{55}}\\
    \hline
    TS-72 & Verificare che il sistema permetta il riconoscimento vocale tramite Whisper di OpenAI. & RVD-2 & \textcolor{xmarkcolor}{\ding{55}}\\
    \hline
    TS-73 & Verificare che il sistema permetta l'utilizzo di LLM locali tramite Ollama. & RVO-3 & \textcolor{xmarkcolor}{\ding{55}} \\
    \hline
    TS-74 & Verificare che il sistema funzioni correttamente con la presenza di 1000 documenti, tra i formati supportati, in esso. & RVO-13 & \textcolor{xmarkcolor}{\ding{55}} \\
    \hline
    TS-75 & Verificare che il sistema processi correttamente ogni documento, tra i formati supportati, con dimensione fino ai 500KB. & RVO-14 & \textcolor{xmarkcolor}{\ding{55}} \\
    \hline
    TS-76 & Verificare che il sistema processi correttamente file audio, tra i formati supportati, di dimensione fino a 5MB. & RVZ-15 & \textcolor{xmarkcolor}{\ding{55}} \\
    \hline
    TS-77 & Verificare che il sistema funzioni correttamente nel browser Google Chrome dalla versione 110 e successive. & RVO-16 & \textcolor{xmarkcolor}{\ding{55}} \\
    \hline
    TS-78 & Verificare che il sistema funzioni correttamente nel browser Mozilla Firefox dalla versione 116 e successive. & RVO-17 & \textcolor{xmarkcolor}{\ding{55}} \\
    \hline
    TS-79 & Verificare che il sistema funzioni correttamente nel browser Opera dalla versione 96 e successive. & RVO-18 & \textcolor{xmarkcolor}{\ding{55}} \\
    \hline
    TS-80 & Verificare che il sistema funzioni correttamente nel browser Microsoft Edge dalla versione 110 e successive. & RVO-19 & \textcolor{xmarkcolor}{\ding{55}} \\
    \hline
    TS-81 & Verificare che il sistema processi i documenti che vengono caricati, creandone i loro embedding. & RIO-1 & \textcolor{xmarkcolor}{\ding{55}} \\
    \hline
    TS-82 & Verificare che il sistema salvi, in modo persistente, i documenti caricati a sistema. & RIO-2 & \textcolor{xmarkcolor}{\ding{55}} \\
    \hline
    TS-83 & Verificare che il sistema salvi, in modo persistente, tutte le informazioni relative ai documenti presenti in esso. & RIO-3 & \textcolor{xmarkcolor}{\ding{55}} \\
    \hline
    TS-84 & Verificare che il sistema salvi, in modo persistente, i vettori dei documenti caricati ed embeddizzati dal sistema. & RIO-4 & \textcolor{xmarkcolor}{\ding{55}} \\
    \hline
    \rowcolor{white} \caption{Insieme dei test di sistema}
    \label{tab:test}
\end{xltabular}
\endgroup

\newpage
\subsection{Test di integrazione}

\begingroup
\setlength{\tabcolsep}{10pt}
\renewcommand{\arraystretch}{1.5}
\rowcolors{2}{oddrow}{evenrow}
\begin{xltabular}{\textwidth}{| c | X | c |}
    \hline
    \rowcolor{headerrow} \textbf{\textcolor{white}{Codice}} & \textbf{\textcolor{white}{Descrizione}} & \textbf{\textcolor{white}{Stato}}\\
    \hline
    \endhead
    TI-1 & Verificare che AddChatUsecase chiami correttamente ChatRepository. & \textcolor{cmarkcolor}{\ding{51}} \\
    \hline
    TI-2 & Verificare che AddChatMessagesUsecase chiami correttamente ChatRepository. & \textcolor{cmarkcolor}{\ding{51}} \\
    \hline
    TI-3 & Verificare che DeleteAllChatUsecase chiami correttamente ChatRepository. & \textcolor{cmarkcolor}{\ding{51}} \\
    \hline
    TI-4 & Verificare che DeleteChatUsecase chiami correttamente ChatRepository. & \textcolor{cmarkcolor}{\ding{51}} \\
    \hline
    TI-5 & Verificare che GetChatMessagesUsecase chiami correttamente ChatRepository. & \textcolor{cmarkcolor}{\ding{51}} \\
    \hline
    TI-6 & Verificare che GetChatsUsecase chiami correttamente ChatRepository. & \textcolor{cmarkcolor}{\ding{51}} \\
    \hline
    TI-7 & Verificare che AddDocumentUsecase chiami correttamente DocumentRepository e EmbeddingRepository. & \textcolor{cmarkcolor}{\ding{51}} \\
    \hline
    TI-8 & Verificare che DeleteDocumentUsecase chiami correttamente DocumentRepository e EmbeddingRepository. & \textcolor{cmarkcolor}{\ding{51}} \\
    \hline
    TI-9 & Verificare che UpdateDocumentUsecase chiami correttamente DocumentRepository e EmbeddingRepository. & \textcolor{cmarkcolor}{\ding{51}} \\
    \hline
    TI-10 & Verificare che GetDocumentContentUsecase chiami correttamente DocumentRepository. & \textcolor{cmarkcolor}{\ding{51}} \\
    \hline
    TI-11 & Verificare che GetDocumentsUsecase chiami correttamente DocumentRepository. & \textcolor{cmarkcolor}{\ding{51}} \\
    \hline
    TI-12 & Verificare che AddChatController chiami correttamente AddChatUsecase. & \textcolor{cmarkcolor}{\ding{51}} \\
    \hline
    TI-13 & Verificare che AddChatMessagesController chiami correttamente AddChatMessagesUsecase. & \textcolor{cmarkcolor}{\ding{51}} \\
    \hline
    TI-14 & Verificare che DeleteAllChatController chiami correttamente DeleteAllChatUsecase. & \textcolor{cmarkcolor}{\ding{51}} \\
    \hline
    TI-15 & Verificare che DeleteChatController chiami correttamente DeleteChatUsecase & \textcolor{cmarkcolor}{\ding{51}} \\
    \hline
    TI-16 & Verificare che GetChatMessagesController chiami correttamente GetChatMessagesUsecase. & \textcolor{cmarkcolor}{\ding{51}} \\
    \hline
    TI-17 & Verificare che GetChatsController chiami correttamente GetChatsUsecase. & \textcolor{cmarkcolor}{\ding{51}} \\
    \hline
    TI-18 & Verificare che AddDocumentController chiami correttamente AddDocumentUsecase. & \textcolor{cmarkcolor}{\ding{51}}\\
    \hline
    TI-19 & Verificare che DeleteDocumentController chiami correttamente DeleteDocumentUsecase. & \textcolor{cmarkcolor}{\ding{51}} \\
    \hline
    TI-20 & Verificare che UpdateDocumentController chiami correttamente UpdateDocumentUsecase. & \textcolor{cmarkcolor}{\ding{51}} \\
    \hline
    TI-21 & Verificare che GetDocumentContentController chiami correttamente GetDocumentContentUsecase. & \textcolor{cmarkcolor}{\ding{51}} \\
    \hline
    TI-22 & Verificare che GetDocumentsController chiami correttamente GetDocumentsUsecase. & \textcolor{cmarkcolor}{\ding{51}} \\
    \hline
    TI-23 & Verificare che il metodo addDocument di DocumentRepository chiami correttamente il metodo addOne di MinioDataSource. & \textcolor{cmarkcolor}{\ding{51}} \\
    \hline
    TI-24 & Verificare che il metodo deleteDocument di DocumentRepository chiami correttamente il metodo deleteOne di MinioDataSource. & \textcolor{cmarkcolor}{\ding{51}} \\
    \hline
    TI-25 & Verificare che il metodo updateDocument di DocumentRepository chiami correttamente il metodo updateOne di MinioDataSource. & \textcolor{cmarkcolor}{\ding{51}} \\
    \hline
    TI-26 & Verificare che il metodo getDocuments di DocumentRepository chiami correttamente il metodo getAll di MinioDataSource. & \textcolor{cmarkcolor}{\ding{51}} \\
    \hline
    TI-27 & Verificare che il metodo getDocumentContent di DocumentRepository chiami correttamente il metodo getContent di MinioDataSource. & \textcolor{cmarkcolor}{\ding{51}} \\
    \hline
    TI-28 & Verificare che il metodo addEmbedding di EmbeddingRepository  chiami correttamente il metodo addOne di ChromaDataSource. & \textcolor{cmarkcolor}{\ding{51}} \\
    \hline
    TI-29 & Verificare che il metodo deleteEmbedding di EmbeddingRepository  chiami correttamente il metodo deleteOne di ChromaDataSource. & \textcolor{cmarkcolor}{\ding{51}} \\
    \hline
    TI-30 & Verificare che il metodo updateMetadatas di EmbeddingRepository  chiami correttamente il metodo updateOne di ChromaDataSource. & \textcolor{cmarkcolor}{\ding{51}} \\
    \hline
    TI-31 & Verificare che il metodo getIdsEmbedding di EmbeddingRepository  chiami correttamente il metodo getIds di ChromaDataSource. & \textcolor{cmarkcolor}{\ding{51}} \\
    \hline
    TI-32 & Verificare che il metodo addChat di ChatRepository  chiami correttamente il metodo addOne di PostgresDataSource. & \textcolor{cmarkcolor}{\ding{51}} \\
    \hline
    TI-33 & Verificare che il metodo addChatMessages di ChatRepository  chiami correttamente il metodo addMessages di PostgresDataSource. & \textcolor{cmarkcolor}{\ding{51}} \\
    \hline
    TI-34 & Verificare che il metodo deleteAllChat di ChatRepository  chiami correttamente il metodo deleteAll di PostgresDataSource. & \textcolor{cmarkcolor}{\ding{51}} \\
    \hline
    TI-35 & Verificare che il metodo deleteChat di ChatRepository  chiami correttamente il metodo deleteOne di PostgresDataSource. & \textcolor{cmarkcolor}{\ding{51}} \\
    \hline
    TI-36 & Verificare che il metodo getChatMessages di ChatRepository  chiami correttamente il metodo getAllMessages di PostgresDataSource. & \textcolor{cmarkcolor}{\ding{51}} \\
    \hline
    TI-37 & Verificare che il metodo getChats di ChatRepository  chiami correttamente il metodo getAll di PostgresDataSource. & \textcolor{cmarkcolor}{\ding{51}} \\
    \hline
    \rowcolor{white} \caption{Insieme dei test di integrazione}
    \label{tab:testinter}
\end{xltabular}

\subsubsection{Tracciamento test di integrazione}

\begingroup
\setlength{\tabcolsep}{10pt}
\renewcommand{\arraystretch}{1.5}
\rowcolors{2}{oddrow}{evenrow}
\begin{xltabular}{0.7\textwidth}{| c | X |}
    \hline
    \rowcolor{headerrow} \textbf{\textcolor{white}{Codice}} & \textbf{\textcolor{white}{Nome suite test}} \\
    \hline
    \endhead
    TI-1 & AddChatUsecase.test.ts \\
    \hline
    TI-2 & AddChatMessagesUsecase.test.ts \\
    \hline
    TI-3 & DeleteAllChatUsecase.test.ts \\
    \hline
    TI-4 & DeleteChatUsecase.test.ts \\
    \hline
    TI-5 & GetChatMessagesUsecase.test.ts \\
    \hline
    TI-6 & GetChatsUsecase.test.ts \\
    \hline
    TI-7 & AddDocumentUsecase.test.ts \\
    \hline
    TI-8 & DeleteDocumentUsecase.test.ts \\
    \hline
    TI-9 & UpdateDocumentUsecase.test.ts \\
    \hline
    TI-10 & GetDocumentContentUsecase.test.ts \\
    \hline
    TI-11 & GetDocumentsUsecase.test.ts \\
    \hline
    TI-12 & AddChatController.test.ts \\
    \hline
    TI-13 & AddChatMessagesController.test.ts \\
    \hline
    TI-14 & DeleteAllChatController.test.ts \\
    \hline
    TI-15 & DeleteChatController.test.ts \\
    \hline
    TI-16 & GetChatMessagesController.test.ts \\
    \hline
    TI-17 & GetChatsController.test.ts \\
    \hline
    TI-18 & AddDocumentController.test.ts \\
    \hline
    TI-19 & DeleteDocumentController.test.ts \\
    \hline
    TI-20 & UpdateDocumentController.test.ts \\
    \hline
    TI-21 & GetDocumentContentController.test.ts \\
    \hline
    TI-22 & GetDocumentsController.test.ts \\
    \hline
    TI-23 & DocumentRepository.test.ts \\
    \hline
    TI-24 & DocumentRepository.test.ts \\
    \hline
    TI-25 & DocumentRepository.test.ts \\
    \hline
    TI-26 & DocumentRepository.test.ts \\
    \hline
    TI-27 & DocumentRepository.test.ts \\
    \hline
    TI-28 & EmbeddingRepository.test.ts \\
    \hline
    TI-29 & EmbeddingRepository.test.ts \\
    \hline
    TI-30 & EmbeddingRepository.test.ts \\
    \hline
    TI-31 & EmbeddingRepository.test.ts \\
    \hline
    TI-32 & ChatRepository.test.ts \\
    \hline
    TI-33 & ChatRepository.test.ts \\
    \hline
    TI-34 & ChatRepository.test.ts \\
    \hline
    TI-35 & ChatRepository.test.ts \\
    \hline
    TI-36 & ChatRepository.test.ts \\
    \hline
    TI-37 & ChatRepository.test.ts \\
    \hline
    \rowcolor{white} \caption{Tracciamento dei test di integrazione}
    \label{tab:tracctestinter}
\end{xltabular}
\newpage
\subsection{Test di unità}

\begingroup
\setlength{\tabcolsep}{10pt}
\renewcommand{\arraystretch}{1.5}
\rowcolors{2}{oddrow}{evenrow}
\begin{xltabular}{\textwidth}{| c | X | c |}
    \hline
    \rowcolor{headerrow} \textbf{\textcolor{white}{Codice}} & \textbf{\textcolor{white}{Descrizione}} & \textbf{\textcolor{white}{Stato}}\\
    \hline
    \endhead
    TU-1 & Verificare che il componente ChatForm venga renderizzato correttamente. & \textcolor{xmarkcolor}{\ding{55}} \\
    \hline
    TU-2 & Verificare che il componente ChatList venga renderizzato correttamente. & \textcolor{xmarkcolor}{\ding{55}} \\
    \hline
    TU-3 & Verificare che il componente ChatMessages venga renderizzato correttamente. & \textcolor{xmarkcolor}{\ding{55}} \\
    \hline
    TU-4 & Verificare che il componente DeleteChatList venga renderizzato correttamente. & \textcolor{xmarkcolor}{\ding{55}} \\
    \hline
    TU-5 & Verificare che il componente Message venga renderizzato correttamente. & \textcolor{xmarkcolor}{\ding{55}} \\
    \hline
    TU-6 & Verificare che il componente ModelToggle venga renderizzato correttamente. & \textcolor{xmarkcolor}{\ding{55}} \\
    \hline
    TU-7 & Verificare che il componente ThemeToggle venga renderizzato correttamente. & \textcolor{xmarkcolor}{\ding{55}} \\
    \hline
    TU-8 & Verificare che il componente Settings venga renderizzato correttamente. & \textcolor{xmarkcolor}{\ding{55}} \\
    \hline
    TU-9 & Verificare che il componente DataTable venga renderizzato correttamente. & \textcolor{xmarkcolor}{\ding{55}} \\
    \hline
    TU-10 & Verificare che il componente DataTablePagination venga renderizzato correttamente. & \textcolor{xmarkcolor}{\ding{55}} \\
    \hline
    TU-11 & Verificare che il componente DataTableViewOption venga renderizzato correttamente. & \textcolor{xmarkcolor}{\ding{55}} \\
    \hline
    TU-12 & Verificare che il componente DocAction venga renderizzato correttamente. & \textcolor{xmarkcolor}{\ding{55}} \\
    \hline
    TU-13 & Verificare che il componente DocActionDelete venga renderizzato correttamente. & \textcolor{xmarkcolor}{\ding{55}} \\
    \hline
    TU-14 & Verificare che il componente DocForm venga renderizzato correttamente. & \textcolor{xmarkcolor}{\ding{55}} \\
    \hline
    TU-15 & Verificare che il componente DocTable venga renderizzato correttamente. & \textcolor{xmarkcolor}{\ding{55}} \\
    \hline
    TU-16 & Verificare che il componente Body venga renderizzato correttamente. & \textcolor{cmarkcolor}{\ding{51}} \\
    \hline
    TU-17 & Verificare che il componente SideBar venga renderizzato correttamente. & \textcolor{xmarkcolor}{\ding{55}} \\
    \hline
    TU-18 & Verificare che AddChatController restituisca una risposta con status 200 in caso di operazione con esito positivo. & \textcolor{cmarkcolor}{\ding{51}} \\
    \hline
    TU-19 & Verificare che AddChatMessagesController restituisca una risposta con status 200 in caso di operazione con esito positivo. & \textcolor{cmarkcolor}{\ding{51}} \\
    \hline
    TU-20 & Verificare che DeleteAllChatController restituisca una risposta con status 200 in caso di operazione con esito positivo. & \textcolor{cmarkcolor}{\ding{51}} \\
    \hline
    TU-21 & Verificare che DeleteChatController restituisca una risposta con status 200 in caso di operazione con esito positivo. & \textcolor{cmarkcolor}{\ding{51}}\\
    \hline
    TU-22 & Verificare che GetChatMessagesController restituisca una risposta con status 200 in caso di operazione con esito positivo. & \textcolor{cmarkcolor}{\ding{51}} \\
    \hline
    TU-23 & Verificare che GetChatsController restituisca una risposta con status 200 in caso di operazione con esito positivo. & \textcolor{cmarkcolor}{\ding{51}} \\
    \hline
    TU-24 & Verificare che AddDocumentController restituisca una risposta con status 200 in caso di operazione con esito positivo. & \textcolor{cmarkcolor}{\ding{51}} \\
    \hline
    TU-25 & Verificare che DeleteDocumentController restituisca una risposta con status 200 in caso di operazione con esito positivo. & \textcolor{cmarkcolor}{\ding{51}} \\
    \hline
    TU-26 & Verificare che UpdateDocumentController restituisca una risposta con status 200 in caso di operazione con esito positivo. & \textcolor{cmarkcolor}{\ding{51}} \\
    \hline
    TU-27 & Verificare che GetDocumentsController restituisca una risposta con status 200 in caso di operazione con esito positivo. & \textcolor{cmarkcolor}{\ding{51}} \\
    \hline
    TU-28 & Verificare che GetDocumentContentController restituisca una risposta con status 200 in caso di operazione con esito positivo. & \textcolor{cmarkcolor}{\ding{51}} \\
    \hline
    TU-29 & Verificare che AddChatController restituisca una risposta con status 500 in caso di operazione con esito negativo. & \textcolor{cmarkcolor}{\ding{51}} \\
    \hline
    TU-30 & Verificare che AddChatMessagesController restituisca una risposta con status 500 in caso di operazione con esito negativo. & \textcolor{cmarkcolor}{\ding{51}} \\
    \hline
    TU-31 & Verificare che DeleteAllChatController restituisca una risposta con status 500 in caso di operazione con esito negativo. & \textcolor{cmarkcolor}{\ding{51}} \\
    \hline
    TU-32 & Verificare che DeleteChatController restituisca una risposta con status 500 in caso di operazione con esito negativo. & \textcolor{cmarkcolor}{\ding{51}} \\
    \hline
    TU-33 & Verificare che GetChatMessagesController restituisca una risposta con status 500 in caso di operazione con esito negativo. & \textcolor{cmarkcolor}{\ding{51}} \\
    \hline
    TU-34 & Verificare che GetChatsController restituisca una risposta con status 500 in caso di operazione con esito negativo. & \textcolor{cmarkcolor}{\ding{51}} \\
    \hline
    TU-35 & Verificare che AddDocumentController restituisca una risposta con status 500 in caso di operazione con esito negativo. & \textcolor{cmarkcolor}{\ding{51}} \\
    \hline
    TU-36 & Verificare che DeleteDocumentController restituisca una risposta con status 500 in caso di operazione con esito negativo. & \textcolor{cmarkcolor}{\ding{51}} \\
    \hline
    TU-37 & Verificare che UpdateDocumentController restituisca una risposta con status 500 in caso di operazione con esito negativo. & \textcolor{cmarkcolor}{\ding{51}} \\
    \hline
    TU-38 & Verificare che GetDocumentsController restituisca una risposta con status 500 in caso di operazione con esito negativo. & \textcolor{cmarkcolor}{\ding{51}} \\
    \hline
    TU-39 & Verificare che GetDocumentContentController restituisca una risposta con status 500 in caso di operazione con esito negativo. & \textcolor{cmarkcolor}{\ding{51}} \\
    \hline
    TU-40 & Verificare che la server action addChat lanci un errore contenente il messaggio ricevuto dal controller in caso di operazione con esito negativo. & \textcolor{cmarkcolor}{\ding{51}} \\
    \hline
    TU-41 & Verificare che la server action addChatMessages lanci un errore contenente il messaggio ricevuto dal controller in caso di operazione con esito negativo. & \textcolor{cmarkcolor}{\ding{51}} \\
    \hline
    TU-42 & Verificare che la server action deleteAllChat lanci un errore contenente il messaggio ricevuto dal controller in caso di operazione con esito negativo. & \textcolor{cmarkcolor}{\ding{51}} \\
    \hline
    TU-43 & Verificare che la server action deleteChat lanci un errore contenente il messaggio ricevuto dal controller in caso di operazione con esito negativo. & \textcolor{cmarkcolor}{\ding{51}} \\
    \hline
    TU-44 & Verificare che la server action getChatMessages lanci un errore contenente il messaggio ricevuto dal controller in caso di operazione con esito negativo. & \textcolor{cmarkcolor}{\ding{51}} \\
    \hline
    TU-45 & Verificare che la server action getChats lanci un errore contenente il messaggio ricevuto dal controller in caso di operazione con esito negativo. & \textcolor{cmarkcolor}{\ding{51}} \\
    \hline
    TU-46 & Verificare che la server action addDocument lanci un errore contenente il messaggio ricevuto dal controller in caso di operazione con esito negativo. & \textcolor{cmarkcolor}{\ding{51}} \\
    \hline
    TU-47 & Verificare che la server action deleteDocument lanci un errore contenente il messaggio ricevuto dal controller in caso di operazione con esito negativo. & \textcolor{cmarkcolor}{\ding{51}} \\
    \hline
    TU-48 & Verificare che la server action updateDocument lanci un errore contenente il messaggio ricevuto dal controller in caso di operazione con esito negativo. & \textcolor{cmarkcolor}{\ding{51}} \\
    \hline
    TU-49 & Verificare che la server action getDocuments lanci un errore contenente il messaggio ricevuto dal controller in caso di operazione con esito negativo. & \textcolor{cmarkcolor}{\ding{51}} \\
    \hline
    TU-50 & Verificare che la server action getDocumentContent lanci un errore contenente il messaggio ricevuto dal controller in caso di operazione con esito negativo. & \textcolor{cmarkcolor}{\ding{51}} \\
    \hline
    \rowcolor{white} \caption{Insieme dei test di unità}
    \label{tab:testunita}
\end{xltabular}

\endgroup

\subsubsection{Tracciamento test di unità}


\begingroup
\setlength{\tabcolsep}{10pt}
\renewcommand{\arraystretch}{1.5}
\rowcolors{2}{oddrow}{evenrow}
\begin{xltabular}{0.7\textwidth}{| c | X |}
    \hline
    \rowcolor{headerrow} \textbf{\textcolor{white}{Codice}} & \textbf{\textcolor{white}{Nome suite test}} \\
    \hline
    \endhead
    TU-1 & ChatForm.test.tsx \\
    \hline
    TU-2 & ChatList.test.tsx \\
    \hline
    TU-3 & ChatMessages.test.tsx \\
    \hline
    TU-4 & DeleteChatList.test.tsx \\
    \hline
    TU-5 & Message.test.tsx \\
    \hline
    TU-6 & ModelToggle.test.tsx \\
    \hline
    TU-7 & ThemeToggle.test.tsx \\
    \hline
    TU-8 & Settings.test.tsx \\
    \hline
    TU-9 & DataTable.test.tsx \\
    \hline
    TU-10 & DataTablePagination.test.tsx \\
    \hline
    TU-11 & DataTableViewOptions.test.tsx \\
    \hline
    TU-12 & DocAction.test.tsx \\
    \hline
    TU-13 & DocActionDelete.test.tsx \\
    \hline
    TU-14 & DocForm.test.tsx \\
    \hline
    TU-15 & DocTable.test.tsx \\
    \hline
    TU-16 & Body.test.tsx \\
    \hline
    TU-17 & SideBar.test.tsx \\
    \hline
    TU-18 & AddChatController.test.ts \\
    \hline
    TU-19 & AddChatMessagesController.test.ts \\
    \hline
    TU-20 & DeleteAllChatControlle.test.ts \\
    \hline
    TU-21 & DeleteChatController.test.ts \\
    \hline
    TU-22 & GetChatMessagesController.test.ts \\
    \hline
    TU-23 & GetChatsController.test.ts \\
    \hline
    TU-24 & AddDocumentController.test.ts \\
    \hline
    TU-25 & DeleteDocumentController.test.ts \\
    \hline
    TU-26 & UpdateDocumentController.test.ts \\
    \hline
    TU-27 & GetDocumentsController.test.ts \\
    \hline
    TU-28 & GetDocumentContentController.test.ts \\
    \hline
    TU-29 & AddChatController.test.ts \\
    \hline
    TU-30 & AddChatMessagesController.test.ts \\
    \hline
    TU-31 & DeleteAllChatControlle.test.ts \\
    \hline
    TU-32 & DeleteChatController.test.ts \\
    \hline
    TU-33 & GetChatMessagesController.test.ts \\
    \hline
    TU-34 & GetChatsController.test.ts \\
    \hline
    TU-35 & AddDocumentController.test.ts \\
    \hline
    TU-36 & DeleteDocumentController.test.ts \\
    \hline
    TU-37 & UpdateDocumentController.test.ts \\
    \hline
    TU-38 & GetDocumentsController.test.ts \\
    \hline
    TU-39 & GetDocumentContentController.test.ts \\
    \hline
    TU-40 & addChat.test.ts \\
    \hline
    TU-41 & addChatMessages.test.ts \\
    \hline
    TU-42 & deleteAllChat.test.ts \\
    \hline
    TU-43 & deleteChat.test.ts \\
    \hline
    TU-44 & getChatMessages.test.ts \\
    \hline
    TU-45 & getChats.test.ts \\
    \hline
    TU-46 & addDocument.test.ts \\
    \hline
    TU-47 & deleteDocument.test.ts \\
    \hline
    TU-48 & updateDocument.test.ts \\
    \hline
    TU-49 & getDocuments.test.ts \\
    \hline
    TU-50 & getDocumentContent.test.ts \\
    \hline
    \rowcolor{white} \caption{Tracciamento dei test di unità}
    \label{tab:tracctestunita}
\end{xltabular}