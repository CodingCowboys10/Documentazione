\section{Resoconto} \label{sec:resoconto}
\subsection{Discussioni} \label{subsec:resdiscussione}
\begin{enumerate}
    \item Viene modificato il quantitativo di funzionalità che il Proof of Concept dovrà implementare. In particolare, se in un primo momento l'azienda proponente aveva richiesto la possibilità di inserire un documento sul quale interagire con il \ccgloss{chatbot} successivamente alla fase di \ccgloss{embedding}, ora sarà richiesta solo la seconda, con la fase di embedding già fatta.
    \item Il gruppo ha esposto ai rappresentanti della proponente l'intenzione di utilizzare la libreria \ccgloss{React} per il \ccgloss{front-end} dell'applicazione, prendendo il posto del framework Angular inizialmente ipotizzato.\\
    Il gruppo viene informato della possibilità di ricevere in dotazione un account \ccgloss{OpenAI}, utile per comparare i risultati del modello locale \ccgloss{open-source} che sarà utilizzato, andando a creare idealmente due versioni possibili del PoC.
    \item Viene richiesta la redazione di una fase avanzata dell'analisi dei requisiti per la prossima settimana, da mostrare all'azienda al momento della presentazione di un preview del PoC. In particolare, oltre a riportare in modo formale l'analisi dei costi già presentata in forma di bozza, è richiesta l'analisi dei requisiti legati alla qualità dell'output che il chatbot fornisce all'utente. In tal senso, è sottolineata l'importanza di trovare delle metriche per determinare il grado di correttezza dell'output del chatbot, da utilizzare come strumento di misura della qualità del prodotto finale del progetto.
\end{enumerate}
L'incontro termina con la conferma da parte di AzzurroDigitale di voler giungere alla presentazione di un PoC nella prima settimana di dicembre.

\subsection{Prossima riunione} \label{subsec:riunione}
È stata fissata una riunione per la prossima settimana.
