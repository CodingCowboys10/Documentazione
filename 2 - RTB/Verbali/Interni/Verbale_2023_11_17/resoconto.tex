\section{Resoconto} \label{sec:resoconto}
\subsection{Discussioni} \label{subsec:resdiscussione}
\begin{enumerate}
    \item Viene esposto da Francesco Ferraioli il funzionamento del \ccgloss{workflow} e delle automazioni che sono state realizzate in \ccgloss{Jira}.  Viene esposto da Leonardo Lago il funzionamento delle automazioni per il versionamento della documentazione realizzate tramite \ccgloss{GitHub}, \ccgloss{GitHub Action} e uno script in \ccgloss{Python}. Entrambi i processi sono ora completi e pienamente utilizzabili.
    \item Viene discussa l'efficacia e efficienza delle tecnologie fino ad ora trovate. In particolare sono state esaminate le differenze, pregi e difetti dell'utilizzo di un modello \ccgloss{LLM} gratuito in locale (\ccgloss{Ollama}) rispetto ad un modello a pagamento utilizzabile tramite chiamate API (\ccgloss{OpenAI}).
    \item Viene deciso, per migliorare il tracciamento dei compiti, di utilizzare una tabella contente le azioni da intraprendere, a chi sono assegnata e la data di scadenza entro cui devono essere completate.
\end{enumerate}

\subsection{Azioni da intraprendere}
{
    \setlength{\tabcolsep}{10pt}
    \renewcommand{\arraystretch}{1.5}
    \rowcolors{2}{oddrow}{evenrow}
    \begin{tabularx}{\textwidth}{| l | l | l | X |}
         \hline
         \rowcolor{headerrow}\textbf{\textcolor{white}{Codice issue}} & \textbf{\textcolor{white}{Assegnatario}} & \textbf{\textcolor{white}{Scadenza}} & \textbf{\textcolor{white}{Descrizione}} \\
         \hline
         CC-21 & Leonardo Lago & 2023/11/20 & Redazione resoconto verbale interno 2023/11/17 \\
         \hline
    \end{tabularx}
}

\subsection{Prossima riunione} \label{subsec:riunione}
Viene fissata una riunione per lunedì 20 Novembre.
