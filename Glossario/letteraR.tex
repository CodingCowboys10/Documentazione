\chapter{R}

\section{React}
React (noto anche come React.js o ReactJS) è una libreria \emph{open-source\ped{G}}, \emph{front-end\ped{G}}, JavaScript per la creazione di interfacce utente.

\section{Repository}\label{sec:Repo}
Un repository, o repo, è un archivio digitale centralizzato che gli sviluppatori utilizzano per apportare e gestire le modifiche al codice sorgente di un'applicazione. Gli sviluppatori hanno la necessità di archiviare e condividere cartelle, file di testo e altri tipi di documenti durante lo sviluppo del software. Un repo ha caratteristiche che consentono agli sviluppatori di tenere traccia delle modifiche al codice, di modificare simultaneamente i file e di collaborare in modo efficiente allo stesso \emph{progetto\ped{G}} da qualsiasi luogo. 

\section{Requisito}\label{sec:Requisiti}
Un requisito è una descrizione di qualcosa che il sistema dovrà fare, o di un vincolo che il sistema dovrà rispettare durante l’esecuzione dei suoi compiti.

\section{Responsabile}
Chi, nell'ambito di un \emph{progetto\ped{G}}, governa il team e rappresenta il progetto verso l'esterno. Ha responsabilità di scelta e approvazione oltre che di pianificazione, gestione delle risorse, coordinamento e gestione delle relazioni esterne. 

\section{Risorsa}\label{sec:Risorse}
Una risorsa è una qualsiasi entità o elemento utilizzato dal sistema per eseguire operazioni, processi o attività.

\section{RTB}\label{sec:Requirements and Technology Baseline}
Acronimo per Requirements and Technology Baseline. Prima revisione di avanzamento obbligatoria nell'ambito del \emph{progetto\ped{G}} didattico in corso, fissa i requisiti da soddisfare e motiva la tecnologie adottate, dimostrandone adeguatezza e compatibilità.
\section{Ruolo}\label{sec:Ruoli}
Incarico che ogni membro di un team che lavora a un \emph{progetto\ped{G}} può assumere, in particolare i ruoli possibili in questo \emph{progetto\ped{G}} sono \emph{responsabile\ped{G}}, \emph{amministratore\ped{G}}, \emph{analista\ped{G}}, \emph{progettista\ped{G}}, \emph{programmatore\ped{G}} e \emph{verificatore\ped{G}}.
