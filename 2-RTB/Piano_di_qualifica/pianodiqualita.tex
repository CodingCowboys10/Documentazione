\chapter{Piano di qualità}

\section{Introduzione}
\label{sec:qualityintro}
La qualità di un progetto è fortemente influenzata dalla qualità dei processi che lo compongono.\\ Stabilire delle metriche che valutino i processi e ne giudichino la qualità è perciò imperativo per raggiungere standard qualitativi apprezzabili.\\ Verranno indicati, per ogni metrica che lo prevede, il valore accettabile e quello preferibile.

\section{Qualità di processo} \label{sec:qualityproc}
\subsection{Varianza di Budget}
\begin{description}
    \item[Codice:] MPC-1
    \item[Processo:] Fornitura.
    \item[Formula:] 
    \begin{equation}
        100\biggl(\frac{\text{Budget Consuntivato} - \text{Budget Preventivato}}{\text{Budget Preventivato}}\biggr)
    \end{equation}
    \item[Descrizione:] Questa metrica valuta la percentuale di variazione del budget tra preventivo e consuntivo in uno sprint. Il valore è positivo quando viene preventivato un budget inferiore a quello effettivamente utilizzato, mentre è negativo quando viene preventivato un budget maggiore a quello effettivamente utilizzato.
\end{description}

\subsection{Varianza dell’impegno orario}
\begin{description}
    \item[Codice:] MPC-2
    \item[Processo:] Fornitura.
    \item[Formula:]
    \begin{equation}
        100\biggl(\frac{\text{Ore Consuntivate} - \text{Ore Preventivate}}{\text{Ore Preventivate}}\biggr)
    \end{equation}
    \item[Descrizione:] Questa metrica valuta la percentuale di variazione dell'impegno orario complessivo tra preventivo e consuntivo in uno sprint. Il valore è positivo quando viene preventivato un impegno orario inferiore a quello effettivamente svolto, mentre è negativo quando viene preventivato un impegno orario maggiore a quello effettivamente svolto.
\end{description}

\subsection{Earned Value}
\begin{description}
    \item[Codice:] MPC-3
    \item[Processo:] Fornitura.
    \item[Formula:]
    \begin{equation}
        \text{Budget Preventivato} * \% \text{Completamento di Attività Sprint} 
    \end{equation}
    \item[Descrizione:] Questa metrica rappresenta il valore effettivo del lavoro realizzato alla fine di uno sprint. Il valore è sempre positivo e deve essere comparato a quello dell'actual cost (MPC-4):
    \begin{itemize}
        \item se l'earned value è maggiore significa che è stato speso di meno di quello preventivato per lo sprint;
        \item se l'earned value è minore significa che è stato speso di più di quello preventivato per lo sprint.
    \end{itemize}
\end{description}

\subsection{Actual Cost}
\begin{description}
    \item[Codice:] MPC-4
    \item[Processo:] Fornitura.
    \item[Formula:]
    \begin{equation}
        \sum_{i=1}^{nSprint} \text{Budget Consuntivato}_{i}
    \end{equation}
    \item[Descrizione:] Questa metrica rappresenta il costo totale effettivamente sostenuto in base al lavoro eseguito nello sprint.
\end{description}

\subsection{Planned Value}
\begin{description}
    \item[Codice:] MPC-5
    \item[Processo:] Fornitura.
    \item[Formula:]
    \begin{equation}
        \text{Acutal Cost}_{sprint - 1} + \text{Budget Preventivato}_{sprint} 
    \end{equation}
    \item[Descrizione:] Questa metrica rappresenta il totale dei costi pianificati allo sprint e viene calcolata prima che esso inizi.
\end{description}

\subsection{Cost Variance}
\begin{description}
    \item[Codice:] MPC-6
    \item[Processo:] Fornitura.
    \item[Formula:] 
    \begin{equation}
        \text{Earned Value} - \text{Actual Cost}
    \end{equation}
    \item[Descrizione:] Questa metrica rappresenta lo scostamento dai costi pianificati. Un valore positivo indica che il lavoro effettivamente prodotto è costato meno di quello preventivato. Un valore negativo indica che il progetto sta costando più del previsto.
\end{description}

\subsection{Schedule Variance}
\begin{description}
    \item[Codice:] MPC-7
    \item[Processo:] Fornitura.
    \item[Formula:]
    \begin{equation}
        \text{Earned Value} - \text{Planned Value}
    \end{equation}
    \item[Descrizione:] Questa metrica rappresenta lo scostamento dai tempi pianificati. Un valore positivo indica che il progetto è in anticipo rispetto a quanto preventivato. Un valore negativo indica che si è svolto meno lavoro rispetto a quello previsto.
\end{description}

\subsection{Cost Performance Index}
\begin{description}
    \item[Codice:] MPC-8
    \item[Processo:] Fornitura.
    \item[Formula:]
    \begin{equation}
        \frac{\text{Earned Value}}{\text{Actual Cost}}
    \end{equation}
    \item[Descrizione:] Questa metrica rappresenta l'indice di efficienza economica del progetto. Un valore inferiore ad 1 indica che il progetto è in sovra-budget ovvero che sta richiedendo un budget maggiore rispetto a quello preventivato. Un valore superiore ad 1 indica che il progetto è in sotto-budget ovvero che sta richiedendo un budget minore rispetto a quello preventivato.
\end{description}

\subsection{Schedule Performance Index}
\begin{description}
    \item[Codice:] MPC-9
    \item[Processo:] Fornitura.
    \item[Formula:]
    \begin{equation}
        \frac{\text{Earned Value}}{\text{Planned Value}}
    \end{equation}
    \item[Descrizione:] Questa metrica rappresenta l'indice di efficienza temporale del progetto. Un valore inferiore ad 1 indica che il progetto sta procedendo più lentamente di quanto preventivato. Un valore superiore ad 1 indica che il progetto sta procedendo più velocemente di quanto preventivato.
\end{description}

\subsection{Estimate to Complete}
\begin{description}
    \item[Codice:] MPC-10
    \item[Processo:] Fornitura.
    \item[Formula:] 
    \begin{equation}
        \frac{\text{Budget at Completion} - \text{Earned Value}}{\text{Cost Performance Index}}
    \end{equation}
    \item[Descrizione:] Questa metrica rappresenta il costo totale ancora da sostenere per il completamento del progetto.
\end{description}

\subsection{Estimate at Completion}
\begin{description}
    \item[Codice:] MPC-11
    \item[Processo:] Fornitura.
    \item[Formula:]
    \begin{equation}
        \text{Actual Cost} + \text{Estimate to Complete}
    \end{equation}
    \item[Descrizione:] Questa metrica rappresenta il costo totale alla fine del progetto in base all'andamento attuale.
\end{description}

\subsection{Budget at Completion}
\begin{description}
    \item[Codice:] MPC-12
    \item[Processo:] Fornitura.
    \item[Formula:] Non è presenta una formula.
    \item[Descrizione:] Questa metrica rappresenta il budget totale del progetto.
\end{description}

\subsection{Code Coverage}
\begin{description}
    \item[Codice:] MPC-13
    \item[Processo:] Sviluppo.
    \item[Formula:] Non è presenta una formula.
    \item[Descrizione:] Questa metrica rappresenta la percentuale di codice attraversato dei test rispetto al totale della code base.
\end{description}

\subsection{Misure di mitigazione insufficienti}
\begin{description}
    \item[Codice:] MPC-14
    \item[Processo:] Risoluzione dei problemi.
    \item[Formula:] Non è presenta una formula.
    \item[Descrizione:] Questa metrica rappresenta il numero totale di misure di mitigazione previste che si sono rivelate insufficienti.
\end{description}

\subsection{Rischi inattesi}
\begin{description}
    \item[Codice:] MPC-15
    \item[Processo:] Risoluzione dei problemi.
    \item[Formula:] Non è presenta una formula.
    \item[Descrizione:] Questa metrica rappresenta il numero totale di rischi inattesi (non analizzati) che si sono verificati.
\end{description}

{
\setlength{\tabcolsep}{10pt}
\renewcommand{\arraystretch}{1.5}
\rowcolors{2}{oddrow}{evenrow}
\begin{xltabular}{\textwidth}{| l | X | c | c |}
    \hline
    \rowcolor{headerrow} \textbf{\textcolor{white}{Codice}} & \textbf{\textcolor{white}{Nome}} & \textbf{\textcolor{white}{Accettabile}} & \textbf{\textcolor{white}{Preferibile}} \\
    \hline
    \endfirsthead
    \hline
    \rowcolor{headerrow} \textbf{\textcolor{white}{Codice}} & \textbf{\textcolor{white}{Nome}} & \textbf{\textcolor{white}{Accettabile}} & \textbf{\textcolor{white}{Preferibile}} \\ 
    \endhead
    MPC-1 & Varianza di Budget & \pm 10\% & \pm 0\% \\
    \hline
    MPC-2 & Varianza dell’impegno orario & \pm 5\% & \pm 0\% \\
    \hline
    MPC-3 & Earned Value & >= MPC-4 &  \\
    \hline
    MPC-4 & Actual Cost & - & - \\
    \hline
    MPC-5 & Planned Value & - & - \\
    \hline
    MPC-6 & Cost Variance & \pm 150 & 0 \\
    \hline
    MPC-7 & Schedule Variance & \pm 150 & 0 \\
    \hline
    MPC-8 & Cost Performance Index & 1 \pm 0.1 & 1 \\
    \hline
    MPC-9 & Schedule Performance Index & 1 \pm 0.1 & 1 \\
    \hline
    MPC-10 & Estimate to Complete & - & -\\
    \hline
    MPC-11 & Estimate at Completion & \pm5\% di MPC-12 & MPC-12 \\
    \hline
    MPC-12 & Budget at Completion & - & - \\
    \hline
    MPC-13 & Code Coverage & 75\% & 100\% \\
    \hline
    MPC-14 & Misure di mitigazione insufficienti & 3 & 0 \\
    \hline
    MPC-15 & Rischi inattesi & 3 & 0 \\
    \hline
    \rowcolor{white} \caption{Metriche di qualità di processo}
    \label{tab:mpc}
\end{xltabular}
}

\section{Qualità di prodotto} \label{sec:qualityprod}
\subsection{Indice di Gulpease}
\begin{description}
    \item[Codice:] MPD-1
    \item[Processo:] Documentazione.
    \item[Formula:] 
    \begin{equation}
    89 +
    \frac{\text{300 (numero Frasi) - 10 (numero Lettere)}}{\text{numero Parole}}
    \label{MPD-1}
    \end{equation}
    \item[Descrizione:] L'indice di Gulpease misura il grado di leggibilità di un testo, stimando la difficoltà di lettura per una persona in base alla sua istruzione. Calcolato su una scala da 1 a 100, un indice superiore a 80 identifica un testo facilmente comprensibile per chiunque con un'istruzione elementare. Un valore pari o superiore a 60 identifica un testo comprensible per le persone con istruzione media. Un punteggio superiore a 40 indica che il testo è comprensibile alle persone con un'istruzione superiore, dunque al target di questo progetto.\\Il valore minimo accettato farà dunque riferimento a quest'ultima soglia, con un valore preferibile fissato a 60 o più.
\end{description}

\subsection{Requisiti obbligatori soddisfatti}
\begin{description}
    \item[Codice:] MPD-2
    \item[Processo:] Sviluppo.
    \item[Formula:]
    \begin{equation}
        100\biggl(\frac{\text{numero Requisiti Obbligatori Soddisfatti}}{\text{numero Requisiti Obbligatori}}\biggr)
    \end{equation}
    \item[Descrizione:] Indica la percentuale dei requisiti obbligatori soddisfatti.\\ In quanto obbligatori, non è definita alcuna soglia di tolleranza tra valore accettabile e preferibile. È necessario giungere al 100\% di completamento.
\end{description}

\subsection{Requisiti desiderabili soddisfatti}
\begin{description}
    \item[Codice:] MPD-3
    \item[Processo:] Sviluppo.
    \item[Formula:]
    \begin{equation}
        100\biggl(\frac{\text{numero Requisiti Desiderabili Soddisfatti}}{\text{numero Requisiti Desiderabili}}\biggr)
    \end{equation}
    \item[Descrizione:] Indica la percentuale dei requisiti desiderabili soddisfatti.\\ In quanto non obbligatori, il valore accettabile è fissato a 0\%, con valore preferibile 100\%.
\end{description}

\subsection{Requisiti opzionali soddisfatti}
\begin{description}
    \item[Codice:] MPD-4
    \item[Processo:] Sviluppo.
    \item[Formula:]
    \begin{equation}
        100\biggl(\frac{\text{numero Requisiti Opzionali Soddisfatti}}{\text{numero Requisiti Opzionali}}\biggr)
    \end{equation}
    \item[Descrizione:] Indica la percentuale dei requisiti opzionali soddisfatti.\\ In quanto non obbligatori, il valore accettabile è fissato a 0\%, con valore preferibile 100\%.
\end{description}


{
\setlength{\tabcolsep}{10pt}
\renewcommand{\arraystretch}{1.5}
\rowcolors{2}{oddrow}{evenrow}
\begin{xltabular}{\textwidth}{| l | X | c | c |}
    \hline
    \rowcolor{headerrow} \textbf{\textcolor{white}{Codice}} & \textbf{\textcolor{white}{Nome}} & \textbf{\textcolor{white}{Accettabile}} & \textbf{\textcolor{white}{Preferibile}} \\
    \hline
    \endfirsthead
    \hline
    \rowcolor{headerrow} \textbf{\textcolor{white}{Codice}} & \textbf{\textcolor{white}{Nome}} & \textbf{\textcolor{white}{Accettabile}} & \textbf{\textcolor{white}{Preferibile}} \\ 
    \endhead
    MPD-1 & Indice di Gulpease & \ge 40 & \ge 60 \\
    \hline
    MPD-2 & Requisiti obbligatori soddisfatti & 100\% & 100\% \\
    \hline
    MPD-3 & Requisiti desiderabili soddisfatti & \ge 0\% & 100\% \\
    \hline
    MPD-4 & Requisiti opzionali soddisfatti & \ge 0\% & 100\% \\
    \hline
    \rowcolor{white} \caption{Metriche di qualità di prodotto}
    \label{tab:mpd}
\end{xltabular}
}

\section{Insieme dei test} \label{sec:test}
In questa sezione vengono definiti i test finalizzati a dimostrare l'adempimento, da parte del prodotto del progetto, ai requisiti individuati nella relativa fase di analisi.\\
Tali test, coerentemente a quanto prefigge il \ccgloss{V model} e il documento Norme\_di\_progetto, sono individuati parallelamente alle attività di sviluppo.\\
Ogni test sarà identificato da un codice alfanumerico, da una descrizione generale, dall'identificazione del requisito da cui tale test deriva e dal tracciamento della implementazione, o meno, di tale nel prodotto.

\subsection{Test di sistema}

\begingroup
\setlength{\tabcolsep}{10pt}
\renewcommand{\arraystretch}{1.5}
\rowcolors{2}{oddrow}{evenrow}
\begin{xltabular}{\textwidth}{| c | X | c | c |}
    \hline
    \rowcolor{headerrow} \textbf{\textcolor{white}{Codice}} & \textbf{\textcolor{white}{Descrizione}} & \textbf{\textcolor{white}{Requisito}} & \textbf{\textcolor{white}{Soddisfatto}}\\
    \hline
    \endhead
    TS-1 & Verificare che l’utente possa visualizzare la lista di tutti i modelli LLM supportati dal sistema. & RFO-1 & \textcolor{xmarkcolor}{\ding{55}} \\
    \hline
    TS-2 & Verificare che l’utente possa selezionare il LLM che il sistema deve utilizzare per le operazioni sui documenti e per la generazione delle risposte. & RFO-2 & \textcolor{xmarkcolor}{\ding{55}} \\
    \hline
    TS-3 & Verificare che l'utente possa visualizzare la lista dei documenti presenti nel sistema & RFO-3 & \textcolor{xmarkcolor}{\ding{55}} \\
    \hline
    TS-4 & Verificare che l’utente possa visualizzare il nome del documento di interesse. & RFO-4 & \textcolor{xmarkcolor}{\ding{55}} \\
    \hline
    TS-5 & Verificare che l’utente possa visualizzare la data di inserimento del documento di interesse. & RFO-5 & \textcolor{xmarkcolor}{\ding{55}} \\
    \hline
    TS-6 & Verificare che l’utente possa visualizzare i tag applicati al documento di interesse. & RFZ-6 & \textcolor{xmarkcolor}{\ding{55}} \\
    \hline
    TS-7 & Verificare che l’utente possa visualizzare il contenuto del documento di interesse. & RFO-7 & \textcolor{xmarkcolor}{\ding{55}} \\
    \hline
    TS-8 & Verificare che l’utente possa visualizzare lo stato del documento di interesse (bloccato o non bloccato). & RFD-8 & \textcolor{xmarkcolor}{\ding{55}} \\
    \hline
    TS-9 &  Verificare che l’utente possa visualizzare la dimensione del documento di interesse. & RFO-9 & \textcolor{xmarkcolor}{\ding{55}} \\
    \hline
    TS-10 & Verificare che l’utente possa ricercare un documento per nome. & RFO-10 & \textcolor{xmarkcolor}{\ding{55}} \\
    \hline
    TS-11 & Verificare che l’utente possa ricercare un documento per data di inserimento nel sistema. & RFO-11 & \textcolor{xmarkcolor}{\ding{55}} \\
    \hline
    TS-12 & Verificare che l’utente possaricercare un documento per i propri tag. & RFZ-12 & \textcolor{xmarkcolor}{\ding{55}} \\
    \hline
    TS-13 & Verificare che l’utente possa aggiungere un tag ad un documento. & RFZ-13 & \textcolor{xmarkcolor}{\ding{55}} \\
    \hline
    TS-14 & Verificare che l’utente possa rimuovere un tag da un documento a cui è associato. & RFZ-14 & \textcolor{xmarkcolor}{\ding{55}} \\
    \hline
    TS-15 & Verificare che l’utente possa creare un nuovo tag da salvare nel sistema. & RFZ-15 & \textcolor{xmarkcolor}{\ding{55}} \\
    \hline
    TS-16 & Verificare che l’utente possaaggiungere un nome al tag durante la sua creazione. & RFZ-16 & \textcolor{xmarkcolor}{\ding{55}} \\
    \hline
    TS-17 & Verificare che l’utente possa aggiungere un colore al tag durante la sua creazione. & RFZ-17 & \textcolor{xmarkcolor}{\ding{55}} \\
    \hline
    TS-18 & Verificare che l’utente possa aggiungere una descrizione al tag durante la sua creazione. & RFZ-18 & \textcolor{xmarkcolor}{\ding{55}} \\
    \hline
    TS-19 &  Verificare che l’utente possa visualizzare la lista di tutti i tag presenti nel sistema e associabili a un documento. & RFZ-19 & \textcolor{xmarkcolor}{\ding{55}} \\
    \hline
    TS-20 & Verificare che l'utente possa visualizzare il nome di ogni tag presente nel sistema. & RFZ-20 & \textcolor{xmarkcolor}{\ding{55}} \\
    \hline
    TS-21 & Verificare che l’utente possa visualizzare il colore di ogni tag presente nel sistema. & RFZ-21 & \textcolor{xmarkcolor}{\ding{55}} \\
    \hline
    TS-22 & Verificare che l’utente possa visualizzare la descrizione di ogni tag presente nel sistema. & RFZ-22 & \textcolor{xmarkcolor}{\ding{55}} \\
    \hline
    TS-23 & Verificare che l’utente possa eliminare uno dei tag presenti nel sistema e con esso tutte le associazioni a ogni documento. & RFZ-23 & \textcolor{xmarkcolor}{\ding{55}} \\
    \hline
    TS-24 & Verificare che l’utente possa eliminare uno dei documenti presenti nel sistema & RFO-24 & \textcolor{xmarkcolor}{\ding{55}} \\
    \hline
    TS-25 & Verificare che l’utente possa confermare l’eliminazione di uno dei documenti presenti nel sistema, eliminando definitivamente  ogni informazione associata a quel documento. & RFO-25 & \textcolor{xmarkcolor}{\ding{55}} \\
    \hline
    TS-26 & Verificare che l’utente possa aggiungere un documento nel sistema tramite trascinamento (drag and drop) & RFD-26 & \textcolor{xmarkcolor}{\ding{55}} \\
    \hline
    TS-27 & Verificare che l’utente possa aggiungere un documento nel sistema tramite navigazione del file system. & RFO-27 & \textcolor{xmarkcolor}{\ding{55}} \\
    \hline
    TS-28 & Verificare che l'utente possa visualizzare un messaggio che lo informa che non è stato possibile inserire il documento a causa del nome del file già in uso. & RFD-28 & \textcolor{xmarkcolor}{\ding{55}} \\
    \hline
    TS-29 & Verificare che l’utente possa visualizzare un messaggio che lo informa che non è stato possibile inserire il documento a causa del formato del file non supportato. & RFD-29 & \textcolor{xmarkcolor}{\ding{55}} \\
    \hline
    TS-30 & Verificare che l’utente possa visualizzare un messaggio che lo informa che non è stato possibile inserire il documento a causa della corruzione del file. & RFD-30 & \textcolor{xmarkcolor}{\ding{55}} \\
    \hline
    TS-31 & Verificare che l'utente possa bloccare un documento, così che il sistema non fornisca risposte su tale documento senza doverlo eliminare. & RFD-31 & \textcolor{xmarkcolor}{\ding{55}} \\
    \hline
    TS-32 & Verificare che l'utente possa sbloccare un documento in precedenza bloccato, così che il sistema possa nuovamente fornire risposte su quel particolare documento. & RFD-32 & \textcolor{xmarkcolor}{\ding{55}} \\
    \hline
    TS-33 & Verificare che l’utente possa visualizzare la lista delle lingue supportate dal chatbot. & RFD-33 & \textcolor{xmarkcolor}{\ding{55}} \\
    \hline
    TS-34 &  Verificare che l'utente possa selezionare la lingua utilizzata dal sistema nel fornire le risposte alle sue domande. & RFD-34 & \textcolor{xmarkcolor}{\ding{55}} \\
    \hline
    TS-35 & Verificare che l'utente possa digitare la domanda da porgere al chatbot tramite tastiera. & RFO-35 & \textcolor{xmarkcolor}{\ding{55}} \\
    \hline
    TS-36 & Verificare che l’utente possa inserire la domanda da porgere al chatbot tramite microfono. & RFD-36 & \textcolor{xmarkcolor}{\ding{55}} \\
    \hline
    TS-37 & Verificare che il sistema, dopo non aver registrato alcun input vocale nel tempo limite a seguito del tentativo da parte dell'utente di inserire una domanda tramite microfono, deve notificare un messaggio all'utente che avvisa la mancata trascrizione della domanda. & RFD-37 & \textcolor{xmarkcolor}{\ding{55}} \\
    \hline
    TS-38 & Verificare che l’utente possa visualizzare la risposta alla domanda che ha inviato in precedenza, qualora l'informazione sia contenuta all'interno di uno dei documenti presenti nel sistema. &  RFO-38 & \textcolor{xmarkcolor}{\ding{55}} \\
    \hline
    TS-39 & Verificare che l’utente possa visualizzare una risposta di cortesia prodotta dal sistema dopo la ricezione di una domanda non pertinente con alcuna informazione presente in tutti i documenti. & RFO-39 & \textcolor{xmarkcolor}{\ding{55}} \\
    \hline
    TS-40 & Verificare che l'utente possa visualizzare un messaggio che lo informa che c'è stato un errore nel ricevere la risposta entro il tempo limite. & RFO-40 & \textcolor{xmarkcolor}{\ding{55}} \\
    \hline
    TS-41 & Verificare che l'utente possa creare una nuova sessione di conversazione col chatbot. & RFD-41 & \textcolor{xmarkcolor}{\ding{55}} \\
    \hline
    TS-42 & Verificare che l'utente possa visualizzare la lista delle sessioni di conversazione col chatbot attive. & RFD-42 & \textcolor{xmarkcolor}{\ding{55}} \\
    \hline
    TS-43 & Verificare che l'utente possa eliminare una delle sessioni di conversazioni attive nel sistema e con essa tutti i messaggi scambiati in quella conversazione. & RFD-43 & \textcolor{xmarkcolor}{\ding{55}} \\
    \hline
    TS-44 & Verificare che l'utente possa confermare l’eliminazione di una sessione di conversazione e solo dopo deve avvenire l'eliminazione effettiva dei dati associati. & RFD-44 & \textcolor{xmarkcolor}{\ding{55}} \\
    \hline
    TS-45 & Verificare che l'utente possa visualizzare lo scambio di domande e risposte avvenuto in precedenza con il sistema in una stessa sessione & RFO-45 & \textcolor{xmarkcolor}{\ding{55}} \\
    \hline
    TS-46 & Verificare che l'utente possa eliminare lo scambio di domande e risposte avvenuto in precedenza con il chatbot in una stessa sessione. & RFO-46 & \textcolor{xmarkcolor}{\ding{55}} \\
    \hline
    TS-47 & Verificare che l'utente possa confermare l'eliminazione dello scambio di domande e risposte, avvenuto in precedenza con il chatbot in una stessa sessione. & RFO-47 & \textcolor{xmarkcolor}{\ding{55}} \\
    \hline
    TS-48 & Verificare che l'utente possa visualizzare il nome del documento da cui deriva la risposta. & RFO-48 & \textcolor{xmarkcolor}{\ding{55}} \\
    \hline
    TS-49 & Verificare che l'utente possa visualizzare il numero della pagina del documento da cui deriva la risposta. & RFD-49 & \textcolor{xmarkcolor}{\ding{55}} \\
    \hline
    TS-50 & Verificare che l'utente possa sentire la lettura della risposta ricevuta. & RFD-50 & \textcolor{xmarkcolor}{\ding{55}} \\
    \hline
    TS-51 & Verificare che il sistema supporti il caricamento di file PDF. & RVO-1 & \textcolor{xmarkcolor}{\ding{55}} \\
    \hline
    TS-52 & Verificare che il sistema supporti il caricamento di file PDF. & RVD-2 & \textcolor{xmarkcolor}{\ding{55}} \\
    \hline
    TS-53 & Verificare che il sistema supporti il caricamento di file con formato .docx, prodotti con Microsoft Word 2007 e versioni successive. & RVD-3 & \textcolor{xmarkcolor}{\ding{55}} \\
    \hline
    TS-54 & Verificare che il sistema supporti il caricamento di file con formato .mp3. & RVZ-4 & \textcolor{xmarkcolor}{\ding{55}} \\
    \hline
    TS-55 & Verificare che il sistema supporti il caricamento di file con formato .mp4. & RVZ-5 & \textcolor{xmarkcolor}{\ding{55}} \\
    \hline
    TS-56 & Verificare che il sistema permetta l'utilizzo di LLM tramite \ccgloss{OpenAI}. & RVO-6 & \textcolor{xmarkcolor}{\ding{55}}\\
    \hline
    TS-57 & Verificare che il sistema permetta il riconoscimento vocale tramite \ccgloss{Whisper} di OpenAI. & RVD-7 & \textcolor{xmarkcolor}{\ding{55}}\\
    \hline
    TS-58 & Verificare che il sistema permetta l'utilizzo di LLM locali tramite \ccgloss{Ollama}. & RVO-8 & \textcolor{xmarkcolor}{\ding{55}} \\
    \hline
    TS-59 & Verificare che il sistema funzioni correttamente con la presenza di 1000 documenti, tra i formati supportati, in esso. & RPO-1 & \textcolor{xmarkcolor}{\ding{55}} \\
    \hline
    TS-60 & Verificare che il sistema processi correttamente ogni documento, tra i formati supportati, con dimensione fino ai 500KB. & RPO-2 & \textcolor{xmarkcolor}{\ding{55}} \\
    \hline
    TS-61 & Verificare che il sistema processi correttamente file audio, tra i formati supportati, di dimensione fino a 5MB. & RPZ-3 & \textcolor{xmarkcolor}{\ding{55}} \\
    \hline
    TS-62 & Verificare che il sistema funzioni correttamente nel browser Google Chrome dalla versione 110 e successive. & RPO-4 & \textcolor{xmarkcolor}{\ding{55}} \\
    \hline
    TS-63 & Verificare che il sistema funzioni correttamente nel browser Mozilla Firefox dalla versione 116 e successive. & RPO-5 & \textcolor{xmarkcolor}{\ding{55}} \\
    \hline
    TS-64 & Verificare che il sistema funzioni correttamente nel browser Opera dalla versione 96 e successive. & RPO-6 & \textcolor{xmarkcolor}{\ding{55}} \\
    \hline
    TS-65 & Verificare che il sistema funzioni correttamente nel browser Microsoft Edge dalla versione 110 e successive. & RPO-7 & \textcolor{xmarkcolor}{\ding{55}} \\
    \hline
    TS-66 & Verificare che il sistema garantisca la creazione di almeno una sessione di conversazione. & RPO-8 & \textcolor{xmarkcolor}{\ding{55}} \\
    \hline
    TS-67 & Verificare che il sistema garantisca la creazione di almeno due sessioni di conversazione. & RPD-9 & \textcolor{xmarkcolor}{\ding{55}} \\
    \hline
    TS-68 & Verificare che il sistema processi i documenti che vengono caricati, creandone i loro embedding. & RIO-1 & \textcolor{xmarkcolor}{\ding{55}} \\
    \hline
    TS-69 & Verificare che il sistema salvi, in modo persistente, i documenti caricati a sistema. & RIO-2 & \textcolor{xmarkcolor}{\ding{55}} \\
    \hline
    TS-70 & Verificare che il sistema salvi, in modo persistente, tutte le informazioni relative ai documenti presenti in esso. & RIO-3 & \textcolor{xmarkcolor}{\ding{55}} \\
    \hline
    TS-71 & Verificare che il sistema salvi, in modo persistente, i vettori dei documenti caricati ed embeddizzati dal sistema. & RIO-4 & \textcolor{xmarkcolor}{\ding{55}} \\
    \hline
    TS-72 & Verificare che il sistema interrompa in modo automatico il processo di generazione della risposta ad una domanda, qualora esso dovesse impiegare un tempo superiore ai 30 secondi. & RIO-5 & \textcolor{xmarkcolor}{\ding{55}} \\
    \hline
    TS-73 & Verificare che il sistema interrompa in modo automatico la trascrizione della domanda via input vocale, qualora non fosse rilevato alcuna voce per 5 secondi. & RID-6 & \textcolor{xmarkcolor}{\ding{55}} \\
    \hline
    TS-74 & Verificare che il sistema salvi, in modo persistente, i messaggi scambiati tra utente e chatbot. & RIO-7 & \textcolor{xmarkcolor}{\ding{55}} \\
    \hline
    TS-75 & Verificare che il sistema effettui correttamente una ricerca semantica (semantic search) tra la domanda posta dall'utente e i gli embedding dei documenti caricati restituendo il documento, o pagina di esso, inerente alla domanda. & RIO-8 & \textcolor{xmarkcolor}{\ding{55}} \\
    \hline
    TS-76 & Verificare che il sistema interroghi il LLM scelto in base alla domanda posta dall'utente e al relativo documento. & RIO-9 & \textcolor{xmarkcolor}{\ding{55}} \\
    \hline
    \rowcolor{white} \caption{Insieme dei test di sistema}
    \label{tab:test}
\end{xltabular}
\endgroup

