\chapter{Requisiti}\label{sec:requisiti}
\section{Introduzione}
In questa sezione del documento vengono riportati tutti i requisiti, relativi al prodotto del progetto, individuati a seguito della fase di analisi.\\
Ogni requisito sarà identificato da un codice alfanumerico, che andrà ad identificare tipologia (funzionale, di qualità, di vincolo, prestazionale), classificazione (obbligatorio, desiderabile, opzionale) e numero del requisito.

\section{Requisiti funzionali}

\begingroup
\setlength{\tabcolsep}{10pt}
\renewcommand{\arraystretch}{1.5}
\rowcolors{2}{oddrow}{evenrow}
\begin{xltabular}{\textwidth}{| c | X | c | c |}
    \hline
    \rowcolor{headerrow} \textbf{\textcolor{white}{Codice}} & \textbf{\textcolor{white}{Descrizione}} & \textbf{\textcolor{white}{Classificazione}} & \textbf{\textcolor{white}{Fonte}}\\
    \hline
    \endhead
    RFO-1 & L’utente deve poter visualizzare la lista dei documenti presenti nel sistema. & Obbligatorio & UC-1 \\
    \hline
    RFO-2 & L’utente deve poter visualizzare un particolare documento presente nel sistema. & Obbligatorio & UC-1.1 \\
    \hline
    RFO-3 & L'utente deve poter visualizzare il nome del documento di interesse. & Obbligatorio & UC-1.1.1 \\
    \hline
    RFO-4 & L’utente deve poter visualizzare la data di inserimento del documento di interesse. & Obbligatorio & UC-1.1.2 \\
    \hline
    RFZ-5 & L’utente deve poter visualizzare i tag applicati al documento di interesse. & Opzionale & UC-1.1.3 \\
    \hline
    RFO-6 & L’utente deve poter visualizzare la preview del documento di interesse. & Obbligatorio & UC-1.1.4 \\
    \hline
    RFO-7 & L’utente deve poter ricercare un documento per nome. & Obbligatorio & UC-2.1 \\
    \hline
    RFO-8 & L’utente deve poter ricercare un documento per data di inserimento. & Obbligatorio & UC-4.2 \\
    \hline
    RFZ-9 & L’utente deve poter ricercare un documento per il proprio tag. & Opzionale & UC-2.3 \\
    \hline
    RFZ-10 & L’utente deve poter aggiungere un tag ad un documento. & Opzionale & UC-3 \\
    \hline
    RFZ-11 & L’utente deve poter rimuovere un tag da un documento. & Opzionale & UC-4 \\
    \hline
    RFZ-12 & L’utente deve poter creare un nuovo tag. & Opzionale & UC-5 \\
    \hline
    RFZ-13 & L’utente deve poter aggiungere un nome nella creazione del tag. & Opzionale & UC-5.1 \\
    \hline
    RFZ-14 & L’utente deve poter aggiungere un colore nella creazione del tag. & Opzionale & UC-5.2 \\
    \hline
    RFZ-15 & L’utente deve poter aggiungere una descrizione nella creazione del tag. & Opzionale & UC-5.3 \\
    \hline
    RFZ-16 & L’utente deve poter visualizzare la lista di tutti i tag creati. & Opzionale & UC-6 \\
    \hline
    RFZ-17 & L’utente deve poter visualizzare uno dei tag creati. & Opzionale & UC-6.1 \\
    \hline
    RFZ-18 & L’utente deve poter visualizzare il nome del tag di interesse. & Opzionale & UC-6.1.1 \\
    \hline
    RFZ-19 & L’utente deve poter visualizzare il colore del tag di interesse. & Opzionale & UC-6.1.2 \\
    \hline
    RFZ-20 & L’utente deve poter visualizzare la descrizione del tag di interesse. & Opzionale & UC-6.1.3 \\
    \hline
    RFZ-21 & L’utente deve poter eliminare uno dei tag presenti nel sistema. & Opzionale & UC-7 \\
    \hline
    RFO-22 & L’utente deve poter eliminare uno dei documenti presenti nel sistema. & Obbligatorio & UC-8 \\
    \hline
    RFO-23 & L’utente deve poter confermare l’eliminazione di uno dei documenti presenti nel sistema. & Obbligatorio & UC-8.1 \\
    \hline
    RFD-24 & L’utente deve poter aggiungere un documento nel sistema tramite trascinamento. & Desiderabile & UC-9.1 \\
    \hline
    RFO-25 & L’utente deve poter aggiungere un documento nel sistema tramite navigazione del file system. & Obbligatorio & UC-9.2 \\
    \hline
    RFD-26 & L'utente deve poter visualizzare un messaggio che lo notifica che c'è stato un errore nell'inserimento del documento dovuto al nome del file già in uso. & Desiderabile & UC-10.1 \\
    \hline
    RFD-27 & L'utente deve poter visualizzare un messaggio che lo notifica che c'è stato un errore nell'inserimento del documento dovuto al formato del file non supportato. & Desiderabile & UC-10.2 \\
    \hline
    RFD-28 & L'utente deve poter visualizzare un messaggio che lo notifica che c'è stato un errore nell'inserimento del documento dovuto alla corruzione del file. & Desiderabile & UC-10.3 \\
    \hline
    RFD-29 & L’utente deve poter visualizzare la lista delle lingue supportate dal chatbot. & Desiderabile & UC-11 \\
    \hline
    RFD-30 & L’utente deve poter selezionare la lingua del sistema. & Desiderabile & UC-12 \\
    \hline
    RFO-31 & L’utente deve poter digitare la domanda da porgere al chatbot. & Obbligatorio & UC-13.1 \\
    \hline
    RFD-32 & L’utente deve poter inserire tramite microfono la domanda da porgere al sistema. & Desiderabile & UC-13.2 \\
    \hline
    RFD-33 & L’utente, dopo aver tentato di inserire senza successo la domanda da porre al sistema tramite input vocale, deve visualizzare un messaggio di errore. & Desiderabile & UC-14 \\
    \hline
    RFO-34 & L’utente deve poter inviare una domanda al sistema. & Obbligatorio & UC-15 \\
    \hline
    RFO-35 & L’utente deve visualizzare un messaggio che lo informa che si è verificato un errore durante l'invio della domanda. & Obbligatorio & UC-16 \\
    \hline
    RFO-36 & L’utente deve poter visualizzare la risposta alla domanda che ha inviato in precedenza. & Obbligatorio & UC-17.1 \\
    \hline
    RFO-37 & L’utente deve poter visualizzare la risposta di cortesia alla domanda senza risposta che ha inviato in precedenza. & Obbligatorio & UC-17.2 \\
    \hline
    RF0-38 & L'utente deve poter visualizzare un messaggio che lo informa che c'è stato un errore nel ricevere la risposta. & Obbligatorio & UC-18 \\
    \hline
    RFD-39 & L’utente deve poter creare una nuova sessione di conversazione col chatbot. & Desiderabile & UC-19 \\
    \hline
    RFD-40 & L'utente deve poter visualizza un messaggio che lo notifica che c'è stato un errore nella creazione di una nuova sessione. & Desiderabile & UC-20 \\
    \hline
    RFD-41 & L’utente deve poter visualizzare la lista delle sessioni di conversazione col chatbot aperte. & Desiderabile & UC-21 \\
    \hline
    RFD-42 & L’utente deve poter visualizzare una delle sessioni di conversazione col chatbot. & Desiderabile & UC-21.1 \\
    \hline
    RFD-43 & L'utente deve poter visualizzare un messaggio che lo notifica che c'è stato un errore nella visualizzazione delle sessioni aperte. & Desiderabile & UC-22 \\
    \hline
    RFD-44 & L’utente deve poter eliminare una delle sessioni di conversazioni attive nel sistema. & Desiderabile & UC-23 \\
    \hline
    RFD-45 & L’utente deve poter confermare l’eliminazione di una sessione di conversazione. & Desiderabile & UC-23.1 \\
    \hline
    RFD-46 & L’utente deve poter visualizzare lo scambio di domande e risposte avvenuto in precedenza con il sistema in una sessione. & Desiderabile & UC-24 \\
    \hline
    RFD-47 & L’utente deve poter eliminare lo scambio di domande e risposte, avvenuto in precedenza con il chatbot, della sessione dal sistema. & Desiderabile & UC-25 \\
    \hline
    RFD-48 & L’utente deve poter visualizzare il nome del documento relativo alla risposta. & Desiderabile & UC-26.1 \\
    \hline
    RFD-49 & L'utente deve poter visualizzare la pagina del documento relativo alla risposta. & Desiderabile & UC-26.2 \\
    \hline
    RFD-50 & L’utente deve poter visualizzare una preview del documento relativo alla risposta. & Desiderabile & UC-26.3 \\
    \hline
    RFD-51 & L’utente deve poter sentire la lettura della risposta ricevuta. & Desiderabile & UC-27 \\
    \hline
    \rowcolor{white} \caption{Requisiti funzionali del prodotto}
    \label{tab:reqfun}
\end{xltabular}
\endgroup

\section{Requisiti di qualità}

\begingroup
\setlength{\tabcolsep}{10pt}
\renewcommand{\arraystretch}{1.5}
\rowcolors{2}{oddrow}{evenrow}
\begin{xltabular}{\textwidth}{| c | X | c | c |}
    \hline
    \rowcolor{headerrow} \textbf{\textcolor{white}{Codice}} & \textbf{\textcolor{white}{Descrizione}} & \textbf{\textcolor{white}{Classificazione}} & \textbf{\textcolor{white}{Fonte}}\\
    \hline
    \endhead
    RQO-1 & Deve essere fornito un documento di analisi dei costi, rischi e tecnologie. & Obbligatorio & Proponente \\
    \hline
    RQO-2 & Deve essere fornito un documento che descrive le attività di bug reporting effettuate. & Obbligatorio & Capitolato \\
    \hline
    RQO-3 & Il progetto deve essere svolto seguendo le regole stabilite dal documento Norme\_di\_progetto. & Obbligatorio & Interna \\
    \hline
    RQO-4 & Deve essere fornito al proponente il codice sorgente in un repository GitHub. & Obbligatorio & Proponente \\
    \hline
    \rowcolor{white} \caption{Requisiti di qualità del prodotto}
    \label{tab:reqqua}
\end{xltabular}
\endgroup

\section{Requisiti di vincolo}

\begingroup
\setlength{\tabcolsep}{10pt}
\renewcommand{\arraystretch}{1.5}
\rowcolors{2}{oddrow}{evenrow}
\begin{xltabular}{\textwidth}{| c | X | c | c |}
    \hline
    \rowcolor{headerrow} \textbf{\textcolor{white}{Codice}} & \textbf{\textcolor{white}{Descrizione}} & \textbf{\textcolor{white}{Classificazione}} & \textbf{\textcolor{white}{Fonte}}\\
    \hline
    \endhead
    RVO-1 & Il sistema deve supportare il caricamento di file PDF. & Obbligatorio & Proponente \\
    \hline
    RVD-2 & Il sistema deve supportare il caricamento di  file Word. & Desiderabile & Proponente \\
    \hline
    RVZ-3 & Il sistema deve supportare il caricamento di  file video. & Opzionale & Proponente \\
    \hline
    RVO-4 & Il sistema deve assicurare la piena sicurezza dei dati contenuti nei documenti. & Obbligatorio & Proponente \\
    \hline
    RVD-5 & La web application deve essere sviluppata in \ccgloss{Angular} o \ccgloss{React}. & Desiderabile & Proponente \\
    \hline
    RVD-6 & Dev'essere utilizzato il \ccgloss{framework} \ccgloss{Langchain} per il collegamento al database vettoriale e al \ccgloss{LLM}. & Desiderabile & Proponente \\
    \hline
    RVD-7 & Dev'essere utilizzato il database vettoriale \ccgloss{ChromaDB} per l'\ccgloss{embedding} dei documenti. & Desiderabile & Proponente \\
    \hline
    RVD-8 & Dev'essere utilizzato \ccgloss{Javascript} e \ccgloss{Nodejs} per lo sviluppo del back-end. & Desiderabile & Proponente \\
    \hline
    RVD-9 & Dev'essere utilizzato il framework \ccgloss{Nextjs} per lo sviluppo del back-end. & Desiderabile & Proponente \\
    \hline
    RVZ-10 & Dev'essere utilizzata la libreria \ccgloss{SQLite} per il salvataggio dei documenti. & Opzionale & Interno \\
    \hline
    RVZ-11 & Dev'essere utilizzato il framework \ccgloss{Tailwind} per lo stile del frontend. & Opzionale & Interno \\
    \hline
    \rowcolor{white} \caption{Requisiti di vincolo del prodotto}
    \label{tab:reqvin}
\end{xltabular}
\endgroup

\section{Requisiti prestazionali}

\begingroup
\setlength{\tabcolsep}{10pt}
\renewcommand{\arraystretch}{1.5}
\rowcolors{2}{oddrow}{evenrow}
\begin{xltabular}{\textwidth}{| c | X | c | c |}
    \hline
    \rowcolor{headerrow} \textbf{\textcolor{white}{Codice}} & \textbf{\textcolor{white}{Descrizione}} & \textbf{\textcolor{white}{Classificazione}} & \textbf{\textcolor{white}{Fonte}}\\
    \hline
    \endhead
    RP-1 & ... & ... & ... \\
    \hline
    ... & ... & ... & ... \\
    \hline
    \rowcolor{white} \caption{Requisiti prestazionali del prodotto}
    \label{tab:reqpre}
\end{xltabular}
\endgroup

\section{Riepilogo}
\begingroup
\setlength{\tabcolsep}{10pt}
\renewcommand{\arraystretch}{1.5}
\rowcolors{2}{oddrow}{evenrow}
\begin{xltabular}{\textwidth}{| X | c | c | c |}
    \hline
    \rowcolor{headerrow} \textbf{\textcolor{white}{Requisito}} & \textbf{\textcolor{white}{Obbligatorio}} & \textbf{\textcolor{white}{Desiderabile}} & \textbf{\textcolor{white}{Opzionale}}\\
    \hline
    \endhead
    Funzionale & 0 & 0 & 0 \\
    \hline
    Di qualità & 0 & 0 & 0 \\
    \hline
    Di vincolo & 0 & 0 & 0 \\
    \hline
    Prestazionali & 0 & 0 & 0 \\
    \hline
    \cellcolor{headerrow} \textbf{\textcolor{white}{Totale}} & 0 & 0 & 0 \\
    \hline
    
    \rowcolor{white} \caption{Riepilogo dei requisiti}
    \label{tab:riepilogo}
\end{xltabular}
\endgroup
