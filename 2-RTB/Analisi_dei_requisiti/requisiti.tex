\chapter{Requisiti}\label{sec:requisiti}
\section{Introduzione}
In questa sezione del documento vengono riportati tutti i requisiti, relativi al prodotto del progetto, individuati a seguito della fase di analisi.\\
Ogni requisito sarà identificato da un codice alfanumerico, che andrà ad identificare tipologia (funzionale, di qualità, di vincolo, prestazionale), classificazione (obbligatorio, desiderabile, opzionale) e numero del requisito.

\section{Requisiti funzionali}

\begingroup
\setlength{\tabcolsep}{10pt}
\renewcommand{\arraystretch}{1.5}
\rowcolors{2}{oddrow}{evenrow}
\begin{xltabular}{\textwidth}{| c | X | c | c |}
    \hline
    \rowcolor{headerrow} \textbf{\textcolor{white}{Codice}} & \textbf{\textcolor{white}{Descrizione}} & \textbf{\textcolor{white}{Classificazione}} & \textbf{\textcolor{white}{Fonte}}\\
    \hline
    \endhead
    RFO-1 & L’utente deve poter accedere alla parte del sistema relativa alla gestione della documentazione. & Obbligatorio & UC-1 \\
    \hline
    RFO-2 & L’utente deve poter accedere alla parte del sistema relativa al chatbot. & Obbligatorio & UC-2 \\
    \hline
    RFO-3 & L’utente deve poter visualizzare la lista dei documenti presenti nel sistema. & Obbligatorio & UC-3 \\
    \hline
    RFO-3.1 & L’utente deve poter visualizzare un particolare documento presente nel sistema. & Obbligatorio & UC-3.1 \\
    \hline
    RFO-3.1.1 & L'utente deve poter visualizzare il nome del documento di interesse. & Obbligatorio & UC-3.1.1 \\
    \hline
    RFO-3.1.2 & L’utente deve poter visualizzare la data di inserimento del documento di interesse. & Obbligatorio & UC-3.1.2 \\
    \hline
    RFZ-3.1.3 & L’utente deve poter visualizzare i tag applicati al documento di interesse. & Opzionale & UC-3.1.3 \\
    \hline
    RFO-3.1.4 & L’utente deve poter visualizzare la preview del documento di interesse. & Obbligatorio & UC-3.1.4 \\
    \hline
    RFO-4.1 & L’utente deve poter ricercare un documento per nome. & Obbligatorio & UC-4.1 \\
    \hline
    RFO-4.2 & L’utente deve poter ricercare un documento per data di inserimento. & Obbligatorio & UC-4.2 \\
    \hline
    RFZ-4.3 & L’utente deve poter ricercare un documento per il proprio tag. & Opzionale & UC-4.3 \\
    \hline
    RFZ-5 & L’utente deve poter aggiungere un tag ad un documento. & Opzionale & UC-5 \\
    \hline
    RFZ-6 & L’utente deve poter rimuovere un tag da un documento. & Opzionale & UC-6 \\
    \hline
    RFZ-7 & L’utente deve poter creare un nuovo tag. & Opzionale & UC-7 \\
    \hline
    RFZ-7.1 & L’utente deve poter aggiungere un nome nella creazione del tag. & Opzionale & UC-7.1 \\
    \hline
    RFZ-7.2 & L’utente deve poter aggiungere un colore nella creazione del tag. & Opzionale & UC-7.2 \\
    \hline
    RFZ-7.3 & L’utente deve poter aggiungere una descrizione nella creazione del tag. & Opzionale & UC-7.3 \\
    \hline
    RFZ-8 & L’utente deve poter visualizzare la lista di tutti i tag creati. & Opzionale & UC-8 \\
    \hline
    RFZ-8.1 & L’utente deve poter visualizzare uno dei tag creati. & Opzionale & UC-8.1 \\
    \hline
    RFZ-8.1.1 & L’utente deve poter visualizzare il nome del tag di interesse. & Opzionale & UC-8.1.1 \\
    \hline
    RFZ-8.1.2 & L’utente deve poter visualizzare il colore del tag di interesse. & Opzionale & UC-8.1.2 \\
    \hline
    RFZ-8.1.3 & L’utente deve poter visualizzare la descrizione del tag di interesse. & Opzionale & UC-8.1.3 \\
    \hline
    RFZ-9 & L’utente deve poter eliminare uno dei tag presenti nel sistema. & Opzionale & UC-9 \\
    \hline
    RFO-10 & L’utente deve poter eliminare uno dei documenti presenti nel sistema. & Obbligatorio & UC-10 \\
    \hline
    RFO-10.1 & L’utente deve poter confermare l’eliminazione di uno dei documenti presenti nel sistema. & Obbligatorio & UC-10.1 \\
    \hline
    RFO-11 & L'utente deve poter visualizzare un messaggio che lo notifica che c'è stato un errore nell'eliminazione di un documento. & Obbligatorio & UC-11 \\
    \hline
    RFD-12.1 & L’utente deve poter aggiungere un documento nel sistema tramite trascinamento. & Desiderabile & UC-12.1 \\
    \hline
    RFO-12.2 & L’utente deve poter aggiungere un documento nel sistema tramite navigazione del file system. & Obbligatorio & UC-12.2 \\
    \hline
    RFD-13 & L'utente deve poter visualizzare un messaggio che lo notifica che c'è stato un errore nell'inserimento del documento. & Desiderabile & UC-13 \\
    \hline
    RFD-13.1 & L'utente deve poter visualizzare un messaggio che lo notifica che c'è stato un errore nell'inserimento del documento dovuto al nome del file già in uso. & Desiderabile & UC-13.1 \\
    \hline
    RFD-13.2 & L'utente deve poter visualizzare un messaggio che lo notifica che c'è stato un errore nell'inserimento del documento dovuto al formato del file non supportato.. & Desiderabile & UC-13.2 \\
    \hline
    RFD-13.3 & L'utente deve poter visualizzare un messaggio che lo notifica che c'è stato un errore nell'inserimento del documento dovuto alla corruzione del file. & Desiderabile & UC-13.3 \\
    \hline
    RFD-14 & L’utente deve poter visualizzare la lista delle lingue supportate dal chatbot. & Desiderabile & UC-14 \\
    \hline
    RFD-15 & L’utente deve poter selezionare la lingua del sistema. & Desiderabile & UC-15 \\
    \hline
    RFO-16 & L’utente deve poter inserire una domanda per il chatbot. & Obbligatorio & UC-16 \\
    \hline
    RFO-16.1 & L’utente deve poter digitare la domanda da porgere al chatbot. & Obbligatorio & UC-16.1 \\
    \hline
    RFD-16.2 & L’utente deve poter inserire tramite microfono la domanda da porgere al sistema. & Desiderabile & UC-16.2 \\
    \hline
    RFD-17 & L’utente dopo aver tentato di inserire senza successo la domanda da porre al sistema tramite input vocale deve visualizzare un messaggio di errore. & Desiderabile & UC-17 \\
    \hline
    RFO-18 & L’utente deve poter inviare una domanda al sistema. & Obbligatorio & UC-18 \\
    \hline
    RFO-19 & L’utente deve visualizzare un messaggio che lo informa che si è verificato un errore durante l'invio della domanda. & Obbligatorio & UC-19 \\
    \hline
    RFO-20 & L’utente deve poter visualizzare la risposta alla domanda che ha inviato in precedenza. & Obbligatorio & UC-20 \\
    \hline
    RF0-21 & L'utente deve poter visualizzare un messaggio che lo informa che c'è stato un errore nel ricevere la risposta. & Obbligatorio & UC-21 \\
    \hline
    RFD-22 & L’utente deve poter creare una nuova sessione di conversazione col chatbot. & Desiderabile & UC-22 \\
    \hline
    RFD-23 & L'utente deve poter visualizza un messaggio che lo notifica che c'è stato un errore nella creazione di una nuova sessione. & Desiderabile & UC-23 \\
    \hline
    RFD-24 & L’utente deve poter visualizzare la lista delle sessioni di conversazione col chatbot aperte. & Desiderabile & UC-21 \\
    \hline
    RFD-24.1 & L’utente deve poter visualizzare una delle sessioni di conversazione col chatbot. & Desiderabile & UC-21.1 \\
    \hline
    RFD-25 & L'utente deve poter visualizzare un messaggio che lo notifica che c'è stato un errore nella visualizzazione delle sessioni aperte. & Desiderabile & UC-25 \\
    \hline
    RFD-26 & L’utente deve poter eliminare una delle sessioni di conversazioni attive nel sistema. & Desiderabile & UC-26 \\
    \hline
    RFD-27 & L'utente deve poter visualizzare un messaggio che lo notifica che c'è stato un errore nell'eliminazione di una sessione. & Desiderabile & UC-27 \\
    \hline
    RFD-28 & L’utente deve poter visualizzare lo scambio di domande e risposte avvenuto in precedenza con il sistema in una sessione. & Desiderabile & UC-28 \\
    \hline
    RFD-29 & L'utente deve poter visualizzare un messaggio che lo notifica che c'è stato un errore nella visualizzazione della chat history. & Desiderabile & UC-29 \\
    \hline
    RFD-30 & L’utente deve poter eliminare lo scambio di domande e risposte, avvenuto in precedenza con il chatbot, dalla sessione del sistema. & Desiderabile & UC-30 \\
    \hline
    RFD-31 & L'utente deve poter visualizzare un messaggio che lo notifica che c'è stato un errore nell'eliminazione della chat history. & Desiderabile & UC-31 \\
    \hline
    RFD-32 & L’utente deve poter visualizzare le informazioni del documento da cui deriva la risposta ricevuta. & Desiderabile & UC-32 \\
    \hline
    RFD-32.1 & L’utente deve poter visualizzare il nome del documento relativo alla risposta. & Desiderabile & UC-32.1 \\
    \hline
    RFD-32.2 & L'utente deve poter visualizzare la pagina del documento relativo alla risposta. & Desiderabile & UC-32.2 \\
    \hline
    RFD-32.3 & L’utente deve poter visualizzare una preview del documento relativo alla risposta. & Desiderabile & UC-32.3 \\
    \hline
    RFD-33 & L’utente deve poter sentire la lettura della risposta ricevuta. & Desiderabile & UC-33 \\
    \hline
    \rowcolor{white} \caption{Requisiti funzionali del prodotto}
    \label{tab:reqfun}
\end{xltabular}
\endgroup

\section{Requisiti di qualità}

\begingroup
\setlength{\tabcolsep}{10pt}
\renewcommand{\arraystretch}{1.5}
\rowcolors{2}{oddrow}{evenrow}
\begin{xltabular}{\textwidth}{| c | X | c | c |}
    \hline
    \rowcolor{headerrow} \textbf{\textcolor{white}{Codice}} & \textbf{\textcolor{white}{Descrizione}} & \textbf{\textcolor{white}{Classificazione}} & \textbf{\textcolor{white}{Fonte}}\\
    \hline
    \endhead
    RQO-1 & Deve essere fornito un documento di analisi dei costi, rischi e tecnologie. & Obbligatorio & Proponente \\
    \hline
    RQO-2 & Deve essere fornito un documento che descrive le attività di bug reporting effettuate. & Obbligatorio & Capitolato \\
    \hline
    ... & ... & ... & ... \\
    \hline
    \rowcolor{white} \caption{Requisiti di qualità del prodotto}
    \label{tab:reqqua}
\end{xltabular}
\endgroup

\section{Requisiti di vincolo}

\begingroup
\setlength{\tabcolsep}{10pt}
\renewcommand{\arraystretch}{1.5}
\rowcolors{2}{oddrow}{evenrow}
\begin{xltabular}{\textwidth}{| c | X | c | c |}
    \hline
    \rowcolor{headerrow} \textbf{\textcolor{white}{Codice}} & \textbf{\textcolor{white}{Descrizione}} & \textbf{\textcolor{white}{Classificazione}} & \textbf{\textcolor{white}{Fonte}}\\
    \hline
    \endhead
    RVO-1 & Il sistema deve supportare i file PDF. & Obbligatorio & Proponente \\
    \hline
    RVD-2 & Il sistema deve supportare i file Word. & Desiderabile & Proponente \\
    \hline
    RVZ-3 & Il sistema deve supportare i file video. & Opzionale & Proponente \\
    \hline
    RVO-4 & Il sistema non deve rispondere a domande non pertinenti al contesto di utilizzo. & Obbligatorio & Proponente \\
    \hline
    RVO-5 & Il sistema deve rispondere con una risposta di cortesia a domande la cui risposta non è contenuta in nessun documento. & Obbligatorio & Proponente \\
    \hline
    RVO-6 & Il sistema deve assicurare la piena sicurezza dei dati contenuti nei documenti. & Obbligatorio & Proponente \\
    \hline
    \rowcolor{white} \caption{Requisiti di vincolo del prodotto}
    \label{tab:reqvin}
\end{xltabular}
\endgroup

\section{Requisiti prestazionali}
Durante l'analisi condotta dal nostro gruppo e nel confronto con il proponente, non sono emersi requisiti specifici riguardanti le prestazioni dell'applicazione. Questo è dovuto alla natura aleatoria delle prestazioni di un sistema in rete, influenzato da numerosi fattori non controllabili.\\Per quanto concerne la scelta del modello, è importante considerare che se l'utente opta per il modello OpenAI, richiedendo un modello online, ciò comporterà uno sforzo minimo da parte del gestore del servizio dell'applicazione. Al contrario, nel caso di modelli locali, la selezione e la scalabilità dipenderanno dalle risorse fisiche disponibili sul server gestore dell'applicazione.

\section{Riepilogo}
\begingroup
\setlength{\tabcolsep}{10pt}
\renewcommand{\arraystretch}{1.5}
\rowcolors{2}{oddrow}{evenrow}
\begin{xltabular}{\textwidth}{| X | c | c | c |}
    \hline
    \rowcolor{headerrow} \textbf{\textcolor{white}{Requisito}} & \textbf{\textcolor{white}{Obbligatorio}} & \textbf{\textcolor{white}{Desiderabile}} & \textbf{\textcolor{white}{Opzionale}}\\
    \hline
    \endhead
    Funzionale & 0 & 0 & 0 \\
    \hline
    Di qualità & 0 & 0 & 0 \\
    \hline
    Di vincolo & 0 & 0 & 0 \\
    \hline
    Prestazionali & 0 & 0 & 0 \\
    \hline
    \cellcolor{headerrow} \textbf{\textcolor{white}{Totale}} & 0 & 0 & 0 \\
    \hline
    
    \rowcolor{white} \caption{Riepilogo dei requisiti}
    \label{tab:riepilogo}
\end{xltabular}
\endgroup
