\chapter{L}

\section{LangChain}
LangChain è un \emph{framework\ped{G}} per lo sviluppo di applicazioni basate su modelli linguistici che permette di collegare il modello ad altre fonti di dati e consentire ad esso di interagire con il suo ambiente.

\section{\LaTeX}
LaTeX è un linguaggio di marcatura per la preparazione di testi, basato sul programma di composizione tipografica \TeX.

\section{LLM}\label{sec:Large Language Model}
I Large Language Model (LLM) sono modelli di apprendimento automatico basati su reti neurali che sono addestrati a partire da grandi quantità di testo. Questi modelli sono in grado di generare testo in linguaggio naturale coerente e comprensibile, in modo simile a quello scritto da un essere umano.

\section{Log}
Il risultato di una registrazione sequenziale e cronologica delle operazioni effettuate da un sistema informatico, sia esso un server, un client, un'applicazione o un programma.
