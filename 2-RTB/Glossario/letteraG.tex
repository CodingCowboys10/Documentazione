\chapter{G}

\section{Gantt}
Con riferimento ai diagrammi di Gantt, diagramma che mostra graficamente la dislocazione temporale di attività, evidenziandone per ognuna inzio e fine, stimati ed effettivi, e possibilità di sequenzialità e parallelismo con altre attività.

\section{Git}
Git è un software di controllo di versione (VCS),è un software gratuito e open source, e le sue eccellenti prestazioni lo rendono il VCS più diffuso e utilizzato.
Git è un sistema distribuito utilizzabile da interfaccia a riga di comando, con funzionalità come \emph{branching\ped{G}} e \emph{merging\ped{G}} e flussi di lavoro multipli.

\section{GitHub}
GitHub è uno strumento web di version control \emph{Git\ped{G}}. Usando questo strumento i programmatori possono lavorare in modo coordinato sulla stessa base di codice, pur sviluppando in modo indipendente.
GitHub offre funzionalità di hosting e revisione del codice, commenti e feedback, collaborazione e gestione del team. I programmatori vengono aggiornati in tempo reale sull'evoluzione del progetto. Inoltre è possibile ripercorrere l'intera storia del codice e ripristinarne una versione precedente, grazie al salvataggio di ogni modifica e \emph{branch\ped{G}} effettuati.

\section{GitHub Action}\label{sec:GitHub Actions}
GitHub Actions è una piattaforma di integrazione continua e distribuzione continua che consente di \emph{automatizzare\ped{G}} il flusso di lavoro, di compilazione, test e distribuzione. Si possono creare flussi di lavoro che compilano e testano ogni \emph{pull request\ped{G}} al \emph{repository\ped{G}} o distribuiscono le \emph{pull request\ped{G}} unite in produzione.

