\chapter{P}

\section{PB}\label{sec:Product Baseline}
Acronimo per Product Baseline. Seconda revisione di avanzamento obbligatoria nell'ambito del \emph{progetto\ped{G}} didattico in corso, vaglia la maturità della \emph{baseline\ped{G}} architetturale del prodotto software e la sua realizzazione.

\section{PDCA}\label{sec:Plan Do Check Act}
Acronimo per Plan Do Check Act. In riferimento al ciclo PDCA, o ciclo di Deming, è un metodo di miglioramento dei processi di ciclo di vita del software composto da quattro fasi, nella quale si pianificano, attuano, valutano e stabiliscono azioni migliorative per i processi.

\section{Preventivo}\label{sec:Preventivi}
Documentazioni e calcoli finalizzati ad anticipare i costi di realizzazione di una determinata attività o lavoro.       

\section{Progetto}\label{sec:Progetti}
In ingegneria del software, un progetto è un insieme pianificato di attività finalizzato allo sviluppo di un prodotto software. Include la definizione di obiettivi, la pianificazione delle attività, lo sviluppo, la gestione delle risorse, il controllo di qualità, la consegna e la manutenzione del software. La gestione efficace di queste fasi è essenziale per garantire il successo del progetto.

\section{Progettista}\label{sec:Progettisti}
Chi, grazie alle competenze tecniche e tecnologiche avanzate che possiede, determina le scelte realizzative in un \emph{progetto\ped{G}}; segue lo sviluppo e non la manutenzione.

\section{Programmatore}\label{sec:Programmatori}
Contribuisce alla realizzazione e manutenzione del prodotto software, ha competenze tecniche ma deleghe limitate.

\section{Project Management}
Il Project Management è una attività all’interno di un’azienda e comprende le fasi di pianificazione e organizzazione del lavoro, incluse le decisioni relative a quanto tempo, denaro e persone saranno necessarie per raggiungere gli obiettivi e ottenere il massimo beneficio. 

\section{Proponente}\label{sec:Proponenti}
Il proponente è chi propone, cioè presenta alla deliberazione di un'assemblea, di un comitato, di un consiglio, ecc. un provvedimento, o fa comunque una proposta. Nel caso specifico l'azienda \emph{AzzurroDigitale\ped{G}} svolgerà il ruolo di proponente del \emph{capitolato\ped{G}}.

\section{PoC}\label{sec:Proof of Concept}
Acronimo per Proof of Concept. Artefatto usa e getta sul piano architetturale, realizzato a inizio progetto per valutare la fattibilità tecnologica del prodotto atteso.

\section{Pull request}
Una Pull Request è una proposta di modifica avanzata da un collaboratore a un \emph{repository\ped{G}} di software. Consiste nell'inviare al creatore originale (o al team di sviluppo) le modifiche apportate al codice sorgente, chiedendo che vengano esaminate e, se ritenute valide, integrate nel progetto principale.

\section{Push}
Il comando git push in Git serve a fare l’upload delle modifiche locali su un \emph{repository\ped{G}} remoto. In questo modo, i \emph{commit\ped{G}} locali sono resi disponibili agli altri collaboratori del \emph{progetto\ped{G}}, che potranno recuperarli tramite un fetch e incorporarli nei rispettivi \emph{repository\ped{G}} locali.
