\section{Resoconto} \label{sec:resoconto}
\subsection{Discussioni} \label{subsec:resdiscussione}
\begin{enumerate}
    \item La riunione è iniziata dall'esito del colloquio per l'RTB, svolto con il professor Cardin la settimana precedente. In particolare, sono state lette attentamente le indicazioni riguardanti l'Analisi\_dei\_\ccgloss{requisiti}\_v1.0. Tale documento dovrà essere modificato e corretto, seguendo le indicazioni fornite dal \ccgloss{committente}, prima di richiedere l'ultimo colloquio di revisione con il professore Vardanega. Soddisfacente invece la valutazione della presentazione, ritenuta ampiamente positiva.
    
    \item Si è successivamente discusso sull'andamento del precedente sprint, che ci ha visto avanzare verso l'inizio della \ccgloss{PB} con il superamento della prima revisione di avanzamento, con una retrospettiva. A seguito di tale, sono stati fissati i \ccgloss{ruoli} e gli obiettivi per il nuovo sprint.

    \item La riunione si è conclusa con la discussione sullo stato dei restanti documenti da presentare per l'ultimo colloquio di revisione. In particolare, oltre alle modifiche già citate dell'Analisi\_dei\_requisiti\_v1.0, che la porteranno alla versione successiva, sono previsti degli aggiornamenti dei test successivi alle modifiche dei requisiti. Il Glossario, così come le Norme\_di\_progetto, è pronto per l'approvazione. Il Piano\_di\_qualifica necessita delle ultime modifiche a seguito della fine del sesto sprint, e conseguente inizio del settimo sprint. La Lettera\_di\_presentazione deve invece essere completata.\\
    L'obiettivo del gruppo è inviare tutto il materiale necessario a sostenere il colloquio con il professor Vardanega entro il weekend.
    
\end{enumerate}

\subsection{Prossima riunione} \label{subsec:riunione}
Viene fissata una riunione per la settimana successiva.