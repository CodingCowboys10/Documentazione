\chapter{Processi organizzativi}

\section{Gestione dei processi} \label{sec:gestionepro}
\subsection{Definizione}
Il processo organizzativo di gestione dei processi contiene le attività e i compiti generici che ogni membro è chiamato a svolgere nella realizzazione e gestione dei processi relativi.
\subsection{Attività}
Il processo di gestione dei processi si articola nelle seguenti attività:
\begin{enumerate}
    \item istanziazione del processo;
    \item pianificazione;
    \item esecuzione e controllo;
    \item revisione e valutazione.
    %\item chiusura.
\end{enumerate}
\subsubsection{Istanziazione del processo}
L'attività si articola nel compito di stabilire l'organizzazione delle risorse necessarie alla realizzazione dei processi attivi nel progetto. Vengono definiti dei ruoli con responsabilità delimitate che i membri del gruppo sono chiamati a ricoprire:
\begin{itemize}
    \item \ccgloss{Responsabile}: figura preposta a controllare e coordinare l'operato del gruppo, curando le relazioni verso l'esterno del progetto.
    \item \ccgloss{Amministratore}: definisce e supervisiona il dominio informatico e tecnologico necessario alle normali mansioni di ogni membro del gruppo, fornendo le risorse utili al way of working collettivo.
    \item Analista: ruolo fondamentale per l'analisi dei requisiti del prodotto e per la corretta comprensione dei problemi relativi al progetto.
    \item Progettista: figura chiamata a progettare l'architettura del software, identificando i componenti principali e specificando standard tecnici, strumenti e piattaforme da utilizzare.
    \item Programmatore: colui che realizza le scelte implementative dei progettisti.
    \item Verificatore: membro preposto alla verifica della qualità di quanto prodotto dagli altri membri del gruppo, deve garantire la conformità e la correttezza del prodotto, nonché della documentazione ad esso associata.
\end{itemize}

\subsubsection{Pianificazione}
L'attività si articola nei seguenti compiti:
\begin{enumerate}
    \item Deve essere redatto il piano di gestione di progetto, già citato nel processo primario di fornitura, sotto il nome di Piano di progetto. Il documento dovrà riportare:
\begin{itemize}
    \item Analisi dei rischi, dove devono essere analizzate e documentate tutte le problematiche riscontrabili durante la realizzazione del progetto. Tale sezione deve seguire correttamente le direttive contenute nel processo di supporto di risoluzione dei problemi.
    \item Pianificazione, con la descrizione del \ccgloss{modello} adottato dal gruppo e l'organizzazione, in \ccgloss{sprint} di due settimane, delle milestone di ogni segmento temporale con relativi obiettivi da raggiungere.
    \item Preventivo, dove devono essere riportate, sotto forma di preventivi orari ed economici, le risorse stanziate per il completamento degli obiettivi. In questa sezione deve essere ben chiara la distribuzione oraria dei ruoli che ogni membro ricoprirà in uno sprint, tenendo traccia dell'evoluzione preventiva del budget di progetto.
    \item Consuntivo, dove devono essere riportate, sotto forma di consuntivi orari ed economici, le risorse effettive impiegate nei singoli sprint. In questa sezione devono essere ben chiari i ruoli, con i relativi costi, ricoperti da ogni membro. Deve essere documentata una fase di analisi retrospettiva dello sprint, dove analizzare il grado di avanzamento effettivo del progetto, le cui considerazioni ricadranno nella rimodulazione della pianificazione e dei preventivi qualora necessario.
\end{itemize}
    \item Le attività e i compiti che ogni membro deve eseguire in uno sprint devono essere puntualmente tradotti in ticket e opportunamente assegnati tramite gli strumenti a sostegno della pianificazione, già citati nel processo primario di fornitura. L'intero processo dovrà essere documentato tramite una tabella di tracciamento delle decisioni e azioni intraprese.
\end{enumerate}


\subsubsection{Esecuzione e controllo}
L'attività si articola nel compito di analizzare ogni singolo processo e prodotto tramite il cruscotto valutativo, o dashboard, contenente i dati misurati tramite le metriche proprie di ogni processo attivo. Per dare significato ai valori di ogni metrica, per ognuna di esse devono essere identificati delle soglie di sufficienza, oltre al valore ottimale al quale il prodotto di ogni processo deve mirare.

\subsubsection{Revisione e valutazione}
L'attività si articola nel compito di analizzare e revisionare i prodotti delle attività di ogni processo attivo nel progetto, valutando la bontà e la conformità dell'esecuzione del processo da parte di tutto il gruppo. Ad accompagnare ogni metrica nel cruscotto del gruppo, deve essere prodotta un'analisi dei valori visibili, pianificando azioni migliorative per i processi con valori non soddisfacenti. Il miglioramento del processo deve seguire quanto riportato nell'omonimo processo organizzativo.

%\subsubsection{Chiusura}
%\subsection{Strumenti}
%\subsection{Metriche}
\newpage

\section{Gestione delle infrastrutture tecniche} \label{sec:gestioinfr}
\subsection{Definizione}
%Organizzazione degli strumenti di supporto ai processi%  
Il processo organizzativo di gestione delle infrastrutture tecniche comprende l'organizzazione degli strumenti di supporto ai processi, assicurando che siano in grado di sostenere le attività tecniche e gli obiettivi del progetto.
\subsection{Attività}
Il processo di gestione delle infrastrutture tecniche si articola nelle seguenti attività:
\begin{enumerate}
    \item istanziazione del processo;
    \item creazione delle infrastrutture;
    \item manutenzione delle infrastrutture.
\end{enumerate}
\subsubsection{Istanziazione del processo}
L'attività si articola nel compito di definire gli obiettivi e i requisiti delle infrastrutture tecniche. Vengono istazniati i protocolli di gestione e si stabiliscono le procedure per il monitoraggio e il controllo delle infrastrutture.

\subsubsection{Creazione delle infrastrutture}
L'attività sia articola nel compito di progettare e l'implementare le infrastrutture tecniche definite precedentemente. Vengono configurati e si mettono in funzione i componenti hardware e software necessari al supporto delle operazioni.

\subsubsection{Manutenzione delle infrastrutture}
%La manutenzione delle infrastrutture è un'attività continua e costante. Coinvolge la sorveglianza e la manutenzione preventiva e correttiva delle infrastrutture tecniche. Si assicura che siano operative, sicure e ottimizzate per il carico di lavoro previsto. %
L'attività si articola nei seguenti compiti:
            \begin{enumerate}
                
            \item Le attività sono suddivise durante le riunioni in task specifiche, composte da sottotask;
            \item I ticket relativi a tali task vengono creati su \ccgloss{Jira}, dettagliando gli obiettivi come elementi da completare (\textit{to do});
            \item Ciascun membro del gruppo si assegna autonomamente una sottotask, indicando lo stato di avanzamento come "in corso" (\textit{in progress});
            \item Al termine, il membro segnala ai verificatori la necessità di esaminare il lavoro completato, trasferendo sottotask allo stato di "da verificare" (\textit{to revise});
            \item Un verificatore designato esamina il lavoro eseguito da un altro membro e sposta la sottotask allo stato di "in fase di verifica" (\textit{in checking});
            \item Dopo la revisione, se l'esito è negativo, la task o sottotask torna allo stato "da completare" (\textit{to do}), e il processo riprende dal punto 3. In caso la revisione di una sottotask abbia esito positivo, lo stato di avanzamento del ticket diventa "completato" (\textit{done}). Quando tutte le sottotask che compongono una determinata task si trovano allo stato (\textit{done}), il lavoro viene sottoposto all'approvazione del responsabile e la task passa allo stato di "da approvare" (\textit{to approve});
            \item Una volta che il responsabile approva il contenuto della task, questa viene spostata allo stato di "completata" (\textit{done}), concludendo il ciclo di vita del ticket.
            \end{enumerate}
In conclusione, questo processo strutturato, dall'assegnazione delle attività alla loro revisione e approvazione, offre un efficace schema di gestione, assicurando il completamento e la validazione delle task in un ciclo di vita chiaro e definito.

\subsection{Strumenti}
Durante lo svolgimento dei compiti vengono utilizzati i seguenti strumenti:
\begin{itemize}
    \item Jira, per l'assegnazione delle issues a ciascun membro, facilitando la frammentazione agile dei compiti. Inoltre il responsabile crea una milestone con una scadenza di due settimane dalla sua creazione, durante la quale vengono delineate e assegnate le specifiche attività ai membri del team.
\end{itemize}
%Nel supporto del processo di gestione delle infrastrutture tecniche, si è adottato l'uso dell'\ccgloss{ITS} \ccgloss{Jira}. Questo sistema integra strumenti di comunicazione che notificano automaticamente ciascun membro in merito alle issue assegnate, garantendo una piena consapevolezza all'interno del team. L'utilizzo di \ccgloss{Jira} all'interno del gruppo facilita la frammentazione agile dei compiti. Il responsabile crea una milestone con una scadenza di due settimane dalla sua creazione, durante la quale vengono delineate e assegnate le specifiche attività ai membri del team.%

\newpage

\section{Miglioramento del processo} \label{sec:migliproc}
\subsection{Definizione}
Il processo organizzativo di miglioramento dei processi individua le attività utili a stabilire, controllare, valutare e migliorare i processi di ciclo di vita del software.
\subsection{Attività}
Il processo di miglioramento del processo si articola nelle seguenti attività:
\begin{enumerate}
    \item istanziazione del processo;
    \item valutazione del processo;
    \item miglioramento del processo.
\end{enumerate}
\subsubsection{Istanziazione del processo}
L'attività si articola nel compito di stabilire dei processi organizzativi relativi all'intera durata del ciclo di vita del software.
\subsubsection{Valutazione del processo}
L'attività si articola nei seguenti compiti:
\begin{enumerate}
    \item Deve essere sviluppata una procedura di valutazione dei processi. La valutazione del singolo processo si baserà sulla qualità e funzionalità dei suoi prodotti, in relazione alle attese e alle necessità che il gruppo attribuisce al processo stesso.
    \item Devono essere svolte delle opportune valutazioni dei singoli processi, regolari nel tempo, così da individuare quanto prima la necessità di migliorare uno di essi. Valutazioni sul singolo processo saranno svolte e documentate nelle riunioni del team, dove potranno così emergere le relative soluzioni in risposta alle problematiche riscontrate nel processo stesso.
\end{enumerate}

\subsubsection{Miglioramento del processo}
L'attività si articola nel compito che prevede la realizzazione di ogni contromisura che il gruppo ritiene utile al miglioramento e alla maturazione di un processo. Questo compito è svolto tramite l'applicazione intelligente del ciclo di Deming, conosciuto anche come ciclo \ccgloss{PDCA}, composto dalle seguenti azioni iterative:
\begin{enumerate}
    \item Plan: il gruppo pianifica le modifiche da adottare nella realizzazione di un processo;
    \item Do: il gruppo attua le modifiche individuate nella precedente fase di pianificazione;
    \item Check: il gruppo effettua una valutazione del processo svolto con le modifiche, individuando i miglioramenti e cosa è ancora migliorabile;
    \item Act: quanto di positivo individuato nella fase di "check" diventa parte effettiva del processo. Quello che è ancora migliorabile, è nuovo oggetto della fase di "plan" del nuovo ciclo.
\end{enumerate}
I risultati dell'evoluzione del way of working devono essere opportunamente documentati nella redazione di verbali interni.

%\subsection{Strumenti}
\newpage

\section{Formazione del personale} \label{sec:formpers}
\subsection{Definizione}
Il processo organizzativo di formazione del personale sottolinea l'importanza di rendere i membri del gruppo adeguatamente formati per permettere il corretto svolgimento delle attività e dei compiti relativi ad ogni processo. Il processo contiene l'insieme delle attività che includono la pianificazione e l'implementazione di misure rivolte alla formazione del personale che compone il gruppo, incentivando l'auto-apprendimento utile a svolgere le mansioni necessarie al corretto svolgimento del progetto.
\subsection{Attività}
Il processo di formazione del personale si articola nelle seguenti attività:
\begin{enumerate}
    \item istanziazione del processo;
    \item materiale di formazione;
    \item istanziazione del piano di formazione.
\end{enumerate}
\subsubsection{Istanziazione del processo}
L'attività si articola nel compito di determinare attentamente i requisiti di progetto, attraverso un'analisi tempestiva mirata a conoscere quanto prima cosa i singoli membri sono chiamati ad imparare, formulando un piano di formazione adeguato. In particolare, la formazione dei membri del gruppo deve riguardare sia le tecnologie che saranno utilizzate nello sviluppo del prodotto del progetto, sia le tecnologie utili alla realizzazione dei processi attivi.
\subsubsection{Materiale di formazione}
L'attività si articola nel compito di individuare e fornire ad ogni membro del gruppo i materiali con cui possono formarsi. Nello studio individuale delle tecnologie adottate per lo sviluppo del prodotto del progetto, ogni membro condivide con tutto il gruppo le documentazioni, di ogni tipologia, che ritengono di sostegno all'auto-apprendimento di ognuno.\\Per formare ogni membro del team all'utilizzo delle tecnologie e  infrastrutture tecniche a sostegno dei processi, in sede di riunione interna, gli amministratori sono chiamati ad esporre tali tecnologie e il loro funzionamento.\\Inoltre, ogni documentazione informativa fornita dall'azienda proponente deve far parte del processo di formazione di ogni membro del gruppo.
\subsubsection{Istanziazione del piano di formazione}
L'attività si articola nei seguenti compiti:
\begin{enumerate}
    \item istanziazione del piano di formazione individuato dalle attività precedenti, registrando eventuali lacune così da poterlo aggiornare migliorandolo e rendendolo adeguato alle necessità;
    \item accertamento della disponibilità di persone formate a sostegno dei membri che richiedono assistenza. Ogni membro del gruppo deve richiedere supporto ai compagni qualora necessario.
\end{enumerate}
\subsection{Strumenti}
Durante lo svolgimento dei compiti vengono utilizzati i seguenti strumenti:
\begin{itemize}
    \item Discord per la realizzazione di video chiamate, nelle quali esporre agli altri membri del gruppo il funzionamento delle tecnologie e delle infrastrutture tecniche, di supporto alla realizzazione del progetto, predisposte da un amministratore;
    \item Telegram per condividere materiali informativi sulle tecnologie individuate dal gruppo, con cui ogni membro può realizzare il processo di auto-apprendimento;
    \item Google Drive per condividere guide riguardanti le tecnologie utilizzate.
\end{itemize}
