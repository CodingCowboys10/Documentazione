\chapter{Processi primari}
\section{Fornitura} \label{sec:fornitura}
\subsection{Definizione}
Il processo primario di fornitura fa riferimento agli oneri del fornitore, dalla decisione di rispondere all'opportunità creata dagli \ccgloss{stakeholders} fino alla realizzazione del progetto correlato. Il processo contiene l'insieme delle attività che includono la formulazione di una proposta al \ccgloss{committente}, la pianificazione delle attività e delle \ccgloss{risorse} necessarie alla realizzazione del \ccgloss{progetto}, la stesura di un piano di progetto, il controllo e la valutazione dello stato di avanzamento, fino al completamento di esso.
\subsection{Attività}
Il processo di fornitura si articola nelle seguenti attività:
\begin{enumerate}
    \item preparazione della proposta;
    \item stipulazione del contratto;
    \item pianificazione;
    \item realizzazione e controllo;
    \item revisione e valutazione;
    \item consegna e completamento.
\end{enumerate}

\subsubsection{Preparazione della proposta}
L'attività si articola nel compito che prevede la formulazione, da parte del fornitore, di una proposta con cui rispondere all'opportunità presentata dai committenti e dai \ccgloss{proponenti}.\\
La preparazione della proposta avviene tramite la redazione di una lettera di presentazione rivolta ai committenti, corredata da un documento indicante il \ccgloss{preventivo} dei costi e dei tempi necessari alla realizzazione del progetto nella sua interezza e da un documento che riporta l'analisi dettagliata dell'opportunità.

\subsubsection{Stipulazione del contratto}
L'attività si articola nel compito che prevede la negoziazione e stipulazione del contratto con il committente.\\
A seguito dell'invio della documentazione descritta dall'attività precedente, il contratto è da ritenersi stipulato a seguito dell'accettazione della proposta da parte del committente, e varranno i vincoli riconosciuti nel documento di preventivo dei costi e tempi.

\subsubsection{Pianificazione}
L'attività si articola nei seguenti compiti:
\begin{enumerate}
    \item Il fornitore è chiamato a realizzare un'analisi dei \ccgloss{requisiti} del progetto, così da poter assicurare la qualità del prodotto. I requisiti identificati in questa fase del progetto, che andranno ad essere raccolti in un documento apposito denominato Analisi dei requisiti, andranno a determinare i test necessari alla verifica e validazione del prodotto finale del progetto.
    \item Deve essere adottato un modello di ciclo di vita del software adeguato alla grandezza, alla complessità e alla portata del progetto.
    \item Devono essere stabiliti, da parte del fornitore, i requisiti necessari alla gestione e verifica del progetto e del prodotto, oltre che alla verifica della qualità di tali.
    \item Deve essere sviluppato e documentato il piano di gestione del progetto, basandosi sui requisiti individuati al punto precedente, e deve essere redatta una documentazione per descrivere tale piano che si baserà sui seguenti punti:
    \begin{itemize}
        \item Struttura organizzativa del progetto e delle autorità. Sono identificati dei \ccgloss{ruoli}, con compiti, costi e responsabilità differenti, che i membri del team fornitore dovranno ricoprire lungo tutto il progetto, variando tra essi.
        \item Ambiente tecnico. Devono essere riportati gli strumenti a sostegno della realizzazione dei compiti e delle relative attività parte dei processi in atto nel progetto.
        \item Scomposizione dei processi e delle attività in \ccgloss{task} eseguibili, in considerazione del budget, delle risorse e del personale disponibile. Nel Piano di progetto, la pianificazione delle risorse deve dar luogo ad un preventivo dei costi, in termine di tempo e risorse, necessari al completamento del progetto. Le task eseguibili saranno assegnate tramite \ccgloss{ticket} a singoli membri del gruppo, coordinando il lavoro del team tramite un opportuno \ccgloss{framework} di \ccgloss{project management}, come \ccgloss{Scrum}.
        \item Gestione delle caratteristiche qualitative dei prodotti dei processi. In tal senso, deve essere redatto un documento contenente tali metriche, denominato Piano di qualifica.
        \item Accertamento della qualità, con il relativo processo di supporto.
        \item Verifica e Validazione, con i relativi processi di supporto.
        \item Revisioni congiunte con il cliente, con il relativo processo di supporto mirato al coinvolgimento costante del proponente nella realizzazione del progetto. Un'opportuna verbalizzazione di tali incontri andrà a documentare il loro effettivo avvenimento.
        \item Gestione dei rischi. Un omonimo capitolo deve essere redatto all'interno del Piano di progetto, andando ad individuare tutti i potenziali rischi riscontrabili, stimando occorrenza e pericolosità, e identificando un'azione mitigativa da intraprendere in caso di occorrenza effettiva della problematica.
        \item Mezzi di sostegno alla pianificazione e all'analisi dell'avanzamento effettivo. Per facilitare le operazioni di pianificazione, devono essere prodotti dei diagrammi di \ccgloss{Gantt}, con cui rappresentare graficamente la dislocazione temporale delle attività, rappresentandone la durata, la sequenzialità e il parallelismo.
        \item Formazione del personale, con il relativo processo organizzativo.
    \end{itemize}
\end{enumerate}

\subsubsection{Realizzazione e controllo}
L'attività si articola nei seguenti compiti:
\begin{enumerate}
    \item Osservazione oculata di ciò che prevede il piano di gestione del progetto sviluppato nell'attività precedente.
    \item Controllo continuo, sistematico ed iterativo dello stato di avanzamento e della qualità dei prodotti del progetto. Questo compito deve inoltre prevedere:
    \begin{itemize}
        \item Monitoraggio dello stato di avanzamento in relazione all'utilizzo di risorse, sia in termini di budget che di personale, tramite la realizzazione di \ccgloss{consuntivi} di periodo, analizzando il discostamento di questi ultimi rispetto le stime preventivate nella fase di pianificazione. A sostegno di tale compito, opportune metriche saranno adottate per costituire i grafici del cruscotto valutativo, o \ccgloss{dashboard}, presente nel documento denominato Piano di qualifica.
        \item Identificazione di problemi con opportune fasi retrospettive, documentando ed analizzando questi momenti, con l'intento di giungere ad una soluzione e ad un miglioramento continuo.
    \end{itemize}
\end{enumerate}

\subsubsection{Revisione e valutazione}
L'attività si articola nei seguenti compiti:
\begin{enumerate}
    \item Il fornitore deve essere parte attiva nella coordinazione delle comunicazioni con l'acquirente. È nell'interesse del team cercare quanto più il confronto con l'azienda proponente, così da ricevere costantemente feedback sull'operato del gruppo.
    \item Il dialogo con il proponente deve essere supportato dal relativo processo di supporto Revisioni congiunte con il proponente.
    \item Deve essere garantito il rispetto di quanto indicato nei processi di supporto relativi a Verifica e Validazione, e in particolare nel documento Piano di qualifica, in modo da dimostrare che prodotti e processi rispettano pienamente i loro relativi requisiti.
    \item Devono essere messe a disposizione dell'acquirente le documentazioni relative all'analisi dei requisiti e alla qualità dei prodotti del progetto.
    \item Deve essere garantito il corretto svolgimento di quanto indicato nel processo di supporto Accertamento della qualità.
\end{enumerate}

\subsubsection{Consegna e completamento}
L'attività si articola nel compito di fornire all'acquirente, una volta ultimato il progetto, l'accesso al prodotto secondo quanto concordato congiuntamente.


\subsection{Strumenti}
Durante lo svolgimento dei compiti vengono utilizzati i seguenti strumenti:
\begin{itemize}
    \item LucidChart per la realizzazione di diagrammi di Gantt;
    \item \ccgloss{Jira Software} e Scrum come strumento di \ccgloss{Issue Tracking System} e di project management \ccgloss{agile}; 
    \item Microsoft Excel per la realizzazione dei grafici relativi a preventivi e consuntivi;
    \item \ccgloss{Telegram} come mezzo per le comunicazioni asincrone tra membri del gruppo;
    \item \ccgloss{Discord} per sostenere le riunioni da remoto tra membri del gruppo;
    \item Google Drive per la condivisione di materiale di supporto al progetto.
\end{itemize}

\subsection{Metriche}
Durante il processo vengono utilizzate le seguenti metriche:
\begin{itemize}
    \item Varianza di Budget (MPC-1);
    \item Varianza dell'impegno orario (MPC-2);
    \item \ccgloss{Earned Value} (MPC-3);
    \item \ccgloss{Actual Cost} (MPC-4);
    \item \ccgloss{Planned Value} (MPC-5);
    \item \ccgloss{Cost Variance} (MPC-6);
    \item \ccgloss{Schedule Variance} (MPC-7);
    \item \ccgloss{Cost Perfromance Index} (MPC-8);
    \item \ccgloss{Schedule Performance Index} (MPC-9);
    \item \ccgloss{Estimate to Complete} (MPC-10);
    \item \ccgloss{Estimate at Completion} (MPC-11);
    \item \ccgloss{Budget at Completion} (MPC-12).
\end{itemize}
Per una descrizione di tali metriche si faccia riferimento al documento Piano\_di\_qualifica\_v1.0.

\newpage

\section{Sviluppo} \label{sec:sviluppo}
\subsection{Definizione}
Il processo primario di sviluppo contiene l'insieme delle attività che includono l'analisi dei requisiti, la progettazione, l'implementazione e i test relativi al prodotto del progetto.

\subsection{Attività}
Il processo di sviluppo si articola nelle seguenti attività:
\begin{enumerate}
    \item istanziazione del processo;
    \item analisi dei requisiti;
    \item progettazione architetturale;
    \item progettazione di dettaglio;
    \item codifica e prova dei componenti;
    \item integrazione dei componenti.
    %\item collaudo;
    %\item installazione;
    %\item supporto per l'accettazione.
\end{enumerate}
\subsubsection{Istanziazione del processo}
L'attività si articola nei seguenti compiti:
\begin{enumerate}
    \item Deve essere adottato un modello di ciclo di vita del software adeguato alla grandezza, alla complessità e alla portata del progetto.
    \item Lo sviluppatore è chiamato a:
    \begin{itemize}
        \item documentare i prodotti di ogni processo secondo quanto stabilito dal relativo processo di supporto di documentazione;
        \item conformare gli output di ogni processo secondo quanto stabilito dal relativo processo di supporto di gestione della configurazione;
        \item documentare e risolvere i problemi riscontrati nell'output di ogni processo secondo quanto stabilito dal relativo processo di supporto di risoluzione dei problemi;
    \end{itemize}
    \item Devono essere adottate tecnologie e strumenti adeguati al sostegno del way of working del gruppo, e saranno individuate tecnologie e linguaggi di programmazione utili alla realizzazione del progetto in considerazione di richieste o consigli del proponente.
    \item Devono essere individuati ruoli con responsabilità e compiti specifici per la realizzazione delle attività di sviluppo. Le attività di analisi saranno gestite da prestabilite figure denominate \ccgloss{analisti}, con quelle di progettazione architetturale e di dettaglio assegnate ai \ccgloss{progettisti}, quelle di codifica ed implementazione compito dei \ccgloss{programmatori} e quelle di collaudo e test oggetto del lavoro dei \ccgloss{verificatori}.
\end{enumerate}

\subsubsection{Analisi dei requisiti}
L'attività si articola nei seguenti compiti:
\begin{enumerate}
    \item Devono essere individuati e documentati i requisiti funzionali, di qualità, di vincolo e prestazionali del prodotto. In particolare nella documentazione, denominata Analisi dei requisiti, dovranno essere presenti riferimenti a:
    \begin{itemize}
        \item definizione dello scopo del prodotto e individuazione delle relative funzionalità;
        \item analisi e classificazione degli utenti che potranno interagire;
        \item definizione dei casi d'uso. Essi dovranno essere dettagliatamente descritti attraverso un diagramma \ccgloss{UML} degli \ccgloss{use cases} e la definizione testuale di:
        \begin{enumerate}
            \item nome del caso d'uso;
            \item descrizione generale;
            \item \ccgloss{attore} primario;
            \item attori secondari, quando presenti;
            \item pre condizioni;
            \item post condizioni;
            \item scenario;
            \item generalizzazioni, quando presenti;
            \item estensioni, quando presenti.
        \end{enumerate}
    \item definizione dei requisiti esponendo:
    \begin{enumerate}
        \item codice alfanumerico del requisito, definito come R[\textit{T}][\textit{C}]-[\textit{N}] dove:
        \begin{enumerate}
            \item \textit{T} identifica la tipologia del requisito, che può essere funzionale (F), di qualità (Q), di vincolo (V), prestazionale (P) o implementativo (I);
            \item \textit{C} identifica la classificazione dell'importanza, basata su una scala a tre valori comprendente requisiti obbligatori (O), desiderabili (D) ed opzionali (Z);
            \item \textit{N} identifica il numero del requisito di quella tipologia.
        \end{enumerate}
        \item descrizione generale;
        \item classificazione dell'importanza, basata su una scala a tre livelli, come precedentemente esposto;
        \item fonte da cui tale requisito deriva.
    \end{enumerate}
    \end{itemize}
    \item Redazione di un documento apposito, richiesto dall'azienda proponente, che comprenda:
    \begin{itemize}
    \item analisi dei costi e delle tecnologie, nella quale riportare la stima dei costi reali basandosi sulle tecnologie, open-source o meno, impiegabili nel progetto;
    \item analisi dei rischi legati alla sicurezza del sistema e alla protezione dei dati sensibili.
    \end{itemize}
    \item I prodotti dell'analisi dei requisiti devono essere oggetto di opportune revisioni da parte del proponente, in accordo con la ricerca di feedback continui nel tempo. Il parere del proponente sarà inoltre necessario per la classificazione dei requisiti del prodotto, oltre che per la definizione delle tecnologie utili alla realizzazione del progetto.
    \item Parallelamente all'attività di analisi e definizione dei requisiti, devono essere definiti i test da eseguire per verificare che il prodotto del progetto soddisfi tali requisiti. Ogni test individuato deve essere documentato nel Piano di qualifica e deve esporre:
    \begin{enumerate}
        \item codice alfanumerico del test, definito come T[\textit{C}]-[\textit{N}] dove:
        \begin{enumerate}
            \item \textit{C} identifica la tipologia del test, che può essere di sistema (S) o di accettazione (A);
            \item \textit{N} identifica il numero del requisito di quella tipologia.
        \end{enumerate}
        \item descrizione generale;
        \item requisito da cui tale test deriva;
        \item tracciamento della soddisfazione del requisito.
    \end{enumerate}
    La definizione dei test a seguito dell'attività di analisi è coerente a quanto prefigge il \ccgloss{V model}.
\end{enumerate}

\subsubsection{Progettazione architetturale}
L'attività si articola nei seguenti compiti:
\begin{enumerate}
    \item È necessario trasformare i requisiti, identificati mediante la relativa attività, in un'architettura del prodotto, identificando i componenti necessari e scopi e funzioni ad essi associati.
    \item Deve essere sviluppata una progettazione che tenga in considerazione l'interazione dei singoli componenti tra loro, verificando che ogni componente faccia riferimento a uno o più requisiti e viceversa.
    \item Devono essere identificati i test con cui verificare la soddisfazione dei requisiti che danno origine ai componenti, e devono essere riportati nel documento opportuno denominato Piano di qualifica.
    \item Devono essere svolti incontri interlocutori con il proponente, al fine di ottenere feedback su quanto prodotto in questa attività. Il parere del proponente sarà inoltre necessario per la definizione dei componenti da progettare, e successivamente codificare e implementare, per la prima fase di studio di fattibilità del progetto e delle sue tecnologie, con la realizzazione di un prodotto software denominato \ccgloss{Proof of Concept}.
\end{enumerate}

\subsubsection{Progettazione di dettaglio}
L'attività si articola nei seguenti compiti:
\begin{enumerate}
    \item Ogni singolo componente deve essere oggetto di una progettazione dettagliata, identificando le unità che lo compongono. La codifica, l'implementazione e il test di ogni unità software deve poter essere assegnabile ad un membro come task: questa direttiva rappresenterà il metro per identificare la grandezza e la complessità che ogni unità dovrà avere;
    \item Una opportuna documentazione deve essere redatta per esporre le specifiche prodotte nelle fasi di progettazione;
    \item Deve essere verificato che le unità progettate soddisfino i requisiti del prodotto;
    \item Devono essere svolti incontri interlocutori con il proponente, al fine di ottenere feedback su quanto prodotto in questa attività.
\end{enumerate}

\subsubsection{Codifica e prova dei componenti}
L'attività si articola nei seguenti compiti:
\begin{enumerate}
    \item Devono essere codificati tutte le unità software identificate durante la progettazione;
    \item Tutte le unità prodotte devono essere testate al fine di accertare il loro funzionamento rispetto i loro rispettivi requisiti e compiti;
    \item Una opportuna documentazione deve essere redatta per esporre i prodotti di questa attività;
    \item Deve essere verificato che quanto prodotto nel corso di questa attività rispetti i requisiti del prodotto.
\end{enumerate}

\subsubsection{Integrazione dei componenti}
L'attività si articola nei seguenti compiti:
\begin{enumerate}
    \item Deve essere sviluppato un piano di integrazione per i componenti codificati nella relativa attività;
    \item I componenti e le singole unità software devono essere integrati secondo quanto deciso dal relativo piano di integrazione, verificando che l'integrazione soddisfi i requisiti del prodotto.
    \item Devono essere svolti incontri interlocutori con il proponente, al fine di ottenere feedback su quanto prodotto in questa attività. In riferimento alla fase di studio di fattibilità del progetto e delle tecnologie, l'incontro con il proponente sarà necessario all'esposizione del prodotto denominato Proof of Concept, con il quale dimostrare la fattibilità del prodotto richiesto dal progetto in relazione alle tecnologie adottate dal gruppo in accordo con il proponente stesso.
\end{enumerate}

%\subsubsection{Collaudo}
%\subsubsection{Installazione}
%\subsubsection{Supporto per l'accettazione}

\subsection{Strumenti}
Durante lo svolgimento dei compiti vengono utilizzati i seguenti strumenti:
\begin{itemize}
    \item StarUML per la creazione dei diagrammi UML dei casi d'uso, prodotti nell'attività di analisi dei requisiti;
    \item \ccgloss{Ollama}, e i relativi modelli linguistici forniti, per l'\ccgloss{embedding} dei documenti e la fase di retrieve per l'implementazione del \ccgloss{chatbot};
    \item \ccgloss{OpenAi} tecnologia usata in alternativa a quelle del punto precedente, ma con stesso fine;
    \item \ccgloss{Langchain} per la creazione dell'applicazione semplificando l'interazione tra \ccgloss{LLM} e database;
    \item \ccgloss{Chromadb} per lo store dei vettori creati con l'embedding dei documenti;
    \item \ccgloss{SQLite} per lo store delle chat;
    \item \ccgloss{React} per la realizzazione dell'interfaccia utente in \ccgloss{TypeScript};
    \item \ccgloss{Node.js} per l'esecuzione di codice TypeScript;
    \item \ccgloss{NextJs} come framework per lo sviluppo;
    \item Web Storm come IDE per la codifica del prodotto;
    \item MinIO per la gestione e l'archiviazione dei documenti.
\end{itemize}

\subsection{Metriche}
Durante il processo vengono utilizzate le seguenti metriche:
\begin{itemize}
    \item \ccgloss{Code Coverage} (MPC-13);
    \item Requisiti obbligatori soddisfatti (MPD-2);
    \item Requisiti desiderabili soddisfatti (MPD-3);
    \item Requisiti opzionali soddisfatti (MPD-4).
\end{itemize}
Per una descrizione di tali metriche si faccia riferimento al documento Piano\_di\_qualifica\_v1.0.
\newpage


%\section{Gestione operativa}
%\subsection{Definizione}
%Installazione ed erogazione del prodotto e/o servizi
%\subsection{Attività}
%Il processo di gestione operativa si articola nelle seguenti attività:
%\begin{enumerate}
%    \item istanziazione del processo;
%    \item testing operativo;
%    \item operazioni di sistema;
%    \item supporto utente.
%\end{enumerate}
%\subsubsection{Istanziazione del processo}
%\subsubsection{Testing operativo}
%\subsubsection{Operazioni di sistema}
%\subsubsection{Supporto utente}
%\subsection{Strumenti}
%\subsection{Metriche}
