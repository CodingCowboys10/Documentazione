\chapter{Pianificazione}\label{chap:pianificazione}
La pianificazione è una parte fondamentale per ogni progetto. Essa ha il compito di definire le attività utili al completamento del progetto, pianificando il suo svolgimento tramite l'assegnazione intelligente di risorse ad ogni attività individuata, permettendo di valutare il reale grado di avanzamento dei lavori stimando e controllando i costi e le tempistiche effettive rispetto a quelle preventivate. Il prodotto di ogni periodo rappresenterà la \ccgloss{baseline} relativa agli obiettivi prefissati per quel lasso temporale.\\ \\
La pianificazione riportata di seguito tenta di raggiungere un buon grado di precisione non solo considerando l'analisi dei rischi individuati nella sezione precedente, ma anche tramite due particolari accorgimenti. La pianificazione generale avviene a ritroso, fissando il punto di arrivo previsto, con relativi vincoli finali, e individuando le attività che portano a tale obiettivo. La pianificazione dettagliata viene limitata ad un orizzonte temporale relativamente vicino, così che eventuali errori nella fase di pianificazione abbiano un impatto minore nello svolgimento del progetto, permettendo un'agile azione di riorganizzazione. 

\section{Modello adottato}
Dopo una prima discussione sui possibili \ccgloss{modelli di sviluppo} utilizzabili, nella realizzazione di questo progetto il gruppo ha deciso di adottare un modello di sviluppo \ccgloss{agile}, consigliato anche dall'azienda proponente.\\ \\
Il motivo di questa scelta è da ricercarsi principalmente nell'elevato grado di flessibilità che tale modello permette. Oltre a mettere in primo piano l'importanza dell'interazione continua e persistente con il proponente, importantissimo per l'analisi dei requisiti del prodotto finale, l'utilizzo di questo modello permette la verifica dello stato di avanzamento reale del progetto, mostrando al proponente quanto fatto, tramite la suddivisione del lavoro in incrementi a valore aggiuntivo.\\ \\
Per adottare questo modello di sviluppo, è stato deciso di utilizzare degli \ccgloss{sprint} dalla durata di due settimane l'uno, aiutandosi nella pianificazione e realizzazione del progetto utilizzando il \ccgloss{framework} \ccgloss{Scrum}, e accompagnando ogni \ccgloss{consuntivo} di periodo con una analisi retrospettiva.

\section{Requirements and Technology Baseline}
\subsection{Primo sprint: 2023/11/06 - 2023/11/19}
\textbf{Obiettivi}
\begin{itemize}
    \item Inizio stesura del Glossario;
    \item Inizio stesura del Piano di Progetto;
    \item Inizio stesura delle Norme di Progetto;
    \item Inizio stesura Analisi dei Requisiti
\end{itemize}

\subsection{Secondo sprint: 2023/11/20 - 2023/12/03}
\textbf{Obiettivi}
\begin{itemize}
    \item Proseguire la stesura del Piano di Progetto;
    \item Perfezionare il documento di Analisi dei requisiti;
    \item Proseguire la stesura delle Norme di Progetto;
    \item Ultimazione stesura del Glossario;
    \item Stesura del Piano di Qualifica;
    \item Progettazione del \ccgloss{PoC};
    \item Completamento compilazione automatica dei documenti \ccgloss{\LaTeX}.
\end{itemize}

\section{Product Baseline}


\newpage