\section{Resoconto} \label{sec:resoconto}
\subsection{Discussioni} \label{subsec:resdiscussione}
\begin{enumerate}
    \item L'incontro si è aperto con la presentazione del Proof of Concept, in forma completa, ai rappresentanti dell'azienda proponente.\\Il PoC esposto,mirato a dimostrare la visione di insieme di tutte le tecnologie funzionanti ed interconnesse, ha raggiunto il suo scopo svolgendo le attività già svolte dalla Demo PoC con l'integrazione di una interfaccia grafica e delle migliorie richieste dai proponenti.\\È stata presentata la nuova interfaccia grafica realizzata mediante libreria \ccgloss{React} che va a sostituire il controllo tramite console, la quale è stata accolta positivamente.\\È stato presentato il sistema di selezione di modelli multipli(5) il quale permette di avere un confronto diretto sulla qualità del modello utilizzato.\\È stato presentato anche il sistema di valutazione dell'efficienza dei modelli con risultati che risaltano la bontà del modello di \ccgloss{OpenAI} a scapito di quelli \ccgloss{Open-source}, in generale i modelli si sono verificati essere: OpenAI risultava preciso e accurato, OpenChat e Starling risultavano precisi ma meno accurati infine llama2 e Mistral risultano poco precisi e poco accurati.\\Si è discusso riguardo all'uso di text splitters come il sistema a chunk, come proposto dai proponenti nella riunione precedente, è stato fatto notare però che l'uso di questa tecnologia in caso di documenti di dimensione ridotta causa la perdita di informazioni a causa della dimensione dei chunk troppo piccola. Si è optato perciò per un sistema di splitting chiamato \ccgloss{Parent Document Retriever} il quale divide il documento in piccole sezioni ma, successivamente, compie una ricerca macroscopica all'interno del documento riducendo così la perdita di informazione, il quale verrà implementato per il \ccgloss{MVP}.\\È stato effettuato un test di funzionamento del sistema mediante il caricamento di un secondo documento, il quale ha evidenziato la presenza di qualche bug da risolvere nel prossimo futuro.\\È stato inoltre dimostrato che il gruppo ha deciso, come da proposta del proponente, di migrare il lavoro in framework NextJS in quanto migliora lo sviluppo e manutenibilità del prodotto.\\Come da contrattazioni avvenute in fase di inizio lavori la sezione riguardante il caricamento dei documenti verrà trattata in un secondo momento in quanto non centrale allo sviluppo del progetto.
    \item A seguito della presentazione del PoC sono stati ricevuti dei consigli sulla possibilità di utilizzo di MinIO, un sistema di archiviazione di oggetti ad alta prestazioni ottimizzato per progetti di AI, il gruppo provvederà ad analizzare suddetta tecnologia ed eventualmente implementarla.
    \item L'incontro si conclude con la conferma, da parte di AzzurroDigitale, dell'operato del PoC. Il gruppo ha l'obbiettivo di redigere e inviare un documento di analisi del progetto e dei costi al proponente per dimostrarne il valore di Business. Inoltre il gruppo si impegnerà per il prossimo incontro a sistemare il bug rinvenuto alla presenza di file multipli. 
\end{enumerate}

\subsection{Prossima riunione} \label{subsec:riunione}
È stata fissata una riunione per mercoledì 20 dicembre nella quale si presenterà la risoluzione del bug prima riportato e si terrà una prima discussione sugli obbiettivi futuri in ottica MVP.
