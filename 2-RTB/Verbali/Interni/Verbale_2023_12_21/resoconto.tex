\section{Resoconto} \label{sec:resoconto}
\subsection{Discussioni} \label{subsec:resdiscussione}
\begin{enumerate}
    \item L'incontro si apre con una discussione sui requisiti prestazionali del prodotto. A seguito di un primo contatto asincrono con l'azienda \ccgloss{proponente}, è stato deciso di contattarla nuovamente. L'intento di questo nuovo contatto è confermare l'inserimento di tali requisiti nel documento di Analisi\_dei\_requisiti.
    \item Viene successivamente discusso l'intento di rendere automatici i riferimenti al Glossario nei vari documenti. Anna Nordio sta, a tal proposito, sviluppando uno script da integrare alle automazioni già presenti nella \ccgloss{repository} del gruppo. L'obiettivo è agevolare il lavoro dei vari redattori, riducendo il tempo impiegato per svolgere azioni meccaniche automatizzabili.
    \item Viene fatto un recap dello stato dei documenti da presentare nella successiva revisione di avanzamento. L'attenzione è focalizzata sui documenti Analisi\_dei\_requisiti e Piano\_di\_qualifica, entrambi da completare quanto prima. Lo stesso vale per Norme\_di\_progetto, in uno stato molto avanzato ma non ancora completo.
\end{enumerate}

\subsection{Votazioni} \label{subsec:resvotazione}
     È stata votata la data preferita nella quale svolgere il diario di bordo natalizio. La preferenza sarà da comunicare agli altri gruppi. \\\\
        \begingroup
            \setlength{\tabcolsep}{10pt}
            \renewcommand{\arraystretch}{1.5}
            \rowcolors{2}{oddrow}{evenrow}
            \begin{tabularx}{0.93\textwidth}{| l | l | X |}
                 \hline
                 \rowcolor{headerrow}\textbf{\textcolor{white}{Proposta}} & \textbf{\textcolor{white}{Sommario}} & \textbf{\textcolor{white}{Mittente}} \\
                 \hline
                 28/29 Dicembre 2023 & 84\%  & Anna Nordio, Francesco Ferraioli, Francesco Giacomuzzo, Leonardo Lago, Marco Dolzan. \\
                 \hline
                 4/5 Gennaio 2024 & 16\% & Andrea Cecchin. \\
                 \hline
                 Astenuti & 0\% &  \\
                 \hline
            \end{tabularx}
        \endgroup

\subsection{Prossima riunione} \label{subsec:riunione}
Viene fissata una riunione per la settimana successiva.
