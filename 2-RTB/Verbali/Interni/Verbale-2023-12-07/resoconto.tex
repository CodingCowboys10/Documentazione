\section{Resoconto} \label{sec:resoconto}
\subsection{Discussioni} \label{subsec:resdiscussione}
\begin{enumerate}
    \item Durante questa settimana, attraverso discussioni interne e colloqui con l'azienda, abbiamo potuto esaminare gli obiettivi per questo periodo, valutando aspetti positivi e criticità. Abbiamo inoltre ricevuto feedback e indicazioni dall'azienda riguardo al \emph{PoC\ped{G}} presentato. Nella riunione odierna sono stati delineati diversi compiti da completare entro il termine di questo sprint. Essi comprendono: 
    \begin{enumerate}
        \item La parte di progettazione e programmazione del PoC, prevedendo la necessità di un refactoring del front-end per renderlo più semplice e pratico da utilizzare; 
        \item la parte di documentazione, ponendo attenzione ai documenti da terminare, quali piano di qualifica e analisi dei requisiti;
        \item l'organizzazione della repository, prevedendo un refactoring della sua struttura per via delle nuove cartelle di documenti da inserire;
        \item L'implementazione di alcune metriche per la valutazione, per quanto riguarda il codice attraverso la \emph{code coverage\ped{G}}; mentre per quanto riguarda il testo presente nei documenti l'indice di \emph{Gunning Fog\ped{G}}. 
     \end{enumerate}

    \item Sulla base dei ruoli e delle attività, la fase conclusiva dell'incontro è stata riservata alla formulazione e all'assegnazione di nuovi compiti ai membri del gruppo tramite l'Issue Traking System \emph{Jira\ped{G}}.

\end{enumerate}


\subsection{Azioni da intraprendere}
{
    \setlength{\tabcolsep}{10pt}
    \renewcommand{\arraystretch}{1.5}
    \rowcolors{2}{oddrow}{evenrow}
    \begin{xltabular}{\textwidth}{| l | l | l | X |}
        \hline
        \rowcolor{headerrow}\textbf{\textcolor{white}{Codice issue}} & \textbf{\textcolor{white}{Assegnatario}} & \textbf{\textcolor{white}{Scadenza}} & \textbf{\textcolor{white}{Descrizione}} \\
        \endhead
        \hline
        CC-68 & Leonardo Lago & 2023/12/17 & Stesura qualità di processo. \\
        \hline
        CC-69 & Andrea Cecchin & 2023/12/17 & Stesura qualità di prodotto. \\
        \hline
        CC-74 & Marco Dolzan & 2023/12/17 & Realizzazione componenti del chat body.  \\
        \hline
        CC-75 & Marco Dolzan & 2023/12/17 & Realizzazione componenti del chat message. \\
        \hline
        CC-76 & Giovanni Menon & 2023/12/17 & Realizzazione componenti del menu llm.\\
        \hline
        CC-77 & Marco Dolzan & 2023/12/17 & Realizzazione componenti del text input.\\
        \hline
        CC-78 & Giovanni Menon & 2023/12/17 & Realizzazione componenti degli llm.\\
        \hline
        CC-79 & Giovanni Menon & 2023/12/17 & Correzione del database vettoriale. \\
        \hline
        CC-80 & Giovanni Menon & 2023/12/17 & Realizzazione stream message. \\
        \hline
        CC-81 & Giovanni Menon & 2023/12/17 & Effettuare prompt testing. \\
        \hline
        CC-82 & Francesco Giacomuzzo & 2023/12/17 & Aggiunta nuovi termini al glossario. \\ 
        \hline
        CC-83 & Francesco Ferraioli & 2023/12/17 & Stesura requisiti di qualità. \\
        \hline
        CC-84 & Anna Nordio & 2023/12/17 & Stesura requisiti dei vincoli. \\
        \hline
        CC-85 & Francesco Ferraioli & 2023/12/17 & Stesura requisiti prestazionali. \\
        \hline
        CC-86 & Leonardo Lago & 2023/12/17 & Creazione tabella di riepilogo dei requisiti. \\
        \hline
        CC-87 & Anna Nordio & 2023/12/17 & Aggiornamento degli scenari relativi a Use Case. \\
        \hline
        CC-88 & Leonardo Lago & 2023/12/17 & Introduzione nuovi Use Case. \\
        \hline
        CC-90 & Francesco Giacomuzzo & 2023/12/08 & Redazione verbale 7 dicembre. \\
        \hline
        CC-91 & Andrea Cecchin & 2023/12/17 & Creazione query di connessione al database. \\
        \hline
        CC-92 & Andrea Cecchin & 2023/12/17 & Implementazione della visualizzazione dei documenti. \\
        \hline 
        CC-93 & Francesco Ferraioli & 2023/12/17 & Introduzione piano di qualifica. \\
        \hline
    \end{xltabular}
}


%\subsection{Prossima riunione} \label{subsec:riunione}
%Viene fissata una nuova riunione per lunedì 11 dicembre.