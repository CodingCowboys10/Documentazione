\section{Resoconto} \label{sec:resoconto}
\subsection{Discussioni} \label{subsec:resdiscussione}
\begin{enumerate}
    \item Vengono assegnati i ruoli che ogni membro sarà chiamato a ricoprire nel prossimo periodo. Tali ruoli sono stati decisi in base alla necessità di ogni figura in questo momento di progetto\ped{G}. Rispetto al periodo precedente è stato assegnato il ruolo del progettista a più membri, che si dedicheranno alla strutturazione del \emph{PoC\ped{G}}.
    \item Nel corso della fase di retrospettiva, è stata esaminata la precedente finestra temporale al fine di valutare il raggiungimento degli obiettivi stabiliti durante il primo \emph{sprint\ped{G}}. Si è condotta un'analisi dettagliata dei punti di forza e delle criticità di ciascuna attività, con l'intento di trasferire tali conoscenze a coloro che assumeranno nuovi ruoli nel prossimo ciclo lavorativo.
    Si è deciso inoltre di evitare di protrarre le riunioni per un tempo eccessivo, e di pianificarne di nuove in caso si sforasse l'ora e mezza.
    \item Si procede con una dimostrazione pratica delle tecnologie \emph{LLM\ped{G}}, finalizzata a illustrare il loro funzionamento e a evidenziare le differenze tra i diversi modelli disponibili.
    \item Sono stati individuati i principali compiti necessari per raggiungere gli obiettivi dello specifico \ccgloss{sprint}. Attraverso l'utilizzo del sistema \emph{ITS\ped{G}} \emph{Jira\ped{G}}, sono state generate le \emph{task\ped{G}} principali e le relative \emph{sottotask\ped{G}}, che rappresentano tali compiti. Queste attività sono state poi assegnate a ciascun membro del team in base al ruolo che ricopre.
    
\end{enumerate}

\subsection{Votazioni} \label{subsec:resvotazione}
\begin{enumerate}
    \item In considerazione delle problematiche riscontrate durante l'apprendimento del framework Angular, si è optato per una transizione verso la libreria React. Tale decisione è stata motivata dalla maggior semplicità di apprendimento di React, come emerso dallo studio individuale, senza che ciò comportasse una diminuzione in termini di efficacia.
    Esito votazione: \\\\
    \begingroup
        \setlength{\tabcolsep}{10pt}
        \renewcommand{\arraystretch}{1.5}
        \rowcolors{2}{oddrow}{evenrow}
        \begin{tabularx}{0.93\textwidth}{| l | l | X |}
             \hline
             \rowcolor{headerrow}\textbf{\textcolor{white}{Proposta}} & \textbf{\textcolor{white}{Sommario}} & \textbf{\textcolor{white}{Mittente}} \\
             \hline
             React & 100\%  & Andrea Cecchin, Marco Dolzan, Francesco Ferraioli, Francesco Giacomuzzo, Leonardo Lago, Giovanni Menon, Anna Nordio\\
             \hline
             Angular & 0\% &  \\
             \hline
             Astenuti & 0\% & \\
             \hline
        \end{tabularx}
    \endgroup
\end{enumerate}

\subsection{Azioni da intraprendere}
{
    \setlength{\tabcolsep}{10pt} 
    \renewcommand{\arraystretch}{1.5}
    \rowcolors{2}{oddrow}{evenrow}
    \begin{xltabular}{\textwidth}{| l | X | X | X |}
         \hline
         \rowcolor{headerrow}\textbf{\textcolor{white}{Codice issue}} & \textbf{\textcolor{white}{Assegnatario}} & \textbf{\textcolor{white}{Scadenza}} & \textbf{\textcolor{white}{Descrizione}} \\
         \endfirsthead
         \hline
         \rowcolor{headerrow}\textbf{\textcolor{white}{Codice issue}} & \textbf{\textcolor{white}{Assegnatario}} & \textbf{\textcolor{white}{Scadenza}} & \textbf{\textcolor{white}{Descrizione}} \\
         \endhead
         \hline
         CC-24 & Francesco Giacomuzzo & 2023/11/21 & Aggiornamento introduzione piano di progetto \\
         \hline
         CC-25 & Andrea Cecchin & 2023/12/4 & Aggiornamento analisi dei rischi \\

         \hline
         CC-31 & Giovanni Menon & 2023/11/26 &  Aggiornamento casi d'uso 1-8 \\
         \hline
         CC-32 & Anna Nordio, Francesco Ferraioli, Giovanni Menon & 2023/11/26 & Stesura dei requisiti \\
         \hline
         CC-34 & Francesco Giacomuzzo & 2023/11/21 & Informazioni e ordine del giorno \\
         \hline
         CC-36 & Francesco Giacomuzzo & 2023/11/21 & Resoconto \\
         \hline
         CC-37 & Leonardo Lago & 2023/11/21 & Bug fix nello script del versionamento \\
         \hline
         CC-38 & Francesco Ferraioli & 2023/11/26 &  Aggiornamento casi d'uso 9-16\\
         \hline
         CC-39 & Anna Nordio & 2023/11/26 &  Aggiornamento casi d'uso 17-26    \\
         \hline
    \end{xltabular}
}


\subsection{Prossima riunione} \label{subsec:riunione}
Viene fissata una riunione per giovedì 23 novembre.