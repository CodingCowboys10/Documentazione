\section{Resoconto} \label{sec:resoconto}
\subsection{Discussioni} \label{subsec:resdiscussione}
\begin{enumerate}
    \item La riunione è iniziata con la fase di retrospettiva, nella quale il gruppo ha esaminato il precedente sprint, valutando il raggiungimento degli obiettivi e pensando a cosa può essere migliorato del \ccgloss{way of working} collettivo. Nonostante la maggior parte  dei documenti abbiano raggiunto un buon grado di maturità, con l'eccezione del Piano di qualifica ancora in una fase iniziale, e la presentazione di una prima versione del \ccgloss{PoC} fissata per mercoledì 6 dicembre quasi pronta alla revisione con il \ccgloss{proponente}, è emersa la difficoltà, da parte di molti membri, nel comunicare e coordinarsi con gli altri colleghi. Viene sottolineata l'importanza di utilizzare \ccgloss{Jira} nel modo opportuno,  in quanto la corretta coordinazione delle attività del gruppo è ben raggiungibile grazie a tale strumento.
    \item Viene discussa la possibilità di chiedere al professore Cardin, \ccgloss{committente} del progetto, un incontro per discutere alcuni dubbi riguardanti il contenuto del documento di analisi dei requisiti. Un colloquio sarebbe senz'altro utile a comprendere la correttezza del documento, e dei prodotti dell'attività di analisi.
    \item Coerentemente a quelli che sono gli obiettivi del gruppo per il prossimo sprint, vengono assegnati i primi ticket e i ruoli che ogni membro è chiamato a rivestire nelle prossime due settimane.
\end{enumerate}

\subsection{Azioni da intraprendere}
{
    \setlength{\tabcolsep}{10pt}
            \renewcommand{\arraystretch}{1.5}
            \rowcolors{2}{oddrow}{evenrow}
            \begin{xltabular}{\textwidth}{| l | X | l | X |}
                 \hline
                 \rowcolor{headerrow}\textbf{\textcolor{white}{Codice issue}} & \textbf{\textcolor{white}{Assegnatario}} & \textbf{\textcolor{white}{Scadenza}} & \textbf{\textcolor{white}{Descrizione}} \\
                 \hline
                 \endhead
                 CC-63 & Andrea Cecchin & 2023/12/05 & Redazione verbale 4 dicembre. \\
                 \hline
                 CC-64 & Francesco Giacomuzzo & 2023/12/10 & Progettazione database per PoC. \\
                 \hline
                 CC-65 & Andrea Cecchin & 2023/12/12 & Connessione database \ccgloss{back-end}. \\
                 \hline
                 CC-66 & Francesco Ferraioli & 2023/12/12 & Query al database back-end. \\
                 \hline
                 CC-67 & Andrea Cecchin & 2023/12/12 & Visualizzazione contenuto database \ccgloss{front-end}. \\
                 \hline
                 CC-52 & Anna Nordio, Marco Dolzan& 2023/12/07 & Fix della chat e dimensione messaggi. \\
                 \hline
                 CC-56 & Giovanni Menon & 2023/12/06 & Collegamento front e back-end. \\
                 \hline
                 CC-26 & Leonardo Lago & 2023/12/07 & Stesura 3° \ccgloss{preventivo} e 2° \ccgloss{consuntivo}. \\
                 \hline
                 CC-68 & Giovanni Menon & 2023/12/11 & Definizione metriche per qualità di processo Pdq. \\
                 \hline
                 CC-69 & Marco Dolzan & 2023/12/11 & Definizione metriche per qualità di prodotto Pdq. \\
                 \hline
                 CC-70 & Andrea Cecchin & 2023/12/11 & Definizione test di sistema Pdq. \\
                 \hline
                 \rowcolor{white} \caption{Assegnazione primi ticket del terzo sprint}
            \end{xltabular}
}


\subsection{Prossima riunione} \label{subsec:riunione}
Viene fissata una riunione per giovedì 7 dicembre.