\section{Resoconto} \label{sec:resoconto}
\subsection{Discussioni} \label{subsec:resdiscussione}
\begin{enumerate}
    \item L'incontro si apre con la discussione sullo stato attuale del \ccgloss{PoC}, riguardo le nuove funzionalità integrate. \\Il gruppo si è concentrato in particolare su un possibile bug riscontrato da uno dei componenti, il quale però parrebbe un falso positivo. \\Si è discusso della funzionalità di un pulsante all'interno della Interfaccia Utente con relativa chiarifica. \\Si è discusso di futuri miglioramenti applicabili al prodotto attuale per colmarne i punti deboli.
    \item Viene successivamente discusso cosa esporre alla riunione del giorno seguente con il Committente e docente, il professor Cardin. \\Sono state redatte delle domande in forma scritta in modo da facilitare l'esposizione.
    \item Viene fatto un resoconto dello stato dei documenti e del software da presentare nella successiva revisione.\\L'esito di questa analisi è stato positivo in quanto i documenti sono in generale pronti e necessitano poche integrazioni, in particolare: \\Le Norme\_di\_progetto necessitano solo un rapido lavoro di uniformazione delle sezioni. \\Il Piano\_di\_Qualifica necessita la stesura dell'introduzione di una sezione e l'aggiornamento dei dati del cruscotto. \\L' Analisi\_dei\_Requisiti è completa e necessita solo le eventuali modifiche dettate dell'incontro prima descritto. \\Il Piano\_di\_Progetto è correttamente aggiornato e non richiede modiche. \\Il Glossario è pressoché ultimato, si sta ultimando l'automatismo per i pedici, e sono necessarie solo delle piccole modifiche interne alla struttura del documento.\\Il PoC è in ottimo stato, sono necessari piccoli aggiustamenti alla connessione tra i due database.
\end{enumerate}


\subsection{Prossima riunione} \label{subsec:riunione}
Viene fissata una riunione per la settimana successiva.