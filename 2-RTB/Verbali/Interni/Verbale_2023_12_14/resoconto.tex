\section{Resoconto} \label{sec:resoconto}
\subsection{Discussioni} \label{subsec:resdiscussione}
\begin{enumerate}
    \item  La riunione è iniziata esaminando il documento di Analisi dei Requisiti, verificando che ogni suo punto rappresentasse effettivamente una funzionalità inclusa nel nostro progetto. Abbiamo successivamente analizzato la sezione relativa ai requisiti funzionali. Notando la natura generica delle indicazione dei \ccgloss{proponenti} in tal senso, abbiamo deciso di approfondire la questione contattandoli via \ccgloss{Slack} per ricevere delle informazioni più precise.
    
    \item È stato effettuato un test con l'indice di Gulpease sul testo contenuto all'interno dei documenti nella \ccgloss{repository}, per avere una metrica di valutazione riguardo alla loro sintassi. Il risultato è stato inserito in un file condiviso su \ccgloss{Drive}. \\
    Riguardo all'indice di Gunnin Fog, inizialmente pensato come metrica utile, è stata presa la decisione di rimuoverlo, in quanto esso non è applicabile alla lingua italiana come Gulpease.

    \item Sono state create delle task relative alle norme di progetto, per completare la stesura del documento. Allo stesso tempo, ulteriori attività sono state assegnate per l'aggiornamento dell'analisi dei requisiti. Questa decisione è stata presa in seguito alle indicazioni del Professor Cardin, emerse durante un colloquio tenutosi martedì 12 dicembre.
    
\end{enumerate}

\subsection{Votazioni} \label{subsec:resvotazione}
\begin{enumerate}
    \item A seguito di un approfondimento del funzionamento di Tailwind e CSS, tecnologie legate alla visualizzazione degli elementi dei siti web, ogni presente è chiamato a dare la propria preferenza tra i due, con il seguente risultato:
 \\\\
        \begingroup
            \setlength{\tabcolsep}{10pt}
            \renewcommand{\arraystretch}{1.5}
            \rowcolors{2}{oddrow}{evenrow}
            \begin{tabularx}{0.93\textwidth}{| l | l | X |}
                 \hline
                 \rowcolor{headerrow}\textbf{\textcolor{white}{Proposta}} & \textbf{\textcolor{white}{Sommario}} & \textbf{\textcolor{white}{Mittente}} \\
                 \hline
                 Tailwind & 85\%  & Andrea Cecchin, Francesco Ferraioli, Francesco Giacomuzzo, Giovanni Menon, Marco Dolzan, Leonardo Lago \\
                 \hline
                 CSS & 15\% &  Anna Nordio\\
                 \hline
                 Astenuti & 0\% & Nessuno \\
                 \hline
            \end{tabularx}
        \endgroup
\end{enumerate}

\subsection{Azioni da intraprendere}
{
    \setlength{\tabcolsep}{10pt}
            \renewcommand{\arraystretch}{1.5}
            \rowcolors{2}{oddrow}{evenrow}
            \begin{tabularx}{\textwidth}{| l | X | l | X |}
                 \hline
                 \rowcolor{headerrow}\textbf{\textcolor{white}{Codice issue}} & \textbf{\textcolor{white}{Assegnatario}} & \textbf{\textcolor{white}{Scadenza}} & \textbf{\textcolor{white}{Descrizione}} \\
                 \hline
                 CC-95 & Anna Nordio & 2023/12/17 & Aggiornamento Use Case. \\
                 \hline
                 CC-96 & Francesco Ferraioli & 2023/12/17 &  Processo di validazione.\\
                 \hline
                 CC-97 & Francesco Giacomuzzo & 2023/12/17 & Processo di verifica.  \\
                 \hline
                 CC-98 & Giovanni Menon & 2023/12/17 & Processo di accertamento qualità.\\
                 \hline
                 CC-99 & Francesco Giacomuzzo & 2023/12/17 & Processo di gestione delle infrastrutture. \\
                 \hline
                 CC-101 &Francesco Giacomuzzo & 2023/12/15 & Stesura Verbale.  \\
                 \hline
            \end{tabularx}
}


\subsection{Prossima riunione} \label{subsec:riunione}
Viene fissata una nuova riunione per lunedì 18 dicembre.
