\section{Resoconto} \label{sec:resoconto}
\subsection{Discussioni} \label{subsec:resdiscussione}
\begin{enumerate}
    \item Considerata la presenza di tutti i membri del gruppo, abbiamo esaminato attentamente il documento dell'Analisi dei Requisiti, verificando che ogni punto analizzato rappresentasse effettivamente una funzionalità inclusa nel nostro progetto. Successivamente, abbiamo collettivamente esaminato la sezione relativa ai requisiti funzionali, notando una generica indicazione da parte dell'azienda sul funzionamento del nostro progetto; abbiamo pertanto deciso di approfondire la questione con la \ccgloss{proponente} attraverso messaggi su \ccgloss{slack}, al fine di ottenere specifiche più dettagliate e quantificabili.
    
    \item E' stato effettuato un test con l'indice di \ccgloss{Gulpease} sul testo contenuto all'interno dei documenti nella\ccgloss{repository}, per avere una metrica di valutazione riguardo alla loro sintassi. Il risultato è stato inserito in un file condiviso su \ccgloss{Drive}.
    
    \item Sono state create delle \ccgloss{task} relative alle norme di progetto per completare la stesura del documento. Allo stesso tempo, ulteriori attività sono stati assegnate per l'aggiornamento dell'analisi dei requisiti. Questa decisione è stata presa in seguito alle indicazioni del Professor Cardin, emerse durante un colloquio tenutosi martedì 12 dicembre.
    
\end{enumerate}

\subsection{Votazioni} \label{subsec:resvotazione}
\begin{enumerate}
    \item A seguito di un approfondimento del funzionamento di \ccgloss{Tailwind} e \ccgloss{CSS}, tecnologie legate alla visualizzazione degli elementi dei siti web, ogni presente è chiamato a dare la propria preferenza tra i due, con il seguente risultato:
 \\\\
        \begingroup
            \setlength{\tabcolsep}{10pt}
            \renewcommand{\arraystretch}{1.5}
            \rowcolors{2}{oddrow}{evenrow}
            \begin{tabularx}{0.93\textwidth}{| l | l | X |}
                 \hline
                 \rowcolor{headerrow}\textbf{\textcolor{white}{Proposta}} & \textbf{\textcolor{white}{Sommario}} & \textbf{\textcolor{white}{Mittente}} \\
                 \hline
                 Tailwind & 85\%  & Andrea Cecchin, Francesco Ferraioli, Francesco Giacomuzzo, Giovanni Menon, Marco Dolzan, Leonardo Lago \\
                 \hline
                 CSS & 15\% &  Anna Nordio\\
                 \hline
                 Astenuti & 0\% & Nessuno \\
                 \hline
            \end{tabularx}
        \endgroup
    \item
\end{enumerate}

\subsection{Azioni da intraprendere}
{
    \setlength{\tabcolsep}{10pt}
            \renewcommand{\arraystretch}{1.5}
            \rowcolors{2}{oddrow}{evenrow}
            \begin{tabularx}{\textwidth}{| l | X | l | X |}
                 \hline
                 \rowcolor{headerrow}\textbf{\textcolor{white}{Codice issue}} & \textbf{\textcolor{white}{Assegnatario}} & \textbf{\textcolor{white}{Scadenza}} & \textbf{\textcolor{white}{Descrizione}} \\
                 \hline
                 CC-95 & Aggiornamento Use Case  & 2023/12/17 & Anna Nordio  \\
                 \hline
                 CC-96 & Processo di validazione & 2023/12/17 & Francesco Ferraioli \\
                 \hline
                 CC-97 & Processo di verifica& 2023/12/17 & Francesco Giacomuzzo \\
                 \hline
                 CC-98 & Processo di accertamento qualità & 2023/12/17 & Francesco Ferraioli\\
                 \hline
                 CC-99 & Processo di gestione delle infrastrutture & 2023/12/17 & Francesco Giacomuzzo\\
                 \hline
                 CC-101 & Stesura Verbale & 2023/12/15 & Francesco Giacomuzzo \\
                 \hline
            \end{tabularx}
}


\subsection{Prossima riunione} \label{subsec:riunione}
Viene fissata una nuova riunione per lunedì 18 dicembre.
