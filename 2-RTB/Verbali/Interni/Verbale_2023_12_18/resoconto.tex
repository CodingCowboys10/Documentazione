\section{Resoconto} \label{sec:resoconto}
\subsection{Discussioni} \label{subsec:resdiscussione}
\begin{enumerate}
    \item La riunione è iniziata con la fase di retrospettiva, nella quale il gruppo ha esaminato i propri ruoli ricoperti durante il precedente \ccgloss{sprint}. La problematica che ha avuto più peso è stata la lentezza della verifica di alcune \ccgloss{pull request} su \ccgloss{GitHub}, che causava ritardi nella terminazione dei compiti e nel passaggio a nuove \ccgloss{task}. Per risolvere questo problema è stato aumentato il numero di verificatori e le ore effettive a loro disposizione, in modo tale da rendere sempre possibile un controllo dei prodotti delle task terminate da inserire nella \ccgloss{repository}.
    \item Come già introdotto nel punto precedente, durante questo sprint il ruolo del verificatore sarà quello più presente, in quanto è necessaria una sicurezza aggiuntiva  riguardo ai documenti prodotti per la candidatura all'\ccgloss{RTB}. In quanto il PoC era già in uno stato tendente al completamento nello sprint precedente, le ore di programmazione e progettazione sono diminuite rispetto alla scorsa pianificazione.\\
    Vi è un caso paragonabile per quanto riguarda i documenti: è stato ridotto il numero di partecipanti del gruppo che si dovrà occupare della loro stesura e del loro completamento.  
    \item La seguente parte della riunione è stata dedicata all'individuazione dei compiti da portare a termine entro la durata dello sprint, i quali riguardano la conclusione della stesura dei documenti, la loro verifica, e un miglioramento della parte grafica del PoC. Vi è inoltre un compito esterno alla repository, che consiste nell'organizzare la struttura del \ccgloss{Drive}, data la numerosa presenza di documenti individuali non inseriti in nelle cartelle. 
    \item E' stato necessario riprogrammare la riunione con l'azienda, in quanto essa sarebbe stata di introduzione all'\ccgloss{MVP}. Non potendo candidarci per l'RTB entro il 25 dicembre, riteniamo che iniziare a pensare all'MVP mentre stiamo lavorando al Proof of Concept non sia una buona pratica.
\end{enumerate}

\subsection{Votazioni} \label{subsec:resvotazione}
     La votazione ha visto come vincitrice l'opzione Edge Runtime : \\\\
        \begingroup
            \setlength{\tabcolsep}{10pt}
            \renewcommand{\arraystretch}{1.5}
            \rowcolors{2}{oddrow}{evenrow}
            \begin{tabularx}{0.93\textwidth}{| l | l | X |}
                 \hline
                 \rowcolor{headerrow}\textbf{\textcolor{white}{Proposta}} & \textbf{\textcolor{white}{Sommario}} & \textbf{\textcolor{white}{Mittente}} \\
                 \hline
                 Edge Runtime & 86\%  & Anna Nordio, Francesco Ferraioli, Francesco Giacomuzzo, Giovanni Menon, Leonardo Lago, Marco Dolzan. \\
                 \hline
                 Next.js Runtime & 0\% &  \\
                 \hline
                 Astenuti & 14\% & Andrea Cecchin. \\
                 \hline
            \end{tabularx}
        \endgroup

%\subsection{Azioni da intraprendere}
%{
%    \setlength{\tabcolsep}{10pt}
 %           \renewcommand{\arraystretch}{1.5}
  %          \rowcolors{2}{oddrow}{evenrow}
   %         \begin{tabularx}{\textwidth}{| l | l | l | X |}
 %                \hline
  %               \rowcolor{headerrow}\textbf{\textcolor{white}{Codice issue}} & \%textbf{\textcolor{white}{Assegnatario}} & \textbf{\textcolor{white}{Scadenza}} & \textbf{\textcolor{white}{Descrizione}} \\
  %               \hline
   %              XX & ... & AAAA/MM/GG & ... \\
    %             \hline
     %       \end{tabularx}
%}


\subsection{Prossima riunione} \label{subsec:riunione}
Viene fissata una nuova riunione per giovedì 21 dicembre.