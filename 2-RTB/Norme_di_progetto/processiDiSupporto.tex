\chapter{Processi di supporto}
\section{Documentazione} \label{sec:documentazione}
\subsection{Definizione}
Il processo di documentazione è un processo il cui scopo è registrare le informazioni prodotte da un processo del ciclo di vita o attività. Il processo contiene l'insieme delle attività che pianificano, progettano, sviluppano, producono, modificano, distribuiscono e conservano i documenti necessari a tutti gli interessati, sia interni che esterni.
\subsection{Attività}
Il processo di documentazione si articola nelle seguenti attività:
\begin{enumerate}
    \item istanziazione del processo;
    \item progettazione e redazione;
    \item produzione;
    \item manutenzione.
\end{enumerate}
\subsubsection{Istanziazione del processo}
L'attività prevede l'identificazione dei vari documenti da redigere e per ognuno di essi la definizione di:
\begin{itemize}
    \item Titolo o nome: Parole chiavi che forniscono una comprensione immediata del contenuto, seguite dal numero di versione.
    \item Scopo: motivo per il quale viene redatto il documento, che viene specificato al proprio interno. I documenti denominati "Verbali" non riportano al loro interno lo scopo generale, ma solamente le ragioni della convocazione della riunione verbalizzata. Questa decisione è stata guidata dalla natura breve e concisa di tali documenti.
    \item Destinatari: a chi il documento vuole rivolgersi inteso sia in ambito interno (i componenti del gruppo), che esterno (committente, proponente, cliente). In base a ciò tutta la documentazione verrà memorizzata in 2 distinte cartelle denominate "Interni" ed "Esterni".
    \item Procedure e responsibilità: chi, all'interno del gruppo, si assume l'impegno e la responsabilità di adempiere ai seguenti oneri:
    \begin{itemize}
        \item Input: riflessione e ragionamento sui contenuti del documento.
        \item Redazione: stesura dei contenuti che si è riscontrato essere congrui al documento.
        \item Verifica: controllo che i contenuti siano stati scritti in lingua corretta, in modo da essere facilmente comprensibili. Viene utilizzato l'indice Gulpease.
        \item Modifica: modifiche correttive e aggiuntive al documento.
        \item Approvazione: controllo che i contenuti siano consoni alla veicolazione dello scopo del documento.
        \item Produzione: l'atto di rendere il testo redatto un documento ufficiale che può essere sottoposto a memorizzazione, distribuzione, manutenzione.
        \item Memorizzazione: il luogo in cui è conservato il documento.
        \item Distribuzione: il modo e i mezzi tramite i quali il documento raggiunge i propri destinatari.
        \item Manutenzione: il modo in cui viene conservato in modo efficiente il documento. 
        \item Gestione della configurazione: il modo in cui viene organizzato e tracciato il documento.
    \end{itemize}
    \item Pianificazione delle versioni intermedie e finali: organizzare il modo ed in che particolari situazioni il documento viene aggiornato.
\end{itemize}

\subsubsection{Progettazione e redazione}
L'attività si articola nei seguenti compiti:
\begin{enumerate}
    \item Ogni documento deve essere progettato in base agli standard adottati per formato, descrizione dei contenuti, numerazione delle pagine, allocazione di immagini e tabelle.
    \item Deve essere verificata la correttezza e l'adeguatezza dei contenuti.
    \item La documentazione redatta deve essere verificata e corretta per formato, contenuto e presentazione. Deve essere inoltre validata e approvata.
\end{enumerate}
\paragraph{Progettazione} 
La progettazione deve quindi:
\begin{enumerate}
    \item Comprendere e analizzare i contenuti necessari al fine di soddisfare lo scopo che il documento si prefigge di raggiungere.
    \item Strutturare il documento in capitoli, sezioni, sottosezioni, sottosottosezioni e paragrafi, in modo da facilitare sia la comprensione e redazione dei contenuti che la navigazione all'interno del documento stesso. Deve essere presente un registro delle modifiche e un indice prima del contenuto, così come l'eventuale elenco delle tabelle e delle figure.
    \item Seguire gli standard di stile e presentazione di seguito elencati:
    \begin{itemize}
        \item La prima pagina deve contenere:
        \begin{itemize}
            \item il logo del gruppo con relativo nome;
            \item il nome dell'Università di Padova e dell'insegnamento di Ingegneria del Software;
            \item il nome del documento;
            \item il numero di versione del documento.
        \end{itemize}
        \item le figure devono essere numerate ed avere una didascalia che ne dia una breve descrizione del contenuto;
        \item le tabelle devono essere numerate ed avere una didascalia che ne dia una breve descrizione del contenuto. L'intestazione delle tabelle deve avere il colore \#006f83 con testo in colore bianco. Le righe di contenuto devono avere il colore \#b2dce3 per le righe dispari e \#7fc5d1 per le righe pari. Qualora la tabella dovesse occupare più pagine, deve essere riportata l'intestazione ad ogni nuova pagina;
        \item il registro delle modifiche deve essere una tabella, senza numerazione e didascalia, che presenta le seguenti colonne:
        \begin{enumerate}
            \item Versione: numero incrementativo assegnato al documento;
            \item Data: data della modifica effettuata, nel formato AAAA/MM/GG (Anno/Mese/Giorno);
            \item Autori: chi ha effettuato la modifica;
            \item Verificatori: chi ha effettuato la verifica della modifica apportata;
            \item Descrizione: breve descrizione della modifica apportata.
        \end{enumerate}
        \item ogni capitolo deve iniziare ad una nuova pagina;
        \item ogni pagina, tranne quelle di che presentano il titolo di un capitolo, deve presentare l'intestazione;
        \item l'intestazione deve contenere a sinistra il logo senza nome del gruppo e a destra il nome del capitolo corrente. Nei documenti denominati "Verbali" a destra deve essere riportato il nome del gruppo (Coding Cowboys) e al centro "Verbale riunione";
        \item ogni pagina deve presentare un piè di pagina;
        \item il piè di pagina deve contenere in centro il numero della pagina corrente e a destra l'indirizzo di posta elettronica del gruppo (codingcowboys.swe@gmail.com). Nei documenti denominati verbali a sinistra deve essere presente la data della riunione;
        \item i documenti denominati "Verbali esterni" devono contenere la firma dei soggetti esterni al gruppo, presenti all'incontro.
    \end{itemize}
\end{enumerate}

\paragraph{Redazione}
La redazione deve seguire i seguenti standard di scrittura:
\begin{itemize}
    \item rispettare le norme tipografiche presenti nell'appendice §A del documento LaTeXpedia utilizzato per l'apprendimento dello strumento \ccgloss{\LaTeX};
    \item i termini e gli acronimi contenuti nel glossario devono essere riportati in corsivo e con a pedice la lettera "G", per evidenziarle rispetto al testo normale;
    \item i riferimenti ai documenti devono contenere la versione alla quale si fa riferimento;
    \item utilizzare un lessico appropriato al contesto di riferimento.
\end{itemize}

\subsubsection{Produzione}
L'attività si articola nei seguenti compiti:
\begin{enumerate}
    \item I documenti devono essere prodotti in un file .pdf tramite la compilazione dei file sorgenti \LaTeX con l'utilizzo dello strumento \TeX. I documenti devono essere forniti ai destinatari attraverso l'esposizione di tali documenti PDF nello strumento \ccgloss{GitHub}. Il compito prevede l'utilizzo di automazioni che facilitino e velocizzino tale pratica.
    \item Devono essere effettuati i controlli necessari in conformità con il processo di gestione della configurazione.
\end{enumerate}
\newpage
\subsubsection{Manutenzione}
L'attività si articola nei seguenti compiti:
\begin{itemize}
    \item Devono essere effettuati compiti necessari alla modifica dei documenti in conformità al processo di manutenzione. Inoltre, i documenti devono essere gestiti in conformità al processo di gestione della configurazione.
\end{itemize}

\subsection{Strumenti}
Durante lo svolgimento dei compiti vengono utilizzati i seguenti strumenti:
\begin{itemize}
    \item \LaTeX per la progettazione e la redazione dei documenti;
    \item \TeX per la compilazione dei file sorgenti in un file PDF;
    \item GitHub per la gestione dei documenti prodotti;
    \item \ccgloss{GitHub Actions} per l'\ccgloss{automazione} del compito di produzione e fornitura;
\end{itemize}

\subsection{Metriche}
Durante il processo vengono utilizzate le seguenti metriche:
\begin{itemize}
    \item indice Gulpease (MPD-1).
\end{itemize}
Per una descrizione di tali metriche si faccia riferimento al documento Piano\_di\_qualifica\_v1.0.
\newpage

\section{Gestione della configurazione} \label{sec:gestconf}
\subsection{Definizione}

Il processo di gestione della configurazione applica procedure amministrative e tecniche durante l'intero ciclo di vita del software. Ha lo scopo di:
\begin{itemize}
    \item identificare, definire e standardizzare gli elementi software in un sistema;
    \item controllare modifiche e rilasci degli elementi;
    \item registrare e segnalare lo stato degli elementi e le richieste di modifica;
    \item garantire la completezza, la coerenza e la correttezza degli elementi;
    \item controllo dell'archiviazione, della gestione e della messa in produzione degli elementi.
\end{itemize}
\subsection{Attività}
Il processo di gestione della configurazione si articola nelle seguenti attività:
\begin{enumerate}
    \item istanziazione del processo;
    \item identificazione della configurazione;
    \item controllo della configurazione;
    %\item stato della configurazione;
    %\item valutazione della configurazione;
    \item rilascio, gestione e messa in produzione.
\end{enumerate}
\subsubsection{Istanziazione del processo}
L'attività si articola nel compito di stabilire un piano per la gestione della configurazione. Il piano prevede l'utilizzo del software di controllo di versione \ccgloss{Git} attraverso la piattaforma GitHub. In particolare gli elementi posti sotto controllo di versione devono essere organizzati in un \ccgloss{repository} pubblico organizzato in cartelle. Nel repository devono essere presenti: 
\begin{itemize}
    \item un file READ.me per la presentazione del repository stesso;
    \item la cartella .github/workflows contenente i file .yml per l'esecuzione delle GitHub Actions;
    \item la cartella 1-Candidatura contenente la documentazione relativa alla candidatura al \ccgloss{capitolato} scelto;
    \item la cartella 2-RTB contenente la documentazione relativa al punto di revisione \ccgloss{Requirment and Technlogy} \ccgloss{Baseline};
    \item la cartella assets contenente immagini utili al file READ.me;
    \item la cartella actions contenente file utili all'esecuzione delle GitHub Actions;
    \item all'interno delle cartelle 1-Candidatura e 2-RTB le cartelle "Documentazione\_interna" e "Documentazione\_esterna".
\end{itemize}
Il repository deve essere organizzato in \ccgloss{branch} per permettere lavoro asincrono e la verifica e la validazione degli elementi al suo interno. Devono essere sempre presenti due branch chiamati "main" e "develop", rispettivamente per il rilascio e per lo sviluppo delle funzionalità. Questi branch devono essere protetti tramite le branch protection rules che limitano e regolano e rendono sistematico l'apporto di modifiche.
Le azioni possibili nel repository sono le seguenti:
\begin{itemize}
    \item Creazione di un branch. Viene creato un branch nel momento in cui si prende in carico una \ccgloss{issue} dal ITS. Il branch deve essere sempre creato a partire dal branch "develop".
    \item \ccgloss{Commit}. Nei branch diversi da "main" e "develop" è possibile creare commit in locale che racchiudono una o più modifiche apportate ad uno o più file.
    \item \ccgloss{Push}. Nei branch diversi da "main" e "develop" è possibile sincronizzare il repository con le modifiche locali poste sotto uno stesso commit, attraverso l'azione di push.
    \item Pull. È possibile sincronizzare la copia locale del repository con i commit presenti solo nel repository remoto attraverso l'azione di pull.
    \item \ccgloss{Pull request}. È possibile richiedere di sincronizzare le modifiche effettuate su un branch con un branch differente attraverso l'azione di pull request. Non è possibile eseguire un'azione di pull request verso il branch "main" tranne dal branch "develop". Ogni pull request per essere approvata deve superare il controllo di un revisore. 
    \item \ccgloss{Merge}. È possibile sincronizzare le modifiche effettuate su un branch con un branch differente attraverso l'azione di merge. L'azione di merge deve essere successiva ad un'azione di pull request approvata;
    \item Release. È possibile effettuare il rilascio del prodotto. L'azione di release deve essere successiva ad un'azione di merge dal branch "develop" al branch "main". 
\end{itemize}

\subsubsection{Identificazione della configurazione}
L'attività si articola nel compito di stabilire uno schema per l'identificazione degli elementi software e non, che devono essere posti sotto controllo di versione. Lo schema utilizzato è il seguente:
\begin{itemize}
    \item è un file .tex o .pdf allora rientra nella documentazione e deve essere all'interno della relativa cartella (1-Candidatura o 2-RTB);
    \item è un file .js, .ts, .tsx, .json e .css allora rientra nel codice e deve essere all'interno della relativa cartella (2-RTB/poc);
    \item è un file .yml allora rientra nelle automazioni di GitHub Actions e deve essere all'interno della cartella .github/\ccgloss{workflows}.
\end{itemize}

\subsubsection{Controllo della configurazione}
L'attività si articola nel compito di identificazione, analisi, valutazione, approvazione, rifiuto, implementazione, verifica e rilascio delle richieste di modifica. Le modifiche agli elementi posti sotto controllo di configurazione devono seguire il seguente schema:
\begin{enumerate}
    \item creazione del branch sul quale apportare le modifiche;
    \item commit delle modifiche;
    \item push dei commit effettuati;
    \item creazione della pull request con relativi verificatori incaricati della verifica;
    \item verifica delle modifiche effettuate che produce due esiti:
    \begin{enumerate}
        \item rifiuto, con conseguente richiesta di modifiche che fa tornare il compito al passo 2;
        \item approvazione delle modifiche;
    \end{enumerate}
    \item esecuzione dei controlli automatici effettuati tramite GitHub Action che produce due esiti:
    \begin{enumerate}
        \item fallimento, con conseguente analisi che fa tornare il compito al passo 2;
        \item superamento dei controlli.
    \end{enumerate}
    \item merge delle modifiche nel branch "develop".
\end{enumerate}
%\subsubsection{Stato della configurazione}
%\subsubsection{Valutazione della configurazione}
\subsubsection{Rilascio, gestione e messa in produzione}
L'attività si articola nel compito di effettuare il rilascio e la messa in produzione in modo controllato. Il rilascio deve essere effettuato attraverso il seguente schema:
\begin{enumerate}
    \item il branch "develop" contiene tutte le modifiche pronte per il rilascio;
    \item pull request dal branch di "develop" verso al branch "main";
    \item validazione delle modifiche effettuate che produce due esiti:
    \begin{enumerate}
        \item rifiuto, con conseguente richiesta di modifiche che fa tornare il compito al punto 1;
        \item approvazione delle modifiche apportate.
    \end{enumerate}
    \item merge delle modifiche nel branch "main".
    \item release del branch "main".
\end{enumerate}
\subsection{Strumenti}
Durante lo svolgimento dei compiti vengono utilizzati i seguenti strumenti:
\begin{itemize}
    \item Git per il controllo di versione;
    \item GitHub come piattaforma per l'utilizzo di Git;
    \item GitHub Actions per l'esecuzione di controlli automatici.
\end{itemize}
%\subsection{Metriche}

\newpage


\section{Accertamento della qualità} \label{sec:qualità}

\subsection{Definizione}
Il processo di accertamento della qualità ha l'obiettivo di garantire che i processi e i prodotti tutti, durante il ciclo di vita del progetto, siano conformi ai requisiti di qualità stabiliti. L'accertamento della qualità deve utilizzare i risultati di altri processi di supporto, come Verifica, Validazione, Revisioni congiunte con il cliente, Verifiche ispettive e Risoluzione dei problemi.

\subsection{Attività}
Il processo di accertamento della qualità si articola nelle seguenti attività:
\begin{enumerate}
    \item istanziazione del processo;
    \item qualità del prodotto;
    \item qualità del processo;
    \item accertamento della qualità del sistema.
\end{enumerate}
\subsubsection{Istanziazione del processo}
L'attività si articola nei seguenti compiti:
\begin{itemize}
    \item Deve essere stabilito un sistema qualità adattato al progetto. Gli obiettivi del processo di accertamento della qualità devono consistere nell'assicurare che i prodotti, sia software che documentali, e i processi impiegati per fornire tali prodotti siano conformi a requisiti prestabiliti e aderiscano ai loro piani stabiliti.
    \item L'intero processo deve essere documentato tramite la redazione di un apposito documento, denominato Piano\_di\_qualifica. Nel suddetto documento dovrà essere riportato:
    \begin{enumerate}
        \item il piano di qualità, contenente le attività mirate a definire gli obiettivi di qualità di prodotti e processi, con definizione di obiettivi e metriche relative alla misurazione della qualità;
        \item il controllo della qualità, contenente la misurazione, la valutazione e l'accettazione dei prodotti e dei processi attivi nel progetto, tramite la realizzazione di un cruscotto di valutazione.
    \end{enumerate}
    
    \item Il processo di accertamento della qualità deve essere coordinato rispettivamente con i processi di Verifica, Validazione, Revisioni congiunte, Verifiche ispettive e Risoluzione dei problemi. Le evidenze sulla qualità dei processi e dei prodotti portate dagli altri processi di supporto devono ricadere nelle decisioni prese nella realizzazione dell'Accertamento della qualità.
    \item Quando vengono rilevati problemi o non conformità, persistenti nel tempo, rispetto ai requisiti di qualità stabiliti, questi devono essere documentati e fungere da input per il processo di Risoluzione dei problemi.
\end{itemize}


\subsubsection{Qualità del prodotto}
L'attività si articola nei seguenti compiti:
\begin{itemize}
    \item Devono essere definiti obiettivi di qualità per prodotti software e documenti. Ogni obiettivo porterà all'individuazione di opportune metriche, documentate riportando:
    \begin{enumerate}
        \item codice alfanumerico della metrica di prodotto, definito come MPD-[\textit{N}], dove \textit{N} identifica il numero di quella metrica di qualità di prodotto;
        \item processo relativo al prodotto;
        \item formula con cui giungere al valore di tale metrica;
        \item descrizione generale;
        \item valore accettabile;
        \item valore preferibile.
    \end{enumerate}
    \item Si deve garantire che i prodotti software e la relativa documentazione siano conformi agli obiettivi qualitativi, tramite misurazione e valutazione degli stessi rispetto le attese. Le misurazioni devono essere effettuate con costanza, ad ogni produzione o modifica di prodotti software e documentali, utilizzandole nell'attività di accertamento della qualità del sistema.
    \item Devono essere definiti i test che porteranno alla validazione e accettazione del prodotto del progetto. Tali test devono essere definiti e documentati secondo quanto già stabilito all'interno del processo primario di Sviluppo.
    \item In preparazione alla consegna dei prodotti software, si deve garantire che questi abbiano pienamente soddisfatto i requisiti contrattuali e siano accettabili per l'acquirente.
    \item In presenza di valori di qualità insufficienti rispetto le soglie di accettazione definite, dovranno essere intraprese azioni di risoluzione dei problemi.
\end{itemize}


\subsubsection{Qualità del processo}
L'attività si articola nei seguenti compiti:
\begin{itemize}
    \item Devono essere definiti obiettivi di qualità per i processi attivi all'interno del progetto. Ogni obiettivo porterà all'individuazione di opportune metriche, documentate riportando:
    \begin{enumerate}
        \item codice alfanumerico della metrica di processo, definito come MPC-[\textit{N}], dove \textit{N} identifica il numero di quella metrica di qualità di processo;
        \item processo relativo;
        \item formula con cui giungere al valore di tale metrica;
        \item descrizione generale;
        \item valore accettabile;
        \item valore preferibile.
    \end{enumerate}
    \item Si deve garantire che tutti i processi siano conformi agli obiettivi qualitativi, tramite misurazione e valutazione degli stessi rispetto le attese. Le misurazioni devono essere effettuate con costanza, utilizzandole nell'attività di accertamento della qualità del sistema.
    \item In presenza di valori di qualità insufficienti rispetto le soglie di accettazione definite, dovranno essere intraprese azioni di risoluzione dei problemi.
\end{itemize}

\subsubsection{Accertamento della qualità del sistema}
L'attività si articola nei seguenti compiti:
\begin{itemize}
    \item Le misurazioni delle metriche di qualità di prodotti e processi devono portare alla realizzazione di un cruscotto di valutazione della qualità. I dati presenti nel cruscotto devono essere documentati tramite la realizzazione di grafici opportuni.
    \item Deve essere costantemente aggiornato il cruscotto valutativo, riportando l'evoluzione dei valori di qualità misurati nel tempo.
    \item Ogni dato riportato nel cruscotto deve essere opportunamente analizzato e commentato, per identificare quanto prima eventuali problematiche. Le decisioni, prese per contrastare eventuali processi o prodotti non conformi agli obiettivi di qualità, devono essere prese sulla base delle evidenze portate dall'analisi critica del cruscotto.
    \item Le valutazioni e le decisioni prese a seguito dell'analisi dei valori riportati nel cruscotto, devono portare ad un miglioramento continuo del sistema qualità del gruppo.
\end{itemize}

\subsection{Strumenti}
Durante lo svolgimento dei compiti vengono utilizzati i seguenti strumenti:
\begin{itemize}
    \item Microsoft Excel per la realizzazione dei grafici da riportare nel cruscotto di valutazione della qualità;
    \item Il sito \url{https://www.webandmultimedia.it/site/index.php} per il calcolo automatico dell'indice di Gulpease di un documento.
\end{itemize}
\newpage


\section{Verifica} \label{sec:verifica}
\subsection{Definizione}
Il processo di verifica è un processo per determinare se i prodotti di un'attività soddisfano i requisiti o le condizioni imposte su di essi nelle attività precedenti. La verifica ha l'obiettivo di accertare la correttezza, la completezza e la coerenza degli elementi del progetto, sia che si tratti di documentazione che codice software. Questo processo deve essere integrato il prima possibile dato che risulta fondamentale per identificare e correggere eventuali errori o discrepanze in modo tempestivo, riducendo il rischio di problemi futuri e garantendo la qualità complessiva del lavoro svolto nel contesto del progetto. La verifica può coinvolgere attività come analisi, revisioni e test che possono essere sia manuali che automatiche.

\subsection{Attività}
Il processo di verifica si struttura in due fasi essenziali:

\begin{enumerate}
    \item istanziazione del processo;
    \item verifica. 
    
\end{enumerate}
\subsubsection{Istanziazione del processo}
Questa attività consiste nei seguenti compiti:
\begin{itemize}
    \item Deve essere effettuata una analisi della quantità di sforzo di verifica necessario per lo svolgimento del progetto. I requisiti del progetto devono essere analizzati per criticità, che può essere valutata in termini di fondi e risorse necessarie, e potenzialità di errori non rilevati.
    \item La verifica deve essere attuata ad ogni introduzione o modifica di prodotti, documentali o software, prima di procedere al rilascio nel repository. 
    \item Sulla base dei compiti di verifica determinati, deve essere sviluppato un documento denominato Piano di Qualifica, il quale deve determinare gli standard qualitativi dei prodotti che guideranno la verifica.
    \item Il piano di verifica deve essere implementato. I problemi e le non conformità rilevati dal processo di verifica devono essere utilizzati nel processo di Risoluzione dei Problemi (\ref{sec:risoprob}).
\end{itemize}

\subsubsection{Verifica}
Questa attività consiste nei seguenti compiti:
\begin{itemize}
    \item Si deve individuare la tipologia di prodotto da verificare, seguendo le seguenti divisioni:
    \begin{itemize}
        \item Documentale;
        \item Software.
    \end{itemize}
    \item Si deve effettuare la verifica unicamente sulle parti del prodotto, software o documentale, che hanno subito modifica.  Esse sono indicate dallo strumento di controllo di versione,  predisposto dal processo Gestione della configurazione.
    \item Il verificatore deve, una volta ricevuto l'incarico, analizzare le differenze con la versione precedente e/o i file aggiunti e verificarne aspetti diversi in base alla tipologia:
    \begin{itemize}
        \item Documentale, andando a verificare:
        \begin{enumerate}
            \item la correttezza sintattica, ortografica e grammaticale del documento;
            \item la correttezza delle informazioni inserite;
            \item la consistenza delle informazioni rispetto a quelle già presenti e future.
        \end{enumerate}
        \item Software, andando a verificare:
        \begin{enumerate}
            \item la mancanza di errori e avvisi da compilatore;
            \item coerenza del nome delle variabili e delle funzioni e dell'identazione del codice;
            \item utilità del codice inserito;
            \item efficienza del codice;
            \item la mancanza di ridondanza nel codice;
            \item comportamento del codice a runtime. 
        \end{enumerate}
    \end{itemize}
    \item L'attività di verifica può portare a due risultati:
    \begin{itemize}
        \item Sono stati riscontrati dei problemi.\\
        Il verificatore è chiamato a lasciare un commento, tramite apposita funzione di \ccgloss{GitHub}, al redattore di quel documento/codice, richiedendone le opportune modifiche.
        \item Non è stato riscontrato alcun problema.\\
        Il verificatore valida le modifiche approvandole secondo ciò che è stabilito nel processo di Gestione della configurazione.
    \end{itemize}
\end{itemize}

\subsection{Strumenti}
Durante lo svolgimento dei compiti vengono utilizzati i seguenti strumenti:
\begin{itemize}
    \item Microsoft Word per verificare la correttezza ortografica e sintattica;
    \item \ccgloss{GitHub} per visualizzare gli elementi modificati e apporre commenti alle sezioni da modificare.
    \item \ccgloss{Web Storm}per la verifica del codice e controllo di funzionamento.
\end{itemize}
\newpage

\section{Validazione} \label{sec:validazione}
\subsection{Definizione}
Il processo di validazione è finalizzato a confermare che il prodotto sviluppato durante il ciclo di vita del progetto soddisfi le esigenze specificate e le aspettative dell'utente finale, ovvero l'azienda proponente. Questo processo assicura la correttezza, l'idoneità e l'adeguatezza del prodotto rispetto ai requisiti stabiliti.
\subsection{Attività}
Il processo di validazione si articola nelle seguenti attività:
\begin{enumerate}
    \item istanziazione del processo;
    \item validazione.
\end{enumerate}
\subsubsection{Istanziazione del processo}
 L'attività si articola nel compito di stabilire le procedure, i protocolli e le modalità operative per la validazione del prodotto. È necessario definire le strategie di validazione e pianificare il percorso da seguire per condurre in modo efficace la validazione. Un test di accettazione consiste in una serie di azioni che devono essere eseguite con successo da un attore specifico.\\
 Tali test devono essere definiti in accordo a quanto previsto dal processo di Sviluppo, e devono essere riportati in un documento denominato Piano di qualifica.
\subsubsection{Validazione}
L'attività si articola nel compito di collaudare il prodotto, effettuando i test di accettazione individuati nella attività di istanziazione, dimostrandone la conformità ai requisiti stabiliti.\\
Prima di richiedere il collaudo ufficiale, i verificatori si dedicano all'esecuzione della suite di test di sistema in un ambiente identico a quello di installazione. È imprescindibile che i test di sistema producano esiti positivi prima di procedere con il collaudo, in quanto costituiscono una condizione preliminare essenziale per il suo avvio.


\newpage

\section{Revisioni congiunte con il cliente} \label{sec:revisionicliente}
\subsection{Definizione}
Il processo di revisione congiunta è un processo di supporto finalizzato alla valutazione, da parte del proponente, dei prodotti e dell'operato del gruppo durante la realizzazione del progetto.
\subsection{Attività}
Il processo di revisioni congiunte con il cliente si articola nelle seguenti attività:
\begin{enumerate}
    \item istanziazione del processo;
    %\item revisione di gestione del progetto;
    \item revisioni tecniche.
\end{enumerate}
\subsubsection{Istanziazione del processo}
L'attività si articola nei seguenti compiti:
\begin{enumerate}
    \item Devono essere sostenuti dal gruppo degli incontri periodici, numerosi in quantità, con il proponente del progetto, secondo la chiara direttiva di ricerca continua di feedback. La richiesta di calendarizzare una riunione congiunta può essere avanzata sia dal proponente che dal gruppo stesso. Quest'ultimo  dovrà prendere l'iniziativa qualora l'input di fissare nuovi incontri non arrivasse dall'azienda proponente.
    \item Il materiale oggetto della revisione congiunta, richiesto di volta in volta dal proponente, deve essere correttamente prodotto dal gruppo, così da permettere la buona riuscita dell'incontro.
    \item Un ordine del giorno di ogni revisione congiunta dovrà essere stilato dalla parte che ha richiesto l'incontro e contestualmente comunicato alla controparte in tempo utile.
    \item Ogni problematica relativa agli incontri e alle comunicazioni con il proponente deve essere opportunamente oggetto del processo di risoluzione dei problemi.
    \item Ogni documentazione informativa sulle tecnologie fornita dall'azienda proponente dovrà rientrare nel piano di formazione personale dei membri del gruppo;
    \item Al termine della riunione, il gruppo dovrà fornire al proponente un documento finalizzato a verbalizzare quanto accaduto, discusso e deciso dalle parti, secondo quanto riportato dal relativo processo di documentazione. L'apposizione della firma del proponente, ad approvare il documento, è necessaria.
\end{enumerate}
%\subsubsection{Revisione di gestione del progetto}
\subsubsection{Revisioni tecniche}
L'attività si articola nel compito di sostenere con l'azienda proponente delle riunioni con oggetto la presentazione dei prodotti software del gruppo. Il team deve mostrare ad ogni revisione quanto richiesto dalla proponente, così da ricevere feedback sulla conformità del prodotto rispetto alle attese e ai requisiti individuati.
\subsection{Strumenti}
Durante lo svolgimento dei compiti vengono utilizzati i seguenti strumenti:
\begin{itemize}
    \item Google Meet per sostenere video colloqui con il proponente;
    \item \ccgloss{Slack} come mezzo per le comunicazioni asincrone con il proponente.
\end{itemize}

\subsection{Metriche}
Durante il processo vengono utilizzate le seguenti metriche:
\begin{itemize}
    \item indice Stakeholders secondo SEMAT.
\end{itemize}
Per una descrizione di tali metriche si faccia riferimento al documento Piano\_di\_qualifica\_v1.0.
\newpage

\section{Verifiche ispettive interne} \label{sec:verificheispettive}
\subsection{Definizione}
Il processo di verifiche ispettive interne raccoglie le attività finalizzate a verificare la conformità delle attività di progetto, in relazione a quanto stabilito da riferimenti normativi come regolamento del progetto didattico e istruzioni del proponente.
\subsection{Attività}
Il processo di verifiche ispettive interne si articola nelle seguenti attività:
\begin{enumerate}
    \item istanziazione del processo;
    \item verifiche ispettive interne.
\end{enumerate}
\subsubsection{Istanziazione del processo}
Questa attività si articola nei seguenti compiti:
\begin{enumerate}
    \item Verifiche dell'operato del gruppo devono essere svolte in corrispondenza di predeterminate \ccgloss{milestone}. In particolare, una verifica dell'avanzamento del progetto dovrà essere sostenuta in corrispondenza della Requirements and Technology Baseline, della \ccgloss{Product Baseline} e, qualora prevista, del Customer Acceptance, in presenza dei committenti. La verifica ispettiva deve essere opportunamente richiesta dal gruppo tramite una richiesta formale inoltrata a entrambi i committenti, in due momenti diversi. Tale richiesta dovrà avvenire tramite l'invio di una lettera di candidatura alla verifica ispettiva, corredata dai riferimenti necessari per visionare la documentazione e i prodotti del progetto richiesti in quel momento.
    \item Tutti i documenti, i prodotti software e ogni genere di risorsa richiesti dai committenti devono essere opportunamente preparati e condivisi con tali figure, in rispetto a quanto sarà concordato di volta in volta. Per sostenere la verifica ispettiva relativa alla Requirements and Technology Baseline il team è chiamato a presentare:
    \begin{itemize}
        \item Analisi dei requisiti;
        \item Glossario;
        \item Norme di progetto;
        \item Piano di progetto;
        \item Piano di qualifica;
        \item Proof of Concept.
    \end{itemize}
    \item Eventuali problematiche, rilevate durante questi incontri ispettivi, devono essere documentate tramite il relativo processo di supporto di risoluzione dei problemi.
\end{enumerate}
\subsubsection{Verifiche ispettive interne}
Questa attività si articola nel compito di partecipare alle verifiche ispettive richieste. Le modalità sono concordate con i committenti, per accertare la correttezza, la qualità e la conformità dei prodotti del gruppo rispetto a quanto stabilito dal regolamento del progetto didattico e dagli altri riferimenti normativi.
\subsection{Strumenti}
Durante lo svolgimento dei compiti vengono utilizzati i seguenti strumenti:
\begin{itemize}
    \item Zoom per sostenere video colloqui con i committenti.
\end{itemize}
%\subsection{Metriche}
\newpage

\section{Risoluzione dei problemi} \label{sec:risoprob}
\subsection{Definizione}
Il processo di risoluzione dei problemi comprende le attività mirate ad anticipare, analizzare e risolvere i problemi che intercorrono in un progetto, siano essi di qualunque natura. L'obiettivo del processo è giungere ad una soluzione per ogni problema nel minor tempo possibile, minimizzando così ogni possibile danno nella normale realizzazione del progetto.
\subsection{Attività}
Il processo di risoluzione dei problemi si articola nelle seguenti attività:
\begin{enumerate}
    \item istanziazione del processo;
    \item risoluzione dei problemi.
\end{enumerate}
\subsubsection{Istanziazione del processo}
L'attività si articola nel compito di stabilire un processo di risoluzione dei problemi, basato su un'attenta analisi dei rischi riscontrabili durante il progetto. Questa analisi, finalizzata ad anticipare ogni possibile problema e opportunamente documentata all'interno del Piano di progetto secondo quanto stabilito dal processo di fornitura, deve essere sviluppata identificando per ogni rischio:
\begin{enumerate}
    \item categoria: rischio organizzativo o tecnologico;
    \item codice identificativo;
    \item descrizione;
    \item probabilità di occorrenza, stabilita su una scala di cinque valori:
    \begin{itemize}
        \item alta: il valore maggiore, indica una quasi certa probabilità che il rischio occorra;
        \item medio-alta;
        \item media: il valore intermedio, indica una probabilità media di occorrenza del rischio;
        \item medio-bassa;
        \item bassa: il valore minore, indica una quasi nulla probabilità che il rischio occorra.
    \end{itemize}
    \item pericolosità, stabilita su una scala di cinque valori:
    \begin{itemize}
        \item alta: il valore maggiore, indica un elevato impatto negativo nella normale realizzazione del progetto;
        \item medio-alta;
        \item media: il valore intermedio, indica un impatto negativo medio;
        \item medio-bassa;
        \item bassa: il valore minore, indica un bassissimo impatto negativo nella normale realizzazione del progetto.
    \end{itemize}
    \item misura mitigativa da adottare in caso di occorrenza del problema.
\end{enumerate}
Nel caso venga individuato un problema non previsto, deve essere fatta dal gruppo un'analisi attenta e profonda finalizzata all'individuazione di una misura mitigativa da adottare in tempi brevi.
\subsubsection{Risoluzione dei problemi}
L'attività si articola nel compito di adottare la misura mitigativa relativa in caso di occorrenza di un rischio. Una volta che la mitigazione sarà tentata, dovrà essere opportunamente documentata l'efficacia della misura mitigativa adottata, riportando in una opportuna sezione del Piano di progetto:
\begin{enumerate}
    \item Efficacia della misura mitigativa, stabilita su una scala di tre valori:
    \begin{itemize}
        \item alta: la misura mitigativa ha neutralizzato l'impatto del rischio secondo le attese;
        \item media: la misura mitigativa ha almeno in parte neutralizzato l'impatto del rischio, ma è individuabile una misura migliore;
        \item bassa: la misura mitigativa è risultata inadeguata, e sarà quanto prima sostituita da una nuova.
    \end{itemize}
    \item Descrizione del risultato della mitigazione.
\end{enumerate}
In caso una misura mitigativa risulti insufficiente a arginare completamente un problema, il team è chiamato a identificare una nuova contromisura, ripetendo in seguito la fase di valutazione di tale.
%\subsection{Strumenti}
\subsection{Metriche}
Durante il processo vengono utilizzate le seguenti metriche:
\begin{itemize}
    \item Misure di mitigazione insufficienti (MPC-14);
    \item Rischi inattesi (MPC-15).
\end{itemize}
Per una descrizione di tali metriche si faccia riferimento al documento Piano\_di\_qualifica\_v1.0.
