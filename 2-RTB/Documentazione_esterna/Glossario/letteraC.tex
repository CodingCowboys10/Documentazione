\chapter{C}

\section{Capitolato}\label{sec:Capitolati}
Il capitolato è un documento in cui sono descritte nel dettaglio tutte le lavorazioni e le opere che saranno realizzate. Vengono indicate: le modalità e i tempi di esecuzione e consegna delle opere;i costi per l'esecuzione.

\section{Chat history}
Cronologia delle conversazioni precedenti tenute con un interlocutore. L'interlocutore nel prodotto si riferirà sempre ad un \ccgloss{chatbot}

\section{Chatbot}
Un chatbot è un programma che usa \emph{AI\ped{G}} ed \emph{LLM\ped{G}} per capire le domande dei clienti e automatizzare le relative risposte, simulando la conversazione umana.

\section{ChatGPT}
ChatGPT è un \emph{chatbot\ped{G}} basato su intelligenza artificiale e apprendimento automatico sviluppato da \emph{OpenAI\ped{G}} specializzato nella conversazione con un utente umano. La sigla GPT sta per Generative Pre-trained Transformer, ovvero "trasformatore generativo pre-addestrato". 

\section{ChromaDB}\label{sec:Chroma}
ChromaDB è un database open source per l'archiviazione e la ricerca di vettori di embeddings.

\section{Code coverage}
La code coverage è un metodo per misurare la quantità di codice di un programma che è stato effettivamente eseguito durante un test, così da determinare se il codice è stato testato in modo completo ed efficace.

\section{Commit}\label{sec:Commits}
Il comando commit è l'azione che nel software \emph{Git\ped{G}} permette di aggiungere, rimuovere e modificare i file del \emph{repository\ped{G}}.

\section{Committente}\label{sec:Committenti}
Il committente è quel soggetto o organizzazione per conto del quale viene realizzata un'opera o svolto un servizio. In questo \ccgloss{progetto} i committenti sono i professori Tullio Vardanega e Riccardo Cardin.

\section{Consulenza}\label{sec:Consulenze}
É la professione di un consulente, ovvero una persona che, avendo accertata qualifica in una materia, consiglia e assiste il proprio committente nello svolgimento di cure, atti, pratiche o progetti fornendo o implementando informazioni, pareri o soluzioni attraverso le proprie conoscenze e le proprie capacità di problem solving.

\section{Consuntivo}\label{sec:Consuntivi}
Rendiconto, sia delle imprese sia degli enti pubblici, dei risultati di un dato periodo di attività.

\section{CSS}\label{sec:Cascading Style Sheets}
CSS, acronimo Cascading Style Sheets, è un linguaggio usato per definire la formattazione e l'aspetto grafico di documenti HTML, XHTML e XML, come ad esempio i siti e le relative pagine web.