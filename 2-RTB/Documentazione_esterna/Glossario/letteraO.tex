\chapter{O}

\section{Ollama}
Ollama è un framework che fornisce un’API per la creazione, l’esecuzione e la gestione di modelli linguistici (LLM), insieme a una libreria di modelli pre-costruiti che possono essere utilizzati in un ampio ventaglio di applicazioni.

\section{OpenAI}
OpenAI è un laboratorio di ricerca sull'intelligenza artificiale con l'obiettivo di promuovere e sviluppare un'intelligenza artificiale amichevole che benefichi l'umanità. L'azienda si impegna a garantire che lo sviluppo dell'IA avvenga in modo etico, sicuro e trasparente, evitando rischi potenzialmente dannosi.
Inoltre, OpenAI fornisce API per modelli come GPT (Generative Pre-trained Transformer). Queste API consentono agli sviluppatori di utilizzare i modelli di OpenAI, come GPT-3, nelle proprie applicazioni e servizi, sfruttando la potenza dell'IA per compiti vari come generazione di testo, traduzione automatica, risposte a domande e molto altro. L'accesso all'API consente a un'ampia gamma di sviluppatori di integrare funzionalità avanzate di intelligenza artificiale nei loro progetti senza la necessità di sviluppare modelli complessi da zero.

\section{Open Source}
Il software open source è sviluppato attraverso una collaborazione aperta e il suo codice sorgente è disponibile per essere utilizzato, esaminato, modificato e ridistribuito da chiunque. 

\section{ORM}\label{sec:Object-Relational Mapping}
ORM sta per "Object-Relational Mapping". Si tratta di una tecnica di programmazione che consente di interagire con un database relazionale utilizzando oggetti nel linguaggio di programmazione utilizzato, invece di dover scrivere direttamente le query SQL.
