\chapter{M}

\section{Merge}
Il comando git merge in \emph{Git\ped{G}} serve a combinare più sequenze di \emph{commit\ped{G}} in una cronologia unificata. Nei casi d’uso più frequenti, git merge viene utilizzato per combinare due \emph{branch\ped{G}}.

\section{Milestone}
La milestone (letteralmente "pietra miliare") in un \emph{progetto\ped{G}} indica e segnala un importante punto di arrivo nel ciclo di un \emph{progetto\ped{G}}.

\section{MinIO}
MinIO è un sistema di archiviazione di oggetti \ccgloss{open source} progettato per fornire uno storage scalabile, ad alte prestazioni e resiliente, che può essere utilizzato sia in locale che in cloud. Si tratta di un server di archiviazione di oggetti compatibile con l'interfaccia \ccgloss{API} di Amazon S3, che consente agli utenti di archiviare e recuperare dati in modo efficiente attraverso richieste HTTP/HTTPS.

\section{Modello di sviluppo}\label{sec:Modelli di sviluppo}
Un modello è un insieme di specifiche, dimostrate corrette, che descrivono un fenomeno di interesse in modo oggettivo. Un modello di sviluppo è il principio teorico che indica il metodo da seguire nel progettare e nello scrivere un programma.

\section{MVP}\label{sec:Minimum Viable Product}
Acronimo per Minimum Viable Product, ovvero la versione di un prodotto che contiene le caratteristiche minime per l'utilizzo da parte dell'utente.
